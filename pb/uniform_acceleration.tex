\section{Uniform Acceleration}


Having established that different
bodies fall from a given height in
equal times, Galileo asked a second
question, one barely considered by
the ancients:

\begin{indent}
\italic{How does the speed of an object
vary during its fall?}
\end{indent}

This was not an easy question to
investigate in Galileo's day. Things
just happen too fast: a body falling
from a height of one meter strikes
the ground in less than half a second.
Simple observation will not produce
much information of value, so one will
have to be more clever. To show that
objects gain speed as they fall, Galileo
considered what happens when a heavy
block falls upon a stake placed in
the ground, as in the figure below.
This is the same setup as the modern
pile-driving machine which is used to
drive timbers or concrete slabs into
the ground for the foundations of a
building.

\image{http://noteimages.s3.amazonaws.com/jim_images/galileo-pile-driver.jpg}{Pile driver}{align: center, width:400}

As Galileo noted, the higher the block
when it is released, the farther the
stake is driven into the ground. He took
this as evidence that the longer the
fall, the greater the final velocity.
Here are his words on the subject
(\italic{Two New Sciences}, page 163):

\begin{quotation}
But tell me, gentlemen, is it not
true that if a block be allowed to
fall upon a stake from a height of four
cubits and drive it into the earth,
say, four finger-breadths, that coming
from a height of two cubits it will
drive the stake a much less distance;
and finally if the block be lifted
only one finger-breadth how much more
will it accomplish than if merely laid
on top of the stake without percussion?
Certainly very little. If it be lifted
only the thickness of a leaf, the effect
will be altogether imperceptible. And
since the effect of the blow depends
upon the velocity of this striking
body, can any one doubt the motion is
very slow .. whenever the effect is
imperceptible?
\end{quotation}

One could, in theory, develop the
pile-driver idea into an experiment
that would give quantitative information
on velocity. But Galileo took a different
route: slow the motion down. Today this
would be easy. Simply use a high-speed
stroboscopic camera to freeze the
motion at regular intervals. Lacking
such gadgets, Galileo used inclined
planes, as in the figure below.

\image{http://noteimages.s3.amazonaws.com/jim_images/galileo-inclined-plane-2.jpg}{Inclined plane}{width: 400}

The idea was to let a ball slide down
the apparatus, slowing the motion
enough for it to be observed. This is
a good start, but was not enough for
Galileo. He wanted numbers: how far
did the ball slide after one second,
two seconds, three seconds, etc.? To
state the problem this way is to raise
a serious practical issue. How could
one measure seconds? Were seconds
even defined in Galileo's day? ((WATER
CLOCK)) The time-measuring devices of
his era quite primitive, and not able
to measure intervals of time accurate
to a fraction of a second, much less
to a second. What Galileo really wanted
was a way of measurng off hte distance
traveled in equal time intervals. And
for that he had a method — a method
based on his experience as a musician.
Galileo tied gut strings — the kind
used for a viola da gamba — across the
bottom of the trough down which the
ball slid. When the ball hit the gut
string, it would jump ever so slightly,
making a clicking sound. He adjusted
the strings so that the clicks came
at regular intervals, like the sound
of a metronome, or like a musician
practicing his scales. Let's label the
strings A, B, C, D, etc. Galileo then
new that the time it takes for the
ball to slide from A to B is the same
as from B to C, from C to D, etc.
Next, Galileo measured the distances
AB, AC, AD, etc.


Galilei, Galileo, Dialogues Concerning Two New Sciences (New York: Dover)
Translated by Henry Crew and Alfonso de Salvio, pp. 178-179:

\begin{quotation}
A falling body accelerates uniformly:
it picks up equal amounts of speed in
equal time intervals, so that, if it
falls from rest, it is moving twice
as fast after two seconds as it was
moving after one second, and moving
three times as fast after three seconds
as it was after one second.*

onso de Salvio, pp. 178-179:

\medskip
 A piece of wooden moulding or
scantling, about 12 cubits long, half
a cubit wide, and three finger-breadths
thick, was taken; on its edge was cut
a channel a little more than one finger
in breadth; having made this groove
very straight, smooth, and polished,
and having lined it with parchment,
also as smooth and polished as possible,
we rolled along it a hard, smooth, and
very round bronze ball. Having placed
this board in a sloping position, by
raising one end some one or two cubits
above the other, we rolled the ball, as
I was just saying, along the channel,
noting, in a manner presently to be
described, the time required to make
the descent. We repeated this experiment
more than once in order to measure the
time with an accuracy such that the
deviation between two observations never
exceeded one-tenth of a pulse-beat.
Having performed this operation
and having assured ourselves of its
reliability, we now rolled the ball
only one-quarter the length of the
channel; and having measured the time
of its descent, we found it precisely
one-half of the former. Next we tried
other distances, compared the time for
the whole length with that for the
half, or with that for two-thirds,
or three-fourths, or indeed for any
fraction; in such experiments, repeated
a full hundred times, we always found
that the spaces traversed were to each
other as the squares of the times, and
this was true for all inclinations of
the plane, i.e., of the channel, along
which we rolled the ball. We also
observed that the times of descent,
for various inclinations of the plane,
bore to one another precisely that
ratio which, as we shall see later, the
Author had predicted and demonstrated
for them.

\medskip
For the measurement of time, we
employed a large vessel of water placed
in an elevated position; to the bottom
of this vessel was soldered a pipe of
small diameter giving a thin jet of
water which we collected in a small
glass during the time of each descent,
whether for the whole length of the
channel or for part of its length;
the water thus collected was weighed,
after each descent, on a very accurate
balance; the differences and ratios of
these weights gave us the differences
and ratios of the times, and this
with such accuracy that although the
operation was repeated many, many times,
there was no appreciable discrepancy
in the results.

\medskip

Galilei, Galileo, Dialogues Concerning
Two New Sciences (New York: Dover)
1954. Translated by Henry Crew and
Alf
\end{quotation}

\subsection*{References}


\href{https://manifold.umn.edu/read/untitled-7ca18210-217d-40f2-83fe-b0add1d84ede/section/bb88e7f4-7451-4b26-9056-a21f51f49091}{Richard T.W. Arthur: On the
Mathematization of Free Fall:
Galileo, Descartes, and a History of
Misconstrual}

\href{http://galileo.phys.virginia.edu/classes/109.mf1i.fall03/lectures09.pdf}{Michael Fowler: Galileo and
Einstein}

\href{https://plato.stanford.edu/entries/galileo/#3}{Stanford Encyclopedia,
Galileo}

\href{https://www.sciencedirect.com/science/article/pii/0315086074900020}{Stillman Drake,
PDF}

\href{https://philpapers.org/rec/DRAGF}{Stilman Drake: Galileo's
1604 Fragment on Falling
Bodies}

\href{https://www.cambridge.org/core/journals/british-journal-for-the-history-of-science/article/galileo-and-the-problem-of-free-fall/EFEE65B06845C3AEAA847AAC7A635796}{R.H. Naylor: Galileo
and the Problem of Free
Fall}