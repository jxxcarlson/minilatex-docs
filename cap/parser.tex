beginMetadata:
{
    "id": "5ecba0eb-57c7-4cdf-a905-669468ba7b62",
    "documentNumber": 478,
    "author": "jxxcarlson",
    "title": "Parser",
    "path": "cap/parser.tex",
    "tags": [],
    "keyString": "parser a=jxxcarlson cap/parser.tex ",
    "timeCreated": 1611620652891,
    "timeModified": 1612750791725,
    "public": false,
    "collaborators": [],
    "docType": "miniLaTeX",
    "versionNumber": 0,
    "versionDate": 0
}
endMetadata
\xlink{uuid:20d52083-a29e-434a-a255-78a9cc9c69f2}{Notes on Crafting an Interpreter}

\include{435}

\section{Parser}


\innertableofcontents

\subsection{Grammars for Lox Expressions}

\begin{colored}[elm]
expression     → literal
               | unary
               | binary
               | grouping ;

literal        → NUMBER | STRING | "true" | "false" | "nil" ;
grouping       → "(" expression ")" ;
unary          → ( "-" | "!" ) expression ;
binary         → expression operator expression ;
operator       → "==" | "!=" | "<" | "<=" | ">" | ">="
               | "+"  | "-"  | "*" | "/" ;
\end{colored}

A more elaborate grammar that takes into account precedence.

\begin{colored}[elm]
expression     → equality ;
equality       → comparison ( ( "!=" | "==" ) comparison )* ;
comparison     → term ( ( ">" | ">=" | "<" | "<=" ) term )* ;
term           → factor ( ( "-" | "+" ) factor )* ;
factor         → unary ( ( "/" | "*" ) unary )* ;
unary          → ( "!" | "-" ) unary
               | primary ;
primary        → NUMBER | STRING | "true" | "false" | "nil"
               | "(" expression ")" ;
\end{colored}

\subsection{Parser helpers}

The function \ccode{applyMany} below is useful in implementing parsers like \ccode{term} and \ccode{factor.}  This function has the signature

\begin{colored}[elm]
applyMany :: Parser a -> (a -> Parser a) -> Parser a
applyMany p fp = ...
\end{colored}

It operates by first running the parser \ccode{p.}  If that succeeds, it runs \ccode{fp} on the result as many times as it can (zero or more).  Thus it is useful in constructing parsers for productions of the form

\begin{colored}[elm]
a -> b c*
\end{colored}


By way of example, 

\begin{colored}[elm]
term :: Parser Expression
term = applyMany factor termWith
\end{colored}

where

\begin{colored}[elm]
termWith :: Expression -> Parser Expression
termWith expr = do
    op <- termOp
    u <- factor
    return (Binary $ BinaryValue {leftExpr = expr, binop = binaryOpOfToken op, rightExpr = u})
\end{colored}

and

\begin{colored}[elm]
termOp :: Parser Token
termOp = TokenParser.choice "expecting  termOp" [ plus, minus]
\end{colored}

\begin{colored}[elm]
applyMany :: Parser a -> (a -> Parser a) -> Parser a
applyMany p fp =
  Parser $ \s -> case runParser p s of 
    (s1, Left err) -> (s, Left err)
    (s2, Right a) -> runParser (applyMany1 a fp) s2
\end{colored}

\begin{colored}[elm]
applyMany1 :: a -> (a -> Parser a) -> Parser a
applyMany1 a fp = 
  Parser $ \s -> case runParser (fp a) s of
        (s1, Left err) -> (s, Right a)
        (s2, Right a2) -> runParser (applyMany1 a2 fp) s2
\end{colored}