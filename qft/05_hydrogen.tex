beginMetadata:
{
    "id": "d9f4dde7-3357-4205-9f44-d4fd219fc3c6",
    "documentNumber": 38,
    "author": "jxxcarlson",
    "title": "Hydrogen Atom",
    "path": "qft/05_hydrogen.tex",
    "tags": [
        "quantum",
        "physics"
    ],
    "keyString": "hydrogen atom a=jxxcarlson qft/05_hydrogen.tex t=quantum t=physics",
    "timeCreated": 1595360330932,
    "timeModified": 1598548775974,
    "public": true,
    "collaborators": [],
    "docType": "miniLaTeX",
    "versionNumber": 1,
    "versionDate": 1598548832971
}
endMetadata
\italic{\xlink{34}{Notes on Quantum Field Theory}}

\setcounter{section}{5}

\section{Hydrogen Atom}

\innertableofcontents

In this section we give a first analysis of the hydrogen atom which leads from the Schroedinger equation for a wave function assumed to be radially symmetric to a computation of the radius and binding energy.  This is followed by some comments on spectroscopy. 

\subsection{Radial symmetry}



Let us consider the case in which the wave function $\psi(x,y,z)$ only depends on the radius, that is, 
$\psi(x,y,z) = \psi(r)$.  To study such wavefunctions,
we must write the Schroedinger equation

\begin{equation}
-\frac{\hbar^2 }{2m} \nabla^2 \psi  + V\psi = E\psi
\end{equation}

in terms of $r$.  To this, end, observe that

\begin{equation}
\frac{\partial r}{\partial x} = \frac{x}{r}
\end{equation}

After some work we find that

\begin{equation}
\frac{\partial^2\psi}{ \partial x^2} 
 = \frac{ x^2 }{ r^2 }\frac{ d^2\psi }{ dr^2 }
   + \frac{1}{r}\frac{d\psi}{dr} - \frac{ x^2 }{ r^3 }\frac{d\psi}{dr}
\end{equation}

Adding these relations for $x$, $y$, and $z$, we find that

\begin{equation}
\nabla^2 \psi = \frac{ d^2\psi  }{ dr^2 }+ \frac{2}{r}\frac{d\psi}{dr} ,
\end{equation}

which can be rewritten as

\begin{equation}
\nabla^2 \psi = \frac{1}{r}\frac{ d^2 }{ dr^2 }r\psi
\end{equation}

Introduce $u(r) = r\psi(r)$.  Then Schroedinger's equation can be written

\begin{equation}
-\frac{\hbar^2}{2m} \frac{ d^2u }{ dr^2 } + V(r)u(r) = Eu(r)
\end{equation}

\subsection{Bohr radius and binding energy}

In the case of an electron moving in a central potential, we
have the Coulomb potential

\begin{equation}
V(r) = -\frac{q^2}{4\pi \epsilon_0 r}
\end{equation}

where $\epsilon_0 = 8.854\times 10^{-12}$ F/m is the permittivity of the vacuum.footnote:[One farad is the capacitance of a pair of plates for which the potential difference is one volt when loaded with one Coulomb of electrical charge.  A volt is Newton-meter per Coulomb, that is , has the units of Joules per Coulomb.  Working it out, one finds that $e^2$ is of unit Joule-meter.  Then $V = e^2/r$ has the units of Joules, which is correct for  potential energy.  Let us introduce the quantity

\begin{equation}
e^2 = \frac{q^2}{4\pi\epsilon_0}
\end{equation}

so that the law for the potential reads more simply as

\begin{equation}
V(r) = -\frac{e^2}{r}
\end{equation}

When $q = 1.6\times10^{-19}$ C is the change on the electron,

\begin{equation}
e^2 = 2.3\times10^{-28}
\end{equation}

Thus Schroedinger's equation takes the form

\begin{equation}
-\frac{\hbar^2}{2m} \frac{ d^2u }{ dr^2 } - \frac{e^2}{r}u(r) = Eu(r)
\end{equation}

For bound states, the energy is negative, so we introduce $B = -E$, the \term{binding energy}, and write

\begin{equation}
\frac{\hbar^2}{2m} \frac{ d^2u }{ dr^2 } + \frac{e^2}{r}u(r) = Bu(r)
\end{equation}

The binding energy is the energy needed to free the electron from the nucleus, i.e., to ionize the atom.


As a trial solution, we take one that vanishes at the origin and decays exponentially, say

\begin{equation}
u(r) = re^{-\alpha r}
\end{equation}

Then Schroedinger's equation becomes

\begin{equation}
\frac{\hbar^2}{2m}\left(  \alpha^2 r e^{-\alpha 4} - 2\alpha e^{-\alpha r} \right) + e^2 (e^{ -\alpha r }) = Bre^{-\alpha r}
\end{equation}

Equality in the preceding expression can hold if and only if 
the coefficients of  $e^{-\alpha r}$ and  $re^{-\alpha r}$ 
on the left and right hand sides are the same.  For the coefficients of $e^{-\alpha r}$, we find that

\begin{equation}
\alpha = \frac{ me^2 }{ \hbar^2 }
\end{equation}
The inverse of this quantity, 

\begin{equation}
a_0= \frac{ \hbar^2 }{ me^2 }  = 0.527\times 10^{-10}\;m,
\end{equation}

is the \term{Bohr radius}.  To calculate the probability that the electron be found within one Bohr radius, we note that the normalized wave function is $\psi(r) = \sqrt{2\alpha}e^{-\alpha r}$.  That is,

\begin{equation}
\int_0^\infty \psi(r)^2 dr = 1
\end{equation}

Then

\begin{equation}
\int_0^{1/\alpha} \psi(r)^2 dr = 1 - \frac{1}{e^2} = 0.865
\end{equation}

There is an 86% chance at any moment that the electron be found within one Bohr radius of the nucleus.  For two Bohr radii, the probability rises to 98%, and for three, to 99.8%.  The hydrogen atom is unlike a classical particle -- it is not like a billiard ball, but rather has a "fuzzy" radius.  One Bohr radius or two, or whatever multiple of $a_0$ suits your tastes and needs.

For the coefficient of $re^{-\alpha r}$, we find

\begin{equation}
B = \frac{ me^4 }{ 2\hbar^2 }
\end{equation}

One finds that

\begin{equation}
B = 2.2\times 10^{-19}\;\text{Joules} = 13.7\;\text{eV}
\end{equation}

This is the binding energy in Bohr's model, that is, the minimum energy needed to separate the electron in the hydrogen atom from the nucleus.

\subsection{Emission of photons and spectrum}


The ground state energy of the electron in a hydrogen atom is given by

\begin{equation}
U = -\frac{me^4 }{ 2\hbar^2 } = -\frac{13.7}{n^2}\;\text{eV}
\end{equation}

As we shall see in a later section, the energy of the $n$-th state is

\begin{equation}
E_n = -\frac{U}{n^2} = -\frac{13.7}{n^2}\;\text{eV}
\end{equation}

Thus the energy of the photon released when the electron drops from the $n$-th state to the $m$-th state is

\begin{equation}
\Delta E = 13.7\left(\frac{1}{m^2} - \frac{1}{n^2}\right)
\end{equation}

For example, if $m = 1$ and $n = 2$, then $\Delta E = 13.7(1-1/4) = 10.3\;\text{eV}$.  Using the Einstein-Planck relation $H = h\nu$ we find that $\nu = 2.5\times 10^{15} Hz$  The wavelength and frequency of light are related by  $c = \lambda \nu$, where $c$ is the velocity of light.  We find the wavelength to be $\lambda = 1.2\times 10^{-7}\;\text{meter}  =120\;\text{nm}$. To put this in context, red and blue light have wavelengths of 650 nm and 475 nm, respectively, while ultraviolet light ranges from 10 to 400 nm.

Bohr's theory of the hydrogen atom resolved a puzzle that had befuddled the spectroscopists for years -- why id the frequencies (hence associated energies) of spectral lines fall into such odd patterns, e.g., a constant times a reciprocal of a difference of reciprocals of squares.  The table below lists a few of the energy levels, differences thereof, and wavelengths associated to these differences.  The wavelengths are organized into series -- Lyman, Ballmer, Paschen -- after names of the spectroscopists who discovered them.  Energies are given in electron volts and wavelengths in nanometers.

\begin{indent}
\strong{Table 1}
\begin{tabular}{ |c|c|c|c|c| }
n &   E  & (n,1) & (n,2) & (n,3)  \\
1 & -13.70 &   -   &    -   &   -     \\
2 & - 3.43 & 10.27 &   -    &   -   \\
3 & - 1.52 & 12.17 & 1.9   &    -   \\
4 & - 0.50 & 12.84 & 2.56  & 0.666  \\
\end{tabular}
\end{indent}


\begin{indent}
\strong{Table 2}
\begin{tabular}{ |c|c|c|c| }
Lyman & Ballmer & Paschen   \\
n & 1,n &   2,n    & 3,n    \\
2 & 122 &    -      &   -     \\
3 & 103 &   656    &   -     \\
4 &  97 &   486    & 1876   \\
\end{tabular}
\end{indent}


