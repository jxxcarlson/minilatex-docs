\italic{\xlink{34}{Notes on Quantum Field Theory}}

\setcounter{section}{16}


\begin{mathmacro}
\newcommand{\ZZ}[0]{\mathbb{Z}}
\newcommand{\CC}[0]{\mathbb{C}}
\newcommand{\UU}[0]{\mathbb{U}}
\newcommand{\FFF}[0]{\mathcal{F}}
\newcommand{\WW}[0]{\mathbb{W}}
\newcommand{\bra}[0]{\langle}
\newcommand{\ket}[0]{\rangle}
\newcommand{\set}[1]{\{#1\}}
\end{mathmacro}


 \section{Discrete  Fourier  transform}

 \innertableofcontents

Consider  complex-valued  functions  of  the  integers,
$q: \ZZ  \longrightarrow \CC $ ,  which  are  periodic  of  period   $N$ .   Thus   $q(n + N) = q(n)$ .   We  write  function  values  as   $q(n)$  or   $q_n$ .  Periodic  functions  of  period   $N$  may  be   identified  with  functions  on   $\ZZ/N$ ,  the  integers  modulo   $N$ .  Let
  $\UU_n = \set{ e^{2\pi i k/N}}$  denote  the  multiplicative  group  of   $N$  -th  roots  of  unity.
 Via  the  natural  identification   $\ZZ/N \cong \UU_n $ ,    $\ZZ/N$  models  the   $N$  -division  points   of  the  unit  circle.



Define  an  inner  product  on  the  space  of  functions  from   $\ZZ/N$  to   $\CC$  by



\begin{equation}
\label{fft_ip}
\langle q, q' \rangle
  =
\frac{1}{N}\sum_{n \in \ZZ/N} q(n)\bar q'(n)
\end{equation}


We  often  omit  the  set  over  which  the  sum  is  taken,  since   $\ZZ/N$  is  understood.



Let   $\delta_k(n) = \delta_{kn}$ .   These  functions  constitute  an  orthogonal  basis  of   $L^2(S)$ ,  where   $||\delta_k||^2 = 1/N$ .   Consider  also  the  functions



\begin{equation}
u_k(n) = e^{2\pi i kn/N}
\end{equation}


The  inner  product  of   $u_k$  and   $u_\ell$  is  the  sum



\begin{equation}
\bra u_k, u_\ell \ket
 =
\frac{1}{N}\sum_k e^{2\pi k(n-\ell)/N}
\end{equation}


By  the  following  result,   $\bra u_k, u_\ell \ket = \delta_{k, \ell}$ ,  so  that  the   $u_k$  form  an  orthonormal  basis.



\begin{lemma}
Let   $m$  and   $N$  be  integers.  Then
 $$
\frac{1}{N}\sum_k e^{2\pi km/N} = \delta_{m,0}
$$
\end{lemma}


 \strong{Proof.} If   $m = 0$ ,  the  result  is  clear.   If   $m \ne 0$ ,  then  by  the  formula  for  the  geometric  series,



\begin{equation}
\sum_k e^{2i\pi km/N} = \frac{e^{2\pi im} - 1}{e^{2\pi im/N} - 1} =0,
\end{equation}


 \strong{Q.E.D.}

Define  the  Fourier  transform   $\FFF(q)$  to  be  the  function



\begin{equation}
Q(k) = \frac{1}{\sqrt N} \sum_n q(n)e^{-2\pi i nk/N}
\end{equation}


and  define  the  inverse  Fourier  transform   $\FFF^{-1}(Q)$  to  be  the  function



\begin{equation}
q(n) = \frac{1}{\sqrt N} \sum_k Q(k)e^{2\pi i nk/N}
\end{equation}


Then  by  the  lemma  above,



\begin{align}
\FFF^{-1}\FFF(q)(n) &=
\frac{1}{N} \sum_k \sum_m q(m) e^{2\pi ik(n - m)/N} \\
&=
\frac{1}{N} \sum_m q(m) \sum_k  e^{2\pi ik(n - m)/N} \\
&= \sum_m q(m)\delta_{nm} \\
&= q(n)
\end{align}


One  shows  that   $\FFF\FFF^{-1}(q)(n) = q(n)$  in  the  same  way.



 \subsection{Discrete  Klein-Gordon  Equation}

In  the  section  on  the  classical  string,  we  considered
 a  chain  of  masses,  aka  the  1-D  crystal,  with  boundary  conditions   $q(0,t) = q(N+1,t) = 0$  for  all   $t$ .   Let  us  reconsider  this  system  with  periodic
 boundary  conditions  on   $N$  masses.   We  also  add  a  "pinning"  term  to  the  Lagrangian:



\begin{equation}
\label{crystallagrangian}
L = \frac{m}{2} \sum_n \dot q(n)^2
- \frac{\kappa}{2}\sum_n [q(n+1) - q(n)]^2
- \frac{s}{2}\sum_n q(n)^2,
\end{equation}


where  the  constants   $m$ ,   $\kappa$ ,  and   $s$  are  positive.
 The  dependence  on   $t$  is  omitted  in  order  not  to  overflow  the  page  with  extraneous  symbols.
 The  new  term  contributes  a  restoring  force  of  strength   $s \ge 0$  which  acts  to  pull  each  atom  to  its  equilibrium  position,  i.e.,  to  "pin"  it  to  that  position.   Let  us  rescale  the  equation,  replacing   $q(n)$  by   $(\sqrt m)q(m)$ .   Set   $\omega^2 = \kappa/m$ ,  and  set   $\sigma^2 = s/m$ .   The  rescaled  Lagrangian  is



\begin{equation}
\label{crystallagrangian2}
L = \sum_n \dot q(n)^2
- \frac{\omega}{2}\sum_n [q(n+1) - q(n)]^2
- \frac{\mu}{2}\sum_n q(n)^2,
\end{equation}


The  equation  of  motion  is  then



\begin{equation}
\label{discrete-wave-equation}
\ddot q(n) = \omega^2[q(n+1) - 2q(n) + q(n-1)] - \mu^2(n) = 0
\end{equation}


The  term  in  brackets  is   $(\Delta^2 q)(n)$ ,  where   $\Delta^2$  is  the  discrete  second  derivative  operator.   It  can  be  factored  as   $\Delta^2 = \Delta_+\Delta_-$ ,  where   $(\Delta_+q)_n = q_{n+1} - q_n$  and
  $(\Delta_-q)_n = q_{n} - q_{n-1}$ .   Thus  we  have



\begin{equation}
\label{discrete-wave-equation2}
\ddot q(n) = \omega^2 (\Delta^2 q)(n) - \mu^2 q(n) = 0,
\end{equation}


or,  more  succinctly,



\begin{equation}
\label{discrete-kgeq}
\ddot q - \omega_0^2 \Delta^2 q + \mu^2 q = 0,
\end{equation}


This
 is  the  discrete  version  of  the  Klein-Gordon  equation   \eqref{kgeq} discussed  in  the  appendix  below.



Let  us  solve   \eqref{discrete-kgeq} using  the  discrete  Fourier  transform.   To  this  end,  consider  the  shift  operators,  defined  by   $(S_{\pm}q)(n) = q(n\pm 1)$ .   Then   $(\Delta_+ q)(n)= [(S_+ - 1 )q](n)$  and    $(\Delta_- q)(n)= [(1 - S_-)q](n)$  Now



\begin{align}
(\FFF S_+q)(k) &= \sum_n  ( S_+ q )(n) e^{-2\pi i kn/N} \\
&= \frac{1}{\sqrt N} \sum_n q(n+1) e^{-2\pi i kn/N} \\
&= \frac{1}{\sqrt N} \sum_n q(n) e^{-2\pi i k(n-1)/N} \\
&= e^{2\pi i k/N} \frac{1}{\sqrt N}  \sum_n q(n) e^{-2\pi kn/N } \\
&=  e^{2\pi i k/N} (\FFF q)(k)
\end{align}


Thus



\begin{equation}
\FFF S_+ = \lambda_+\FFF,
\end{equation}


where   the  function   $\lambda_+$  is  defined  by   $\lambda_+(k) =  e^{ 2\pi ik/N}$ .   A  similar  computation  shows  that



\begin{equation}
\FFF S_- = \lambda_-\FFF,
\end{equation}


where   the  function   $\lambda_-$  is  defined  by   $\lambda_-(k) =  e^{- 2\pi ik/N}$ .



Using  the  factorization



\begin{equation}
-\Delta^2 = - \Delta_+\Delta_- =  (S_+ - 1)(S_-  -  1),
\end{equation}


we  find  that



\begin{equation}
\label{ft-dl}
\FFF(-\Delta^2) = (\lambda_+ - 1)(\lambda_- -1)\FFF =  \lambda^2\FFF,
\end{equation}


for  a  function   $\lambda(k)$  which  we  now  describe.   The  expression
  $(\lambda_+ - 1)(\lambda_- -1)(k)$  evaluates  to
 Since



\begin{equation}
(e^{2\pi i k/N} - 1)(e^{-2\pi i k/N} - 1) = 2(1 - \cos2\pi k/N),
\end{equation}


so  that



\begin{equation}
\lambda^2(k)  = 4(\sin^2 \pi k/N)
\end{equation}


We  can  now  compute  the  Fourier  transform  of  the  equation  of  motion   \eqref{discrete-wave-equation}:



\begin{equation}
\FFF\ddot q - \omega_0^2 \FFF \Delta^2 q + \mu^2 \FFF q = 0
\end{equation}


Write   $Q(k,t)$  for  the  Fourier  transform  of   $q(n,t)$ .   Then
 Use   \eqref{ft-dl} and  the  fact  that  the  Fourier  transform  commutes  with  differentiation  with  respect  to   $t$ ,  to  obtain



\begin{equation}
\ddot Q + \omega_0^2 \lambda^2 Q + \mu^2 Q = 0
\end{equation}


Set



\begin{equation}
\omega_k = 2\omega_0 \sin  \frac{\pi k}{N}
\end{equation}


Then  the  Fourier  transform  of  the  equation  of  motion  becomes



\begin{equation}
\label{eq-mo-uncoupled}
\ddot Q(k,t) +  (\omega_k^2 + \mu^2) Q(k,t) = 0,
\end{equation}


This  system  of  ODE's  is   \italic{uncoupled} --  a  reflection  of  the  fact  that  the  Fourier  transform  of   $-\Delta^2$  is  a  multiplication  operator.   Solutions  of   \eqref{eq-mo-uncoupled},  that  is,  Fourier  coefficients  of  the  solution,  are  linear  combinations  of  the  two  functions



\begin{equation}
Q^\pm(k,t) = e^{\pm i\Omega_kt},
\end{equation}


where   $\Omega_k^2 = \omega_k^2 + \mu^2$ .   Thus  we  have  the  mode  expansions



\begin{equation}
q_n^+ (t) =  \sum_k Q^+ (k,t) e^{2\pi i kn/N}
= \sum_k A_k^+ e^{2\pi i kn/N + i\omega_k t }
\end{equation}


and



\begin{equation}
q_n^-(t) =  \sum_k Q^- (k,t) e^{2\pi i kn/N}  =
 \sum_k A_k^- e^{2\pi i kn/N - i\omega_k t }
\end{equation}


The  two  solutions  represent  traveling  waves  on  the  ring  of  atoms,  the  first  moving  counterclockwise,  the  second  moving  clockwise.   The  general  mode  expanstion  is



\begin{equation}
q_n^-(t)   =
 \sum_k A_k^+ e^{2\pi i kn/N + i\omega_k t }
 + \sum_k A_k^- e^{2\pi i kn/N - i\omega_k t }
\end{equation}


.Remark
 If   $q(n,t)$  is  real,  then   $q(n,t) =q(n,t)^*$  and  so



\begin{align}
\sum_k Q(k,t) e^{2\pi i kn/N } &=  \sum_k  Q(k,t)^* e^{-2\pi i kn/N } \\
 &=  \sum_k Q^*(-k, t) e^{2\pi i kn/N }
\end{align}


Therefore  one  has  the  relation



\begin{equation}
Q(-k,t) = Q^*(k,t)
\end{equation}


 \subsection{Lagrangian  and  Hamiltonian}

Let us write  the  Lagrangian   \eqref{crystallagrangian} in  terms  of  the  Fourier  transform.   To  this  end,  make  the  substitutions



\begin{equation}
q(n,t) = \sum_k Q(k,t) e^{2\pi i nk/N}
\end{equation}


For  the  kinetic  energy  term,
 we  have



\begin{align}
\sum_n \dot q(n,t)^2 &=
\small{\frac{1}{N}} \sum_{n,k , \ell} \dot Q(k,t) \dot Q(\ell,t)
e^{2\pi i nk/N + 2\pi i n\ell/N} \\
&= \small{\frac{1}{N}} \sum_{r , k+\ell = r,n} \dot Q(k,t) \dot Q(\ell,t)
e^{2\pi i n(k + \ell)/N } \\
&= \sum_{k} \dot Q(k,t) \dot Q(-k,t)
\end{align}


In  passing  from  the  first  to  the  second  line,  we  re-order  the  sum,  and  in  passing  from  the  second  to  the  third,  we  apply  the  lemma.



The  argument  for  the  potential  is  similar.   Use  the  identity



\begin{equation}
q(n+1,t) - q(n,t) = ((S_+ - 1)q)(n,t),
\end{equation}


where   $S_+$  is  the  shift  operator,  to  rewrite  the  Fourier  transform  of  the  sum



\begin{equation}
\frac{\omega^2}{2}\sum_n (q(n+1) - q(n))^2
\end{equation}


as



\begin{equation}
\frac{\omega^2}{2N} \sum_{n,k,\ell}
(e^{-2\pi i k/N} - 1)
(e^{-2\pi i \ell/N} - 1)
Q(k) Q(\ell)  e^{2\pi i k n/N + 2\pi i \ell n/N}
\end{equation}


Changing  the  order  of  summation,  one  has



\begin{equation}
\frac{\omega^2}{2N} \sum_{r, k + \ell = r, n}
(e^{-2\pi i k/N} - 1)
(e^{-2\pi i \ell/N} - 1)
Q(k) Q(\ell)  e^{2\pi i n(k + \ell) /N}
\end{equation}


The  sum  over   $k + \ell = r$  is  zero  if   $r \ne 0$  and  is   $N$  otherwise.   Therefore  the  sum  reduces  to



\begin{equation}
\frac{\omega^2}{2N}\sum_{k}
(e^{-2\pi i k/N} - 1)
(e^{2\pi i k/N} - 1)
Q(k) Q(-k)
\end{equation}


and  then  to



\begin{equation}
\frac{\omega^2}{2} \sum_{k}
\omega_k^2
Q_k Q_{-k}
\end{equation}


where   $\omega_k = 2\omega_0\sin \pi k/N$  as  above.   Applying  the  by-now-standard  argument  to  the  last  term  in  the  given  Lagrangian,  we  obtain  its  Fourier  transform:



\begin{equation}
\tilde L = \frac{1}{2}\sum_k \dot Q(k) \dot Q(-k)
-  \frac{1}{2}\sum_k (\omega_k^2)Q(k)Q(-k)
\end{equation}


The  canonical  momentum  is



\begin{equation}
\Pi(k) = \frac{\partial \tilde L}{\partial \dot Q(k)} = \dot Q(-k)
\end{equation}


Thus  we  can  rewrite  the  Lagrangian  as



\begin{equation}
\tilde L = \frac{1}{2} \sum_k \Pi(k)\Pi(-k)
-  \frac{1}{2}\sum_k (\omega_k^2) Q(k) Q(-k),
\end{equation}


To  make  the  connection  with  an  assembly  of  independent  harmonic  oscillators  more  evident,  set



\begin{equation}
\Omega_k^2 = \frac{1}{m}\omega_k^2 = \frac{4\kappa}{m}\sin^2 \frac{\pi k}{N}.
\end{equation}


and   set   $s = \sigma^2/m$ .



Then



\begin{equation}
\tilde L = \frac{1}{2} \sum_k \Pi (k) \Pi (-k)
-  \frac{1}{2} \sum_k (\omega_k^2 )Q(k) Q(-k),
\end{equation}


With  the  Lagrangian  in  this  form,  we  can  find  the  Hamiltonian  via  a  Legendre  transform:



\begin{equation}
H = \sum_k \Pi(k)\dot Q(k) - \tilde L
\end{equation}


By  the  definition  of  canonical  momentum,



\begin{align}
\sum_k \Pi(k)\dot Q(k) &= \sum_k \dot Q(-k)\dot Q(k) \\
&=\sum_k \Pi(k)\Pi(-k)
\end{align}


Then



\begin{equation}
H = \frac{1}{2} \sum_k \Pi(k) \Pi (-k)
+  \frac{1}{2} \sum_k  (\omega_k^2) Q(k) Q(-k),
\end{equation}


 \subsection{Appendix:  Klein-Gordon  equation}

Consider  the  Energy-Momentum  relation  discussed  in  the  section  on  special  relativity:



\begin{equation}
E^2 = p^2 c^2 + m^2 c^4
\end{equation}


In  the  system  of  natural  units  where   $\hbar = 1$  and   $c = 1$ ,  this  reads



\begin{equation}
\label{kg-energy-momentum-relation}
E^2 = p^2 + m^2
\end{equation}


Recall  the  dictionary  used  to  translate  relations  of  this  kind  from  classical  mechanics  to  quantum  mechanics:  replace   $E$  by  the  operator   $i\partial/\partial t$  and  replace   $p$  by   $-i\nabla$ .   If  we  do  this,  then  we  find  the  operator  identity



\begin{equation}
\frac{\partial^2}{ \partial t^2} = \nabla^2 - m^2
\end{equation}


Apply  this  identity  to  a  test  function  and  rewrite  as



\begin{equation}
\label{kgeq}
\frac{\partial^2 \psi}{ \partial t^2} - \nabla^2\psi + m^2\psi = 0.
\end{equation}


This  is  the   \term{Klein-Gordon} equation,  the  continuous  version  of   \eqref{discrete-kgeq} discussed  above.   It  has  plane  wave  solutions  of  the
 form



\begin{equation}
\psi(x,t) = e^{i(kx \pm \omega t)}.
\end{equation}


Substitute  into  the  equation  and  cancel  common  factors  to  obtain  the
 dispersion  relation



\begin{equation}
\label{kg-dispersion}
\omega^2(k) = k^2 + m^2
\end{equation}


Notice  that   \eqref{kg-dispersion} is  essentially  the  Fourier  transform  of  the  energy  momentm  relation   \eqref{kg-energy-momentum-relation}

 \subsection{Wave  velocity}

If  the  mass  term   $m$  is  zero,  then  the  Klein-Gordon  equation  is  a  wave  equation  with  propagation  velocity   $c = 1$ .   Recall  that  for  wave  motion,  there  are  two  notions  of  velocity.   The   \italic{phase  velocity}  $v_{phase} = \omega/k$  and  the   \italic{group  velocity}  $v_{group} = d\omega/dk$ .   For  the  Klein-Gordon  equation



\begin{equation}
v_{phase} = \frac{\omega}{k} = \left(1 + \frac{m^2}{ k^2} \right)^{1/2}
\end{equation}


This  quantity  is  greater  than  the  speed  of  light   $c = 1$  whenever  the  mass  is  positive.   We  shall  discuss  this  apparent  paradox  in  a  moment.   The  group  velocity
 is



\begin{equation}
v_{group} = \frac{d\omega}{dk} = \frac{k}{\omega} = \left(1 + \frac{m^2}{ k^2} \right)^{-1/2},
\end{equation}


which  is  always  less  than  or  equal  to   $c =1$ .   About  the  paradox:   there  isn't  one,  because  a  particle  is  a  localized  wave  packet,  and  these  travel  with  the  group  velocity.





\end{document}
