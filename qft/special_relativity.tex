beginMetadata:
{
    "id": "ec668cf1-1c2d-44a8-a075-89c1f716df5c",
    "documentNumber": 45,
    "author": "jxxcarlson",
    "title": "Special Relativity",
    "path": "qft/special_relativity.tex",
    "tags": [
        "quantum",
        "physics"
    ],
    "keyString": "special relativity a=jxxcarlson qft/special_relativity.tex t=quantum t=physics",
    "timeCreated": 1595380574318,
    "timeModified": 1595380574318,
    "public": true,
    "collaborators": [],
    "docType": "miniLaTeX",
    "versionNumber": 1,
    "versionDate": 1598555540003
}
endMetadata

\italic{\xlink{34}{Notes on Quantum Field Theory}}

\setcounter{section}{13}


\section{Special Relativity}

\innertableofcontents

\subsection{Invariance}

The purpose of this note is to develop the material,
both historical and theoretical, needed to understand
the law of proportionality $E = pc$ between energy
and momentum for any massless particle, and in particular
for the photon.

\subsection{Galilean transformations}

Transformations which leave the laws of Nature
invariant have been part of physics since the time
of Galileo.  Later, especially in the work of Emmy Noether,
it was realized that these transformations,
or symmetries, play a much more fundamental role: they
correspond to conservation laws.

Let us begin with a 1-dimensional universe govefrned
by  the laws of classical mechanics.  A \term{Galilean transformation}
is one of the form

\begin{align}
  x' &= x - a - Vt \\
  t' &= t
\end{align}

If we view these equations as defining a transformation
from $(x,t)$-coordinates to $(x',t')$, coordinates, then the
origin of the $x'$ system at time $t = t'$ corresponds to
$x = a + Vt$ in the $(x,t)$ system.  In particular, an object
at rest in the primed coordinates corresponds to an object
moving with velocity $V$ in the unprimed system.  One
could think of the unprimed system as the "Earth" system
along a railroad track and the primed system as the 
coordinate system of a train moving with velocity $V$.

Now consider Newton's equation for a free particle,
\[
    m\frac{d^2x}{dt^2} = 0
\]
The inverse Galilean transformation is

\begin{align}
  x &= x' + a + Vt' \\
  t &= t'
\end{align}

Substituting these relations into Newton's equation, we find
that
\[
    m\frac{d^2x'}{dt'^2} = 0
\]
The operative laws of physics do not depend on the 
coordinate system used, as long as these coordinate
systems are related by a Galilean transformation.


\subsection{Maxwell's equations}

Let us now consider electrodynamics, the central characters
of which are the electric field $E$ and the magnetic field
$B$.  These fields are governed by Maxwell's equations:

\begin{align}
 \nabla\cdot E &= 4\pi\rho \phantom{\frac{2}{2}}\\
 \nabla\cdot B &= 0 \phantom{\frac{2}{2}}\\
 \nabla\times E &= - \frac{\partial B}{\partial t} \\
 \nabla\times B &= \epsilon_0J -\mu_0\epsilon_0\frac{\partial E}{\partial t} 
\end{align}

Here $\rho$ is the charge density and $J$ is the current.  
The constants are the \term{permitivity}
$\epsilon_0$ and the 
\term{permeability} $\mu_0$ of  free space:

\begin{align}
\epsilon_0 &= 8.85 \times 10^{-12} \; F/m \\
\mu_0 &= 4\pi 10^{-7} \; N/A^2
\end{align}

These constants can be measured by clever experiments
(see the references) based on fundamental 
laws.  For the permittivity, the key fact
 is Coulomb's law:

\begin{equation}
\label{coulombs_law}
  F = \frac{1}{4\pi \epsilon_0} \frac{Q_1Q_2}{r^2}
\end{equation}

The permittivity is the constant that regulates the force
between two charges $Q_i$ at a given distance $r$.   For the permeability,
the key fact is Ampere's law,

\begin{equation}
\label{amperes_law}
  F = \mu_0 \frac{I_1I_2L}{2\pi d^2}
\end{equation}

The permeability is the constant of proportionality that
regulates the force between two parallel wires of length $L$
separated by a distanced $d$ and carrying currents $I_k$.

Consider
Maxwell's equations in a charge-free, current-free region.  Then

\begin{equation}
\nabla\times\nabla\times E =  \frac{1}{c^2} \frac{\partial^2 E}{\partial t^2}
\end{equation}

where

\begin{equation}
c = (\epsilon_0\mu_0)^{-1/2} = 3\times 10^8\; m/sec
\end{equation}

The components of the left-hand side are expressions in the 
second spatial derivatives of the electric field.  One finds
that 

\begin{equation}
\Delta E =  \frac{1}{c^2} \frac{\partial^2 E}{\partial t^2}
\end{equation}

so that the components of the electric field satisfy the wave
equation, e.g.,

\begin{equation}
 \frac{\partial^2 E_x}{\partial x^2}= \frac{1}{c^2} \frac{\partial^2 E_x}{\partial t^2}
\end{equation}

Solutions are given by

\begin{equation}
 E_x(x,t) = \phi(x - ct),
\end{equation}

where $c$ is the propagation velocity of the waves.  Remarkably,
this constant, derived from the permeability and permitivity
of free space, is also the speed of light, as measured by Roemer
and his successors.

\strong{References}

\begin{enumerate}

\item \href{http://www.santarosa.edu/~lwillia2/42/WaveEquationDerivation.pdf}{Derivation of wave equation from Maxwell's equations} 

\item \href{http://www.amnh.org/education/resources/rfl/web/essaybooks/cosmic/p_roemer.html}{Roemer's measurement of the speed of light}

\item \href{http://www.mathpages.com/home/kmath203/kmath203.htm}{Roemers  Hypothesis}

\item \href{http://www.mathpages.com/rr/s3-03/3-03.htm}{De Mora Luminis} from \italic{Reflections on Relativity} by Kevin Brown

\item \href{http://physics.wooster.edu/JrIS/Files/Moore_Web_article.pdf}{Measuring the magnetic permeability of free space} 

\item \href{http://people.physics.tamu.edu/mcintyre/courses/phys208H/labs/Lab_1.pdf}{Measuring the permittivity of the vacuum: Coulomb's law, the Cavendish experiment }

\item \href{http://mysite.du.edu/~jcalvert/phys/elechome.htm}{Electrostatics at home}

\item  \href{http://ocw.mit.edu/courses/physics/8-02t-electricity-and-magnetism-spring-2005/labs/exp02.pdf}{Measuring the Coulomb force (MIT)}

\end{enumerate}



\subsection{Lorentz transformations}

We now ask: \italic{under what transformations are Maxwell's equations invariant?}   Let us consider the wave equation

\begin{equation}
\frac{\partial^2 \psi}{\partial x^2}  
= 
\frac{1}{c^2} 
\frac{\partial^2 \psi}{\partial t^2}
\end{equation}

From the relations defining the Galilean transformation,
one has

\begin{align}
  \frac{\partial}{\partial x'} &=  \frac{\partial}{\partial x} \\
  \frac{\partial}{\partial t'} &= V \frac{\partial}{\partial x}
  + \frac{\partial}{\partial t} 
\end{align}

and so

\begin{equation}
\frac{\partial^2 \psi}{\partial x'^2}  
-
\frac{1}{c^2} 
\frac{\partial^2 \psi}{\partial t'^2}
= 
\frac{\partial^2 \psi}{\partial x^2}  
-
\frac{1}{c^2} \left[
V^2 \frac{\partial^2 \psi}{\partial x^2}  +  V \frac{\partial^2 \psi}{\partial x \partial t}  +  \frac{\partial^2 \psi}{\partial t^2}
\right]
\end{equation}

The wave equation is \italic{not} invariant under Galilean 
transformations.


`TO BE CONTINUED:` *There is a big gap between the above and the below. In the section below, we derive the expression for the 4-momentum based on what the relativistic momentum is -- a rescaling by $\gamma$ of the classical momentum.*

\subsection{Energy-momentum tensor}





As we have seen, the relativistic momentum of a particle
moving in a 1-D universe is 

\begin{equation}
  p = \gamma mv
\end{equation}

Let us imagine that $p$ is one component of a "relativistic
momentum vector" $P = (a, p)$, where $a$ is to be determined.
Apply a Lorentz transformation so that in the $t', x'$ system,  the momentum $p'$ is zero.  In other words, the primed system is a rest frame for the particle.  Then



\begin{equation}
\begin{pmatrix} a' \\ 0 \end{pmatrix}
=
\gamma \begin{pmatrix} 
    1  & - \beta c  \\
  -\beta/c & \phantom{-}1
\end{pmatrix}
\begin{pmatrix} a \\ p \end{pmatrix}
\end{equation}

and so

\begin{equation}
  a = \frac{cp}{\beta} = \frac{c\gamma mv}{v/c} = \gamma mc^2
\end{equation}

What is the nature of the quantity $\gamma mc^2$?
Since $\gamma$ is dimensionless, it has the same units
as does the kinetic energy $(1/2)mv^2$.  Thus it is an energy of
some kind. To see better what this energy is, write

\begin{equation}
\gamma m c^2 
=  mc^2(1 + \small{\frac{1}{2}} \frac{v^2 }{ c^2} + \cdots)
= mc^2 + \small{\frac{1}{2}} mv^2 + \cdots
\end{equation}

Here $m$ is the rest mass -- the mass of the particle is a frame
of reference in which the particle is stationay, e.g., at the origin.
The first term is an energy due entirely to the energy inherent 
in a particle's mass. The second term is the classical kinetic energy.
We think of $\gamma mc^2$ as the \term{total energy} of a particle, 
and we write $E = \gamma mv^2$.  Thus the 4-momentum 
takes fhe form

\begin{equation}
  P = \begin{pmatrix}E \\ p \end{pmatrix}
\end{equation}



The norm-squared of $P$ is the Lorentz scalar

\begin{equation}
  ||P||^2 = \frac{E^2 }{ c^2} - p^2
\end{equation}

Evaluating it in a rest frame, we have

\begin{equation}
  ||P||^2 = - \frac{E^2 }{ c^2}  = m^2 c^2
\end{equation}

so that in general

\begin{equation}
  \frac{E^2 }{ c^2} - p^2  = m^2 c^2
\end{equation}

We can also write this as
\begin{equation}
\label{relativistic_energy_momentum_equation}
E^2 = p^2 c^2 + m^2 c^4
\end{equation}

In the case of a massless particle such as the photon,
the right-hand side vanishes, and so we have

\strong{Energy-momentum proportionality}
\begin{equation}
\label{energy_momentum_proportionality}
\text{For a massless particle,  }
 E = pc.
\end{equation}

Notice that in this case, $P$ is a null-vector for the Minkowski form.