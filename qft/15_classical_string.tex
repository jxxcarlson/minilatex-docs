beginMetadata:
{
    "id": "fae2e72b-17bb-402e-8280-dada35cec527",
    "documentNumber": 47,
    "author": "jxxcarlson",
    "title": "Classical String",
    "path": "qft/classical_string.tex",
    "tags": [
        "quantum",
        "physics"
    ],
    "keyString": "classical string a=jxxcarlson qft/classical_string.tex t=quantum t=physics",
    "timeCreated": 1595381041303,
    "timeModified": 1598555693999,
    "public": true,
    "collaborators": [],
    "docType": "miniLaTeX",
    "versionNumber": 1,
    "versionDate": 1598555720002
}
endMetadata
\setcounter{section}{15}

\section{Classical String}

\innertableofcontents

Consider a string of $N+1$ equal masses connected one to another as in the figure by identical springs.  Let $q_n(t)$ be the displacement from equilibrium at time $t$ of the $n$-th mass.  We can view the system as is, or as a model of a line of atoms confined to move in one dimension -- a 1-D crystal.  The springs then represent the interatomic forces.  Longitudinal waves of compression and rarefication -- sound waves -- propagate in such a system.  An alternative description is to consider a function $q(n,t) = q_n(t)$ of two variables, one discrete, the other continuous.  The discrete variable encodes the position of the $n$-th mass.  If $a$ is the distance between masses when they are all at equilibrium, then the position of the $n$-th mass is $x_n(t) = na + q_n(t)$. 
  




\subsection{ Lagrangian and equations of motion}

The Lagrangian of the 1-D crystal takes the form

\begin{equation}
L = T - V = \small{\frac{1}{2}} m \sum_n \dot q_n^2 - \small{\frac{1}{2}} \kappa \sum_n (q_n - q_{n+1})^2
\end{equation}

The Euler-Lagrange equations

\begin{equation}
\frac{d}{dt} \frac{\partial L}{\partial \dot q_n}
=  \frac{\partial L}{\partial q_n}
\end{equation}

constitute a system of ordinary differential equations

\begin{equation}
m\ddot q_n = - \kappa(q_n - 2q_{n+1}+ q_{n-1})
\end{equation}

which can be written as

\begin{equation}
\label{onedcrystaleqmo}
\ddot q_n + \omega_0^2(q_n - 2q_{n+1}+ q_{n-1}) = 0
\end{equation}

where $\omega_0^2 = \kappa/m$.
Let us examine several approaches to finding solutions of this system.
 




\subsection{Traveling waves}

The right-hand side of the equation of motion \eqref{onedcrystaleqmo} a discrete form of the Laplacian.  Consequently our equation reads: \italic{the second (continuous) derivative of $q$ with respect to $t$ is proportional to the sfecond (discrete) derivative with respect to $n$.} Ignoring the fact that one derivative is continous and the other is discrete, this is a form of the wave equation.  We therefore expect solutions of the form

\begin{equation}
\label{crystaltravelingwaves}
q_n(t) = e^{i(kn - \omega t)}
\end{equation}

where the parameter $k$ is an arbitrary positive real number.
The idea is that $n$, which indexes the masses, is like a spatial variable, and $k$ is like a wave number.

To make the analogy with the wave equation more precise, let $\Delta_+$ and $\Delta_-$ be the forwards  and backwards difference operators,

\begin{align}
  \Delta_+ f_n & = f_{n+1} \\
  \Delta_- f_n & = f_{n-1} \\
\end{align}

Consider the second difference operator

\begin{align}
  \Delta^2 f_n = \Delta_+\Delta_- f_n  = f_{n+1}  - 2f_n + f_{n+1}.
\end{align}

Then the equation of motion reads

\begin{equation}
\label{onedcrystaleqmolaplacian}
\ddot q_n + \omega_0^2 \Delta^2 q_n = 0
\end{equation}

where $\omega_0^2 = \kappa/m$. Were we to replace the second difference operator by a second derivative, this would be a wave equation with propagation velocity $\omega_0$.


Substitute \eqref{crystaltravelingwaves}into the equation of motion to obtain

\begin{equation}
-\omega^2 = \omega_0^2( e^{ik} - 2 + e^{-ik} ).
\end{equation}

Solve for $\omega$ to obtain  

\begin{equation}
\omega_k = \pm 2\omega_0\sin k/2 
\end{equation}

Thus traveling waves with wave number $k$ are given by

\begin{equation}
q_n(t) = e^{ i(kn \pm \omega_k t)}.
\end{equation}

For small wave numbers $k$, that is, long wavelength, one has $\omega_k \sim \omega_0k$, Therefore the long wave propagation velocity is approximately $\omega_0$.

\subsection{Boundary conditions and normal modes}

We now impose the boundary conditions $q_0(t) = 0$, $q_{N+1}(t) = 0$.  Let $q(t) = (q_1(t), \ldots, q_N(t))$ be the vector of generalized coordinates for the masses which are free to move. Write the equation of motion as 

\begin{equation}
\ddot q = \omega_0^2 Aq
\end{equation}

where the matrix $A$ is the discrete Laplacian, which for $N = 4$ is

\begin{equation}
A = 
\begin{pmatrix}
-2 & \phantom{-}1 &  \phantom{-}0 &  \phantom{-}0 \\
 \phantom{-}1 & -2 &  \phantom{-}1 &  \phantom{-}0 \\
 \phantom{-}0 &  \phantom{-}1 & -2 &  \phantom{-}1 \\
 \phantom{-}0 &  \phantom{-}0 &  \phantom{-}1 & -2 
\end{pmatrix}
\end{equation}


The discrete Laplacian is symmetric, negative definite matrix with distinct eigenvalues. Therefore there is a unique orthogonal matrix $C$ such that $C^{-1}(\omega_0^2A)C = -\Omega^2$, where $\Omega$ is a diagonal matrix with entries $\omega_j$.  Let $Q = C^{-1}q$. Then the equation of motion then has the form

\begin{equation}
\ddot Q + \Omega^2 Q = 0
\end{equation}

This is a vector-valued form of the harmonic oscillator equation.  Its component scalar equations are the ODE's

\begin{equation}
\ddot Q_j + \omega_j^2 Q_j = 0, \qquad n = 1, \ldots, N
\end{equation}

Let us call this the \term{normal form} of the equation of motion.
By introducing new coordinates via the rotation $C$, one transforms the given system of identical coupled oscillators into a system of non-identical uncoupled oscillators with angular frequencies $\omega_j$. 

The quantities $Q_j$ are \term{normal coordinates}.  The matrix $C$ relates the given generalized coordinates with the normal coordinates.  The index $j$ is the \term{mode index}.

Let

\begin{equation}
Q^{(j)}_\pm(t) = e^{\pm i \omega_j t} e_j,
\end{equation}

where $e_j$ is the standard basis vector with a 1 in position $j$ and zeroes elsewhere.  Let

\begin{equation}
q^{(j)}_\pm(t) =  CQ^{(j)}_\pm(t) =  e^{\pm i\omega_j t} v_j,
\end{equation}

where $v_j = Ce_j$ is the $j$-the eigenvector of $-\Omega^2$.
Since by construction, $C^{-1}$ transforms solutions of the equation of motion into solutions of the normal form of the  equation of motion, $C$ transforms solutions of the normal form into solutions of the given form.  Therefore $q^{(j)}_\pm(t)$ is a solution of the given form.  It is a \term{normal mode} of the system.  Notice that its components, 
$(q_\pm^{(j)})_n = e^{\pm i\omega_j t} (v_j)_n$ oscillate with common angular frequency $\omega_j$.

The $2N$ vector-valued functions $q^{(j)}_\pm(t)$ are a basis for the space of solutions of the equations of motion.  Thus the general solution can be written

\begin{equation}
q(t) = \sum_j A_j q^{(j)}_+(t)  +   \sum_j B_j q^{(j)}_-(t).
\end{equation}

This is a form of the \term{mode expansion}.




\subsection{Eigenvalues and eigenvectors}

The results of the preceding section show that solutions to.  To find the specific form, we must find the eigenvalues and eigenvectors of the matrix $\omega_0^2A$, or equivalently, of the discrete Laplacian.  We give a standard physics argument for finding them.  To this end, write 

\begin{equation}
q_n(t) = a_n e^{i\omega t},
\end{equation}

where both the amplitude $a_n$ and the angular frequency $\omega$ are to be determined.  Substitute into the equation of motion to obtain

\begin{equation}
-\omega^2 a_n e^{i\omega t} 
= \omega_0^2( a_{n+1} - 2a_n + a_{n-1} ) e^{i\omega t}
\end{equation}

Simplifying, we have

\begin{equation}
\label{aneq}
a_{n+1} - 2a_n + a_{n-1} + \left(\frac{\omega}{\omega_0}\right)^2 a_n = 0
\end{equation}

Again, we recognize the discrete Laplacian, so this can be written as a wave equation,

\begin{equation}
\Delta^2 a_n + \left(\frac{\omega}{\omega_0}\right)^2 a_n = 0
\end{equation}

Since we seek solutions with $a_0 = 0$ and $a_{N+1} = 0$,
it is natural to try functions of the form $a_n = \sin(kn)$, where $k$ is a phase factor to be determined. The second condition reads $\sin((N+1)k) = 0$, so that 

\begin{equation}
\label{crystalphasefactor}
k = \frac{j\pi}{(N+1)}
\end{equation}

for an integer $j$ in the range $1\ldots N$.  Therefore, writing $k_j$ for the phase factors in \eqref{crystalphasefactor}, we have

\begin{equation}
a_n = \sin(k_jn) = \sin \frac{nj\pi}{N+1}
\end{equation}


To find $\omega$, substitute $a_n = \sin k_jn$ into \eqref{aneq}to obtain

\begin{equation}
\sin k_j(n+1) - 2\sin k_jn + \sin k_j(n-1)
  + \left(\frac{\omega}{\omega_0}\right)^2  \sin k_jn  = 0
\end{equation}

Using the addition formula for the sine, we find that

\begin{equation}
2\sin k_jn (1 - \cos k_j) 
+ \left(\frac{\omega}{\omega_0}\right)^2 \sin k_jn = 0
\end{equation}
Cancel $\sin k_jn$ and solve for $\omega$ to obtain

\begin{equation}
\omega = 2\omega_0\sin k_j/2
\end{equation}

Making explicit the dependence on the integer $j$, we haveangular frequencies of oscillation,

\begin{equation}
\omega_j = 2\omega_0 \sin\left( \frac{j\pi}{2N+2} \right), 
\qquad j = 1, \ldots, N
\end{equation}

The integers $j$ index the eigenvalues, that is, the normal modes. For small values of $j$, one has

\begin{equation}
\omega_j \sim \frac{\omega_0\pi}{N+1} j
\end{equation}

As for the normal modes themselves, they have the form

\begin{align}
q^{(j)}_\pm(t) &= (q^{(j)}_1(t), \ldots, q^{(j)}_N(t)) \\ 
&= e^{\pm i\omega_j t}(a_1, \ldots, a_n) \\ 
&= e^{\pm i\omega_j t}\left(\sin \frac{j\pi}{N+1}, \sin \frac{2j\pi}{N+1},\ldots, \sin \frac{Nj\pi}{N+1}\right) \\
&=  e^{\pm i\omega_j t} v_j
\end{align}

The components of the normal modes are

\begin{equation}
q^{(j)}_{\pm, n}(t) =  e^{\pm i\omega_j t} \sin k_jn,
\end{equation}

where

\begin{equation}
k_j = \frac{j\pi}{N+1}
\end{equation}

is the wave number for the spatial oscillation of the system.

The quantities $\omega_j$ and $v_j$ are the sought-for eigenvalues and eigenvectors of the discrete Laplacian.  The eigenvector $v_j$ can be though of as the vector of samples of the function $\sin jx$ at the values $x = \pi n/(N+1)$, where $n$ runs from $1$ to $N$.  In other words, the $j$-th normal mode is a kind of sample over $x$ of the standing wave $e^{i\omega_j}\sin jx$.  Note that all components of a normal mode oscillate at the same frequency, and that frequency increases with mode number for the first half of the range, then decreases.



\subsection{Periodic boundary conditions}

It is interesting and useful to consider our chain of masses with periodic boundary conditions.  Since all masses can move, let the mass index $n$ range from 0 to $N-1$
and impose the condition $q_{n+N}(t) = q_n(t)$.  The condition for $q_n(t) = \exp i(kn - \omega t)$ to satisfy the boundary conditions is then $e^{ikN} = 1$, so that the wave number $k$ is one of

\begin{equation}
k_j = \frac{2\pi j}{N}, \qquad j = 0\ldots N-1,
\end{equation}

where $j$ is the mode number.
The angular angular frequency is derived as a function of the mode number as before; one has

\begin{equation}
\omega_j = 2\omega_0 \sin \frac{2\pi j}{N}
\end{equation}

The mode expansion is

\begin{equation}
q_n(t) = \sum_j [ A_j e^{i(k_jn - \omega_jt)} + \overline{A_j} e^{-i(k_jn - \omega_jt)}],
\end{equation}

where we add a solution and its conjugate to obtain a real solution.




\section{Continuum limit}

Let us suppose that the chain of masses under consideration has length $R$ with masses separated by a distance $a$.  Thus $R = Na$.  Let $\sigma = \kappa a$ be the tension, which has the units of force (Newtons).  Let $m = \rho a$, where $\rho$ is the linear mass density (kg/meter).  We may view $q_n(t)$ as a sample of a function $q(x,t)$, where $x = na$.  This function is a \term{scalar field}: to each space-time pair $(x,t)$ is associated a scalar $q(x,t) \in \Bbb{R}$. In the case at hand, $q(x,t)$ is a \term{longitudinal displacement field}.  We can view it as given the longitudinal displacement of mass in a long thin rod. 

\subsection{Lagrangian}

Let us study the limit behavior of the Lagrangian as $N \to \infty$, that is, as $a \to 0$. Write the potential energy term as

\begin{equation}
V = \frac{\kappa}{2} \sum ( q_{n+1} - q_n )^2 
 = \frac{\kappa a}{2} \sum 
\left( \frac{q_{n+1} - q_n}{a} \right)^2 a
\end{equation}

Thinking of $a$ as $\Delta x$, we recognize the term on the right in parentheses as an approximation to $\partial q/\partial x$, and we recognize the sum as an approximation to an integral over $dx$.  Thus

\begin{equation}
V \sim \frac{\sigma}{2}\int_0^R \left(\frac{\partial q}{\partial x} \right)^2 dx
\end{equation}

The kinetic energy term is

\begin{align}
T &= \frac{m}{2} \sum \dot q_n \\
&= \frac{\rho a}{2} \sum \left(  \frac{\partial q}{\partial t} (na) \right)^2 \\
&\sim \frac{\rho}{2} \int_0^R  \left(\frac{\partial q}{\partial t} \right)^2 dx
\end{align}

Thus the limit form of the Lagrangian is

\begin{equation}
L = T - V 
= 
 \int_0^R \left[ \frac{\rho}{2}\left(\frac{\partial q}{\partial t} \right)^2 
 -
\frac{\sigma}{2} \left(\frac{\partial q}{\partial x} \right)^2 \right]dx
\end{equation}

The integrand in the expression above is the \term{Lagrangian density},

\begin{equation}
\mathcal{L} = \frac{\rho}{2}\left(\frac{\partial q}{\partial t} \right)^2 
 -
\frac{\sigma}{2} \left(\frac{\partial q}{\partial x} \right)^2
\end{equation}

For a continuous system, the Lagrangian is the integral of a Lagrangian density:

\begin{equation}
L = \int \mathcal{L} dx .
\end{equation}


\subsection{Euler-Lagrange equations}

The Euler-Lagrange equations for a continuous system are derived via the same principles used in the discrete case.  Let $q(x, t)$ be a scalar field and $\mathcal{L}(q, \partial q/\partial x, \partial q/\partial t)$ a Lagrangian density. The configuration space in question is the space $\mathcal{C}$ of scalar fields, that is, of scalar functions of $(x,t)$.  A Lagrangian density is a scalar function on the tangent bundle of the configuration space.  Let $\delta q(x,t)$ be a scalar field that vanishes on the boundary of the rectangle $[0,L]\times[0,T]$, and consider
the one parameter family of fields $q_s(x,t) = q(x,t) + s\delta q(x,t)$. Compose the Lagrangian density with this one parameter family to obtain a one parameter family of Lagrangian densities

\begin{equation}
\mathcal{L}_s = \mathcal{L}(q_s, \partial q_s/\partial x,  \partial q_s/\partial t)
\end{equation}

Expand this family as

\begin{equation}
\mathcal{L}_s = \mathcal{L} + s\delta \mathcal{L} + O(s^2)
\end{equation}

Then 

\begin{equation}
\delta \mathcal{L} = \frac{\partial \mathcal{L}}{\partial q }
\delta q
+ \frac{\partial \mathcal{L}}{\partial (\partial q/\partial x) }
\frac{\partial \delta q}{\partial x}
+ \frac{\partial \mathcal{L}}{\partial (\partial q/\partial t) }
\frac{\partial \delta q}{\partial t}
\end{equation}
The contribution of this expression to the action

\begin{align}
S[q] &= \int_0^T L dt \\
&= \int_0^T \int_0^R \mathcal{L} dx dt
\end{align}

is

\begin{align}
\delta \mathcal{S} =  \int_0^T \int_0^R
\left[
\frac{\partial \mathcal{L}}{\partial q }
\delta q
+ \frac{\partial \mathcal{L}}{\partial (\partial q/\partial x) }
\frac{\partial \delta q}{\partial x}
+ \frac{\partial \mathcal{L}}{\partial (\partial q/\partial t) }
\frac{\partial \delta q}{\partial t}
\right] dx dt
\end{align}

The integral is a sum of three integrals $A$, $B$, and $C$. The middle integral is

\begin{equation}
B = \int_0^T \int_0^R \frac{\partial \mathcal{L}}{\partial (\partial q/\partial x) }
\frac{\partial \delta q}{\partial x} dx dt .
\end{equation}

Integrate by parts with respect to $x$ to obtain

\begin{equation}
B = - \int_0^T \int_0^R \frac{\partial}{\partial x}\frac{\partial \mathcal{L}}{\partial (\partial q/\partial x) } \delta q \,dx dt .
\end{equation}

Similar considertions show that

\begin{equation}
C = - \int_0^T \int_0^R \frac{\partial}{\partial x}\frac{\partial \mathcal{L}}{\partial (\partial q/\partial t) } \delta q \,dx dt .
\end{equation}

Thus the variation in the action is $\delta S = A + B + C$, or

\begin{equation}
\delta S 
= \int_0^T \int_0^R\left[ 
\frac{\partial \mathcal{L}}{\delta q}
+ \frac{\partial }{\partial x}\frac{\partial \mathcal{L}}{\partial (\partial q/\partial x) } 
+  \frac{\partial}{\partial t}\frac{\partial \mathcal{L}}{\partial (\partial q/\partial t) } 
\right]\delta q \,dx dt .
\end{equation}

For the varation of the action to vanish for all varions of $q(x,t)$, the expression in brackets must vanish:

\begin{equation}
\frac{\partial \mathcal{L}}{\partial q}
+ \frac{\partial }{\partial x}\frac{\partial \mathcal{L}}{\partial (\partial q/\partial x) } 
+  \frac{\partial}{\partial t}\frac{\partial \mathcal{L}}{\partial (\partial q/\partial t) } = 0 .
\end{equation}

This is the Euler-Lagrange equation
In the case of the thin rod, it reads

\begin{equation}
\rho\frac{\partial^2 q}{\partial t^2} 
-  \sigma\frac{\partial^2 q}{\partial x^2}= 0 .
\end{equation}

This is a wave equation,

\begin{equation}
\frac{\partial^2 q}{\partial x^2}
= \frac{\rho}{\sigma}\frac{\partial^2 q}{\partial t^2} ,
\end{equation}

where waves propagate with phase velocity

\begin{equation}
v = \sqrt{\frac{\sigma}{\rho}} .
\end{equation}





\subsection{Momentum and energy}

The momentum density of our system is given by 

\begin{equation}
\pi(x,t) 
= \frac{\partial \mathcal{L}}{\partial (\partial q/\partial t)} 
= \rho \frac{\partial q}{\partial t}
\end{equation}

The energy density is $\pi\dot q - \mathcal{L}$, where both $\pi$, the momentum density, and $\dot q$ are scalar fields.

\begin{equation}
\mathcal{E} 
= \rho \left(\frac{\partial q}{\partial t} \right)^2 - \mathcal{L}
= \frac{\rho}{2}\left(\frac{\partial q}{\partial t} \right)^2 
+
\frac{\sigma}{2} \left(\frac{\partial q}{\partial x} \right)^2
\end{equation}

The momentum is the integral with respect to $x$ of the momentum density

\begin{equation}
p(t) = \int_0^R \pi(x,t) dx .
\end{equation}

Let us find how it changes with respect to time:

\begin{align}
\frac{ dp}{dt} &= \frac{d}{dt} \int_0^R \pi(x,t) dx \\
&= \frac{d}{dt} \rho \int_0^R \frac{\partial q}{\partial t} dx \\
&=  \rho \int_0^R \frac{\partial^2 q}{\partial t^2} dx 
\end{align}

Apply the wave equation to write

\begin{align}
\rho \int_0^R \frac{\partial^2 q}{\partial t^2} dx  
&= \sigma \int_0^R \frac{\partial^2 q}{\partial x^2} dx  \\
&= \sigma \left(  \frac{\partial q}{\partial x}( R, t ) - \frac{\partial q}{\partial x}(0, t) \right)
\end{align}

The last follows from the boundary conditions $q(0, t) = 0$, 
$q(R, t) = 0$ for all $t$.  Thus momentum is conserved only if the quantity on the right is zero.  


Let us now study the variation of energy with time:

\begin{align}
\frac{dE}{dt} &= \frac{d}{dt} \int_0^R \left[
\frac{\rho}{2}\left(\frac{\partial q}{\partial t} \right)^2 
+
\frac{\sigma}{2} \left(\frac{\partial q}{\partial x} \right)^2
\right] dx \\
&= \int_0^R \left[
\rho\frac{\partial q}{\partial t} \frac{\partial^2 q}{\partial t^2} 
+
\sigma \frac{\partial q}{\partial x} \frac{\partial^2 q}{\partial t\partial x} 
\right] dx \\
&= \int_0^R \left[
\rho\frac{\partial^2 q}{\partial t^2} 
-
\sigma \frac{\partial^2 q}{\partial x^2} 
\right] \frac{\partial q}{\partial t} dx  + B\\
&= B
\end{align}

where the boundary term $B$ is given by

\begin{equation}
B = \frac{\partial q}{\partial x}\frac{\partial q}{\partial t}(R,t) 
- \frac{\partial q}{\partial x}\frac{\partial q}{\partial t}(0,t)
\end{equation}

For the third line, we integrate by parts and apply the wave equation.  As in the case of momentum, conservation depends on boundary conditions.

\subsection{Long thin rod}

Analogous to the equation for the continuum limit of a 1-D crystal is the equation of motion for a long thin rod is

\begin{equation}
\frac{\partial^2 q}{\partial x^2}
= \frac{\rho}{Y}\frac{\partial^2 q}{\partial t^2} 
\end{equation}

where $Y$ is Young's modulus, with units of $\text{Newtons/m}^2$ and
$\rho$ is the bulk mass density, with units of $\text{kg/m}^3$.  Let us consider the case of aluminum, for which $Y = 70\times 10^9 N/m^2$ and
$\rho = 1071 kg/m^3$.  The speed of sound is then

\begin{equation}
v = \sqrt{\frac{Y}{\rho}} = 5.09 \times 10^3 \text{m/sec}
\end{equation}

Consider a rod of length $L = 4 \text{ feet} = 1.219 \text{ meters}$
For a rod with free ends, the ends are antinodes of the waves, so that admissible $L = n\lambda/2$, where $\lambda$ is the wavelength and $n$ is an odd integer.  Thus $\lambda = 2L/n$.  Since $v = \lambda \nu$, where $\nu$ is the frequency.  Consequently the frequency is

\begin{equation}
\nu = \frac{vn}{2L} .
\end{equation}

In the case of our rod, the fundamental frequency is $\nu = 2088 \text{ Herz}$, somewhat more than two octaves above the the frequency of a tuning fork.



\subsection{References}

\begin{thebibliography}

\bibitem{Wikipedia}
\href{
https://en.wikipedia.org/wiki/Eigenvalues_and_eigenvectors_of_the_second_derivative}{Eigenvalues of the discrete Laplacian}

\bibitem{Torre}
\href{http://digitalcommons.usu.edu/cgi/viewcontent.cgi?article=1004&context=foundation_wave}{Notes}

\bibitem{Singing}\href{https://courses.physics.illinois.edu/phys193/Lecture_Notes/Vibrating_Rod/Longitudinally_Vibrating_Singing_Rod.pdf}{Illinois notes}

\end{thebibliography}