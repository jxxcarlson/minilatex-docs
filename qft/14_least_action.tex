\italic{\xlink{34}{Notes on Quantum Field Theory}}

\begin{mathmacro}
\newcommand{\bra}[0]{\langle}
\newcommand{\ket}[0]{\rangle}
\newcommand{\caF}[0]{\mathcal{F}}
\newcommand{\caA}[0]{\mathcal{A}}
\newcommand{\caC}[0]{\mathcal{C}}
\newcommand{\caP}[0]{\mathcal{P}}
\newcommand{\boR}[0]{\boldsymbol{R}}
\newcommand{\bor}[0]{\boldsymbol{r}}
\newcommand{\bov}[0]{\boldsymbol{v}}
\newcommand{\boi}[0]{\boldsymbol{i}}
\newcommand{\boa}[0]{\boldsymbol{a}}
\newcommand{\boL}[0]{\boldsymbol{L}}
\newcommand{\bbR}[0]{\mathbb{R}}
\newcommand{\sett}[2]{\{#1\ |\ #2 \}}
\end{mathmacro}

\setcounter{section}{14}

\section{Principle of least action}

\image{http://psurl.s3.amazonaws.com/images/jc/hero_reflection_1-7f3c.jpg}{}{width: 200, float: right}
We have already seen two examples of how minimum principles yield a physical law.  The fact that the angle of reflection is equal to the angle of reflection was explained by xref::61[Hero of Alexandria] to be a consequence of the assertion that the path from source to mirror to eye minimizes distance.  Likewise, Snell's law of refraction was shown by Fermat to follow from a principle of xref::59[least time]. These to examples, one from around 50 AD, the other from the 17th century, lead one to ask: \emph{is there some general minimum principle form which all physical laws can be derived?}  The answer seems to be yes.  It is the Principle of Least Action.

To state the Principle of Least Action properly, we imagine that our physical system is described by a \term{configuration space} $\caC$ with coordinates $q_i$.  These might be ordinary Cartesian coordinates, or some other kind of "generalized" coordinate.  If the system consists of a  single particle moving on the $x$-axis, the $q = x$ is a a good choice. The configuration space the real line: $\caC = \bbR$.  If the system consists of a particle moving in a plane under the gravitational pull of an immovable mass, then polar coordinates of the  particle centered on the large mass is a good choice.   In this case the generalized coordinates are $q_1 = r$ and $q_2 = \theta$, and the configuration space is  a cylinder: $\caC = S^1\times \bbR$.   In a system of $N$ masses, then the $3N$ quantities given by the $x$, $y$ and $z$ components of position vectors $\bor_i$ is a possible choice.

Consider next a path $q: [a,b] \longrightarrow \caC$ in configuration space.  This path may be a possible motion of a single particle or a system of $N$ particles, or something else entirely.

Associated with generalized coordinates $q_i$ are \term{generalized velocities} $\dot q_i$, which we may view as coordinates in the tangent bundle $ T\caC$ of the configuration spaces.  Thus $(q,\dot q)$ represents  a point in the tangent bundle, i.e., a point $q$ of $\caC$ and an arrow $\dot q$ based at $q$. In this notation, we may speak of a function $L: T\caC \longrightarrow \bbR$.  Consider, for example, the configuration space of a single particle.  The function $K(x, \dot x) = (1/2)m\dot x^2$ is a function on the tangent bundle. This, the kinetic energy function, is independent of position.  The function $U(x, \dot x) = (1/2)kx^2$, a potential energy, is also a function on the tangent bundle.  By choosing the potential as we have, we have defined the classical mass-spring system, aka harmonic oscillator.

We can combine these functions in various ways.  The sum $H = K + U$ is the \term{Hamiltonian}, and the difference, $L = K - U$ is the \term{Lagrangian}.  Once again, these are functions on the tangent bundle of the configuration space.  however, given a path $q: [a, b] \longrightarrow \caC$, one has the composed functions $H\circ q$ and $L \circ q$.  These are defined for any path whatsoever, not just for actual motions of the system, that is, paths which satisfy Newton's laws. For a motion of the system, $H(q(t), \dot q(t)$ is constant.  This is the principle of conservation of energy:

\begin{align}
\frac{dH}{dt} &= \frac{d}{dt}\,\left( \frac{1}{2}m\dot x^2 + \frac{1}{2}\,kx^2\right) \\
&= m\ddot x + kx \\
& = ma - F \\
& = 0
\end{align}

On the other hand the Lagrangian varies with time, even for a physical motion as opposed to an arbitrary path.  In the case of the harmonic oscillator, which we suppose has total energy $E$, there are two noteworthy states.  The first is the state $x = 0$, in which $K = E$.  In this state the kinetic energy is at its maximum and the potential energy is at its minimum, that is, zero.  The second is the the state in which the potential energy is at its maximum, namely $E$, and the kinetic energy is zero.  The difference, $L = K - U$, varies with time between the extremes $K_{max}$ and $-U_{max}$; it represents a kind of instantaneous "cost of doing business" in the physical world.

The \term{action} of a path in configuration space is the integral

\begin{equation}
S(q(t)) = \int_a^bL(q(t), \dot q(t)) dt,
\end{equation}

If $\caP_{ab}(\caC)$ denotes the space of paths from $a$ to $b$ in
the configuration space $\caC$, then we ahve the functional

\begin{equation}
S: \caP_{ab}(\caC) \longrightarrow \bbR
\end{equation}

The Principle of Least Action states that motions of the system are distinguished as minima of this functional.  To find the minima, consider a variation $q(t) + s\delta q(t)$.  Here $\delta q(t)$ is a function with values in $\caC = \bbR^{3N}$ such that $\delta q(a) = \delta q(b) = 0$. Thus  $q_s(t)$ is a one-parameter family of paths with $q_0(t) = q(t)$.  That is, it is a \term{variation} or deformation of the path $q(t)$. We may think of $\delta q(t)$ as a tangent vector to the path $q(t)$.

Let us find the change in the action,

\begin{equation}
\Delta S = S(q(t) + s\delta q(t)) - S(q(t))
\end{equation}

To do, first compute the change in the Lagrangian:

\begin{align}
\Delta L &= L(q + s \delta q(t), q + s \delta \dot q(t)) - L(q, \dot q) \\
&= s\frac{\partial L}{\partial q_i}\delta q_i(t)
 + s\frac{\partial L}{\partial \dot q_i}\dot \delta q_i(t) + O(s^2)
\end{align}

where we sum over repeated indices.
Then

\begin{equation}
\Delta S = s\int_a^b\left(\frac{\partial L}{\partial q_i}\delta q_i(t)
 + \frac{\partial L}{\partial \dot q_i}\dot \delta q_i(t)\right) dt + O(s^2)
\end{equation}

Integrating the second term by parts and using $\delta q_i(a) = \delta q_i(b) = 0$,
we have

\begin{equation}
\label{actionvariation}
\Delta S =  s\int_a^b\left(\frac{\partial L}{\partial q_i}
 - \frac{d}{dt} \frac{\partial L}{\partial \dot q_i}\right)\delta q_i(t)\, dt + O(s^2)
\end{equation}

Since $\delta q(t)$ is an arbitrary function vanishing at the endpoints of the interval, the term in parenthesis must vanish if the change in the action is to vanish to first order.  Thus the condition for an extremum of the action is

\begin{equation}
\label{eulerlagrangeequation}
\frac{\partial L}{\partial q_i}
 - \frac{d}{dt} \frac{\partial L}{\partial \dot q_i} = 0
\end{equation}

These are the \term{Euler-Lagrange equations}.  Note that the quantity

\begin{equation}
\label{actionderivative|}
\frac{\delta S}{\delta q} =  \int_a^b\left(\frac{\partial L}{\partial q_i}
 - \frac{d}{dt} \frac{\partial L}{\partial \dot q_i}\right)\delta q\, dt
\end{equation}

is the directional derivative of $S$ at the path $q(t)$ in the direction of the tangent vector $\delta q(t)$.  To go one step further, consider the quantity $\delta S$ with components

\begin{equation}
\label{actiondifferential}
\delta S_i =  \frac{\partial L}{\partial q_i}
 - \frac{d}{dt} \frac{\partial L}{\partial \dot q_i}
\end{equation}

This is a kind of gradient of the action, where

\begin{equation}
\delta S = \delta S_i q_i
\end{equation}

BLAH BLAH Think abou this some more.





\subsection{Symmetry and form of the Lagrangian}

We now ask: \emph{what form must the Lagrangian take for a given physical system?}  Consider first the case of a free particle in a homogenous, isotropic space.  To say that space is homogenous is to say that there are no preferred points, that all points in space are "the same".  Consequently the Lagrangian cannot depend on the position $\bor$.  However, it can depend on the velocity $\boldsymbol{v} = \dot r $, since $\dot \bor$ is invariant under the translation $\bor \mapsto \bor + \boa$.  This is an application of invariance under translational symmetry.  Isotropy of space is the same as invariance under rotational symmetry, or more colloquially, "there are no preferred directions."  For this reason, the Lagrangian cannot depend on $\bov$ itself, but rather only on $||\bov||^2 = \bov\cdot\bov$.  Consequently it is of the form $a\bov\cdot\bov$ for some constant $a$. Application of the Euler-Lagrange equation with $q_i = x_i$ and $v_i  = \dot q_i$ yields

\begin{equation}
\frac{d}{dt} \frac{\partial L}{\partial v_i} = \frac{\partial L}{\partial x_i} = 0
\end{equation}

But the left-hand side is $a dv_i/dt$.  Thus, we have

\begin{equation}
\frac{dv_i}{dt} = 0.
\end{equation}

That is, a free particle moves with constant velocity. This fact, which also follows from Newton's laws of motion, comes as a consequence of (a) the Principle of Least Action, and (b) the translational and rotational symmetry of space.  In this context, constant velocity is the same as constant momentum.  More generally, one has

\begin{theorem} \strong{(Noether)}
Principle of Least Action + one-parameter group of symmetries = conservation law
\end{theorem}

\subheading{Central force}
As a sample application of Noether's theorem, imagine a one-particle system in the plane which is subject to a force $F$ directed towards the origin. Suppose further that the force is conservative, so that it is derived from a potential $U(r)$.  Thus $F = - \nabla U$.  Let $\bor = x\boi_x + y\boi_y$, where $\boi_x$ and $\boi_y$ are standard unit vectors in the $x$ and $y$ directions. Let $\boi_r$ and $\boi_\theta$ be unit vectors based at $\bor$ where the first points in the direction of $\bor$,  and the second is obtained from the first by rotation by 90 degrees.  The velocity $\dot \bor$ can be written as $\dot x \boi_x + \dot x \boi_y$ or in polar coordinates as $\dot r \boi_r + r\dot \theta \boi_\theta$.  Thus we can write the Lagrangian as

\begin{equation}
L
  = \frac{1}{2} mr^2 \dot \theta^2
  + \frac{1}{2}\, m\dot r^2
  - U(r)
\end{equation}

Consider the generalized momentum

\begin{equation}
p_\theta = \frac{\partial L}{\partial \dot \theta} = mr^2\dot\theta
\end{equation}

This is the \term{angular momentum}.
Since the Lagrangian does not depend on $\theta$, the Euler-Lagrange equations yield

\begin{equation}
\label{angularmomentum}
\frac{dp_\theta}{dt} = \frac{d}{dt}\frac{\partial L}{\partial \dot \theta} = 0
\end{equation}

Angular momentum is constant, i.e., is a conserved quantity.  The underlying reason is that the system is invariant under rotations about the origin, so that the potential function does not depend on $\theta$.  Notice that we can write \eqref{angularmomentum} as

\begin{equation}
\label{keplerlaw2}
\dot\theta = \frac{p_\theta}{m},
\end{equation}

where the right-hand side is constant.  The left hand side is the rate at which the position vector sweeps out area.   Consequently \eqref{keplerlaw2} is a form of Kepler's second law.

Now consider the case in which the potential is $U(r) = - GMm/r$, as is the case in the earth-sun system, where we consider the sun to be immovable.  Then

\begin{equation}
\frac{d}{dt}\frac{\partial L}{\partial \dot r}
= \frac{\partial L}{\partial r} = mr\dot \theta^2 - \frac{GMm}{r^2}
\end{equation}

That is,

\begin{equation}
m\ddot r = \frac{d}{dt}\frac{\partial L}{\partial \dot r}
= \frac{\partial L}{\partial r} = mr\dot \theta^2 - \frac{GMm}{r^2}
\end{equation}

Suppose that the trajectory of the particle is a circle, so that $r$ is constant.  Then

\begin{equation}
r\dot \theta^2 = \frac{GM}{r^2}
\end{equation}

so that

\begin{equation}
\dot \theta^2 = \frac{GM}{r^3}
\end{equation}

Now $\dot\theta$ is the angular velocity, so the period of the mass in its circular orbit is $T = 2\pi/\dot\theta$.  Consequently period is related to the radius of the orbit by

\begin{equation}
T^2 = \frac{4\pi^2}{GM}\, r^3
\end{equation}

This is a special case of Kepler's third law, which state that the square of the period is proportional to the cube of the semi-major axis of an elliptical orbit.





\subsection{Chain of masses}

Consider a system of $N+2$ objects of mass $m$ connected by springs with spring constant $k$, as in the figure below.  With the exception of the leftmost and rightmost masses, which are fixed, the masses are free to move in the direction of the $x$-axis.

\image{http://psurl.s3.amazonaws.com/images/jc/optballs-a18c.png}{Chain of masses}{align: center}

Denote by Let $q_j$ the position of the $j$-th mass.  Then the system in the figure is governed by the Lagrangian

\begin{equation}
L = \sum_{j=0}^{N+1}\frac{m}{2} \dot q_j^2
+ \sum_{j=0}^{N+1}\frac{k}{2}  (q_j - q_{j+1})^2
\end{equation}

The Euler-Lagrange equations yield a system of coupled ODE's:

\begin{equation}
m\ddot q_j = k(q_{j+1} - 2q_j - q_{j-1} )
\end{equation}

It is convenient to write this system as

\begin{equation}
\label{discretestringodes}
\ddot q_j = \omega^2(q_{j+1} - 2q_j - q_{j-1} ),
\end{equation}

where $\omega^2 = k/m$.
One way to solve such a system is to write it as

\begin{equation}
\label{discretestringeqmo}
\frac{d^2 q}{dt^2} = Wq,
\end{equation}

where $W$ is a real symmetric matrix which happens to be negative definite, e.g,,

\begin{equation}
W =
\begin{pmatrix}
-2 & \phantom{-}1 & \phantom{-}0 & \phantom{-}0 & \phantom{-}0 \\
\phantom{-}1 & -2 & \phantom{-}1 & \phantom{-}0 & \phantom{-}0 \\
\phantom{-}0 & \phantom{-}1 & -2 & \phantom{-}1 & \phantom{-}0 \\
\phantom{-}0 & \phantom{-}0 & \phantom{-}1 & -2 & \phantom{-}1 \\
\phantom{-}0 & \phantom{-}0 & \phantom{-}0 & \phantom{-}1 & -2
\end{pmatrix}
\end{equation}

Since $W$ is symmetric, there is an orthogonal matrix $C$ such that $C^{-1}WC$ is diagonal with diagonal entries $-\omega_j^2$.  Let $q = Cq'$.  Then

\begin{equation}
\frac{d^2 q'}{ dt^2} = C^{-1}WCq',
\end{equation}

The resulting system, which has the form

\begin{equation}
\frac{d^2 q'}{ dt^2} =W'q',
\end{equation}

where $W'$ is diagonal with diagonal entries $-\omega_j^2$.
This is a system of uncoupled $ODE's$ which can be immediately solved by functions $q'_j(t) = e^{\pm i \omega_jt}$.  Note that if $w_j$ is the $j$-th eigenvalue of $W$, then the $j$-th frequency is $\omega_j =(-w_j)^{1/2}$.
The functions $q_j'(t)$ give the \term{normal modes} of the system in the $q'$ coordinate system.  Solutions of \eqref{discretestringeqmo} are obtained by expressing the normal modes in the $q$-coordinate system.

Let us solve for the normal modes ind a different way, using the results of the foregoing discussion as a hint for the general form of a normal mode of \eqref{discretestringodes}:

\begin{equation}
q_j(t) = A_j e^{i\Omega t)}
\end{equation}

The amplitudes $A_j$ and the frequency $\Omega$ are to be determined.  Substituting into \eqref{discretestringodes}, we find that

\begin{equation}
\label{discretestringdifferenceeq}
\Omega^2A_j + \omega^2(A_{j+1} - 2A_j + A_{j-1}) = 0
\end{equation}

Define the forward and backward difference operators by  $(\Delta_+ A)_j = A_{j+1} - A_{j}$ and $(\Delta_ A)_j = A_j - A_{j-1}$. Define the second difference operator $\Delta^2 = \Delta_+\Delta_-$.  Then the preceding equation reads

\begin{equation}
\Omega^2A_j + \omega^2( \Delta^2 A)_j = 0
\end{equation}

Using the analogy between finite difference equations and differential equations, we are led to consider solutions of the form $A_j  = \sin j\phi$ for some phase factor $\phi$.  A cosine term is excluded by the  condition that the leftmost mass is fixed.  Because the rightmost mass is fixed, $(N+1)\phi = n\pi$ for some integer $n$ which indexes the mode.  At this point we can conclude that the $n$-th mode is given by functions

\begin{equation}
\label{solutionformAOmega}
q_j^{(n)} = A_j^{(n)} e^{\pm i \Omega_n t}
\end{equation}

where the amplitudes are

\begin{equation}
A_j^{(n)} = \sin \frac{\pi n j}{2(N+1)}
\end{equation}

To determine the frequencies $\Omega_n$, substitute  \eqref{solutionformAOmega} into \eqref{discretestringdifferenceeq} and apply the addition law for the sine and the half-angle formula to obtain

\begin{equation}
\Omega_n = 2\omega\sin \frac{\pi n}{2(N+1)}
\end{equation}

where $n$ ranges over the integers in $[1,N]$.  Thus $\Omega$  ranges over an interval $[\epsilon_N, 2\omega]$, where $\epsilon_N$ is small and $\epsilon_N \to 0$ as $N \to \infty$.

\subsection{Vibrating string and Lagrangian density}


\subsection{Relativistic Lagrangian}

A relativistic Lagrangian must be  a Lorentz scalar, that is, a function invariant under Lorentz transformations.  The basic Lorentz scalar is $-c^2 t^2 +x^2 + y^2 + z^2$.  Define the infinitesimal element of arc length in Minkowsky space  by $ds^2 = c^2 dt^2 - dx^2 - dy^2 - dz^2$.  It is also a Lorentz scalar. If $t \mapsto (t,x(t), y(t), z(t))$ is a path in Minkowski space, then $ds^2 = (c^2 - v^2)dt^2$ where
$v^2 = (dx/dt)^2 + (dy/dt)^2 + (dz/dt)^2$. Thus $ds^2$ is a positive multiple of $dt^2$ for particles traveling at less than the velocity of light.  In that case, we can take the square root and define the elapsed proper time interval along a path $\alpha(t)$ in Minkowski space as

\begin{equation}
\label{relativisticaction}
  \tau = \int_\alpha ds
\end{equation}

Proper time is time as measured by a clock which travels with the observer.

The proper time itself is unsuitable as an action because it has the wrong dimension, namely $[x]$, where the brackets denotes "units of". Actions have dimension $[mv^2][t] = [mx^2 / t^2][t] = [mx^2/t]$.  Consider therefore the quantity

\begin{equation}
S = kmc \int_\alpha ds
\end{equation}

where $m$ is a mass, and $k$ is a constant to be determined.  Write the infinitesimal length element as

\begin{equation}
ds = \sqrt{c^2 dt^2 - dx^2 - dy^2 - dz^2} = c\sqrt{1 - \frac{ v^2}{ c^2}}\,dit
\end{equation}

so that the action is

\begin{equation}
S = kmc^2 \int_{t_1}^{t_2}\sqrt{1 - \frac{ v^2}{ c^2}}\,dt
\end{equation}

The Lagrangian is the integrand, namely,

\begin{equation}
L = kmc^2 \sqrt{1 - \frac{ v^2}{ c^2}}
\end{equation}

Developing the right hand side in a Taylor series, we find that

\begin{align}
L &= kmc^2 \left(1 - \frac{ v^2}{ c^2 }\right)^{1/2 } \\
 &= kmc^2 \left(1 - \frac{1}{2}\frac{ v^2}{ c^2} + \cdots\right) \\
 &= kmc^2 - k\frac{1}{2}\,mv^2 + \cdots
\end{align}

Thus, if we choose $m$ to the the rest mass of the particle and $k = -1$, then

\begin{align}
L &= -mc^2 \left(1 - \frac{ v^2}{ c^2 }\right)^{1/2 } \\
& =  -mc^2  + \frac{1}{2}\,mv^2 + \cdots
\end{align}

The relativistic Lagrangian of a free particle at velocities much less than the speed of light then has the form constant plus classical Lagrangian plus a small correction term.  To conclude this discussion, the action is

\begin{equation}
S = -mc\int_\alpha ds
\end{equation}






Thus for velocities which are small in relation to the speed of light, the Lagrangian for a free particle is essentially the classical free particle Lagrangian plus a constant.  This explains the choice of constant in \eqref{relativisticaction}.

Let us find the momentum:

\begin{equation}
p = \frac{\partial L}{\partial v} = \frac{mv}{\sqrt{1 - \frac{ v^2}{ c^2}}}
\end{equation}

For a more compact notation, set

\begin{equation}
\gamma = \frac{1}{\sqrt{1 - v^2 /c^2}}
\end{equation}

and rewrite the momentum as

\begin{equation}
 p = \gamma mv
\end{equation}

and the Lagrangian as

\begin{equation}
L = - \frac{mc^2}{\gamma}
\end{equation}


What about the energy?  We can find this by finding the Hamiltonian.

\begin{equation}
H = p\cdot v - L = \gamma mv^2 +   \frac{mc^2 }{\gamma}
\end{equation}

Simplifying and putting $E$ in place of $H$, we find

\begin{equation}
E = \gamma m c^2
\end{equation}

The energy is the rest mass energy scaled by the reciprocal of $\gamma$ . For velocities that are small compared to the speed of light, we have the first order Taylor approximation

\begin{equation}
E = mc^2 + \frac{1}{2}mv^2 + \cdots
\end{equation}

In this approximation, the total energy is the rest mass energy plus the classical kinetic energy.


\subsection{References}

\begin{thebibliography}

\bibitem{Torre} \href{http://www.physics.usu.edu/torre/6010_Fall_2010/Lectures/04.pdf}{USU notes (Torre)}

\bibitem{Eduardo} \href{http://eduardo.physics.illinois.edu/phys582/582-chapter3.pdf}{Eduardo's notes}

\bibitem{Leiden} \href{http://home.strw.leidenuniv.nl/~icke/ps/SymmetryLagrangian.pdf}{Symmetries and the Form of the Lagrangian}

\bibitem{Baez} \href{http://math.ucr.edu/home/baez/classical/texfiles/2005/book/classical.pdf}{John Baez notes on Classical Mechanics}

\bibitem{Zwiebach} \href{http://fma.if.usp.br/~amsilva/Livros/Zwiebach/chapter5.pdf}{The Relativistic Lagrangian}

\bibitem{Eagle} \href{http://eagle.phys.utk.edu/guidry/astro490/lectures/lecture490_ch4.pdf}{Lorentz covariance}


\end{thebibliography}
