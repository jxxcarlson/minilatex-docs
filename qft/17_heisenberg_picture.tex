\italic{\xlink{34}{Notes on Quantum Field Theory}}


\setcounter{section}{17}

\italic{As a prelude to studying the quantum string, let us revisit the harmonic oscillator, which we will examine in light of the so-called "Heisenberg picture". It is one of three views of  quantum mechanics -- the Schrödinger picture and the Interaction picture being the other two.}

\section{Harmonic Oscillator: Heisenberg picture}

\innertableofcontents


The aim of quantum mechanics is to compute matrix elements $\langle \phi(t) | A | \psi(t) \rangle$, where $A$ is an observable and the evolution of quantum states in time is given by the action of the propagator: $\psi(t) = U(t)\psi(0)$.  There is, however, another way to compute this quantity.

\begin{align}
\langle \phi(t) | A | \psi(t) \rangle &= \langle U(t) \phi(0) | A | U(t) \psi(0) \rangle \\
&= \langle \phi(0) | U(t) ^{-1}A U(t) |  \psi(0) \rangle \\
&= \langle \phi(0) | A(t)|  \psi(0) \rangle
\end{align}

where by definition $A(t) = U(t) ^{-1}A U(t)$. In this formulation,
the states are constant in time while the observable evolves.
This approach is the "Heisenberg picture". It may seem counter to our intuition, but it yields the same results and in many cases does so more transparently than in the Schrödinger picture.

To use the Heisenberg picture we need the equations of motion for observables.  This is an easy computation using the equation of motion of the propagator, $\dot U = -(i/\hbar) HU$.  Write $U_t$ for $U(t)$, differentiate the definition $A_t = U_t^{-1}A U_t$, and proceed as follows :

\begin{align}
\dot A_t &= - (U_t^{-1}  \dot U_t U_t^{-1} ) A U_t + U_t^{-1}A\dot U_t \\
&=  (i/\hbar)(U_t^{-1}  H) A U_t - (i/\hbar) U_t^{-1}A HU_t \\
&=  (i/\hbar)U_t^{-1} (HA - AH)U_t  \\
&=  (i/\hbar)[H_t ,A_t]
\end{align}

The equation of motion is therefore

\begin{equation}
\dot A(t) =  \frac{i}{\hbar}[H(t),A(t)]
\end{equation}


\subsection{Commutators}

When working in  the Heisenberg picture one has access to all commutator relations that exist in the Schrödinger picture. Suppose that $A$ and $B$ are observables satisfying $[A,B] = C$, and let $A(t)$, $B(t)$ and $C(t)$ their time-evolved versions.
Then

\begin{equation}
[A(t), B(t)] = C(t)
\end{equation}

This follows from a straightforward computation:

\begin{align}
U_t^{-1} [A, B] U_t&= U_t^{-1} (AB  - BA) U_t \\
&= (U_t^{-1} AU_t)(U_t^{-1} BU_t)
 - U_t^{-1}BU_t U_t^{-1} AU_t \\
&= A_tB_t - B_t  U_t \\
&= [A_t, B_t]
\end{align}

But $U_t^{-1} [A, B] U_t = U_t^{-1} C U_t = C_t$, so the result is proved.

We will often have occasion to evaluate commutators of the form $[A^n,B]$.
To evaluate the commutator for  $n=2$, write

\begin{align}
[A^2, B] &= A^2B - BA^2 \\
 &= A (AB - BA) + (AB - BA) A \\
 &= \{A, [A, B] \},
\end{align}

where $\{A,B\} = AB + BA$ is the anti-commutator.  If in addition, $A$ commutes with $[A,B]$, then

\begin{equation}
[A^2, B] = 2A[A,B]
\end{equation}


\subsection{Position and momentum operators}

Let us study the harmonic oscillator in the optics of the Heisenberg picture.  The basic operators are $\hat x$ and $\hat p$, where $[\hat x, \hat p] = i\hbar$, and where the Hamiltonian is

\begin{equation}
\hat H = \frac{\hat p^2}{2m} + \frac{m \omega^2}{2} \hat x^2.
\end{equation}

According to the above $\hat p(t) = U(t)^{-1}\hat p U(t)$ and $\hat x(t) = U(t)^{-1}\hat x U(t)$.  Moreover, these operators satisfy the commutation relation $[\hat x(t), \hat p(t)] = i\hbar$.


To find explicit formulas for $\hat p(t)$ and $\hat x(t)$, we solve the equations of motion.  First,

\begin{equation}
\frac{d\hat x(t)}{dt} = \frac{i}{\hbar}[\hat H(t), \hat x`(t)]
= \frac{i}{2m\hbar}[\hat p^2(t), \hat x(t)]
\end{equation}

But

\begin{equation}
[\hat p^2(t), \hat x(t)] = 2[\hat p(t), [\hat p(t), \hat x(t)]] = -2i\hbar \hat p(t),
\end{equation}

so that

\begin{equation}
\frac{d\hat x(t)}{dt}
= -\frac{\hat p(t)}{m}
\end{equation}

A similar calculation yields

\begin{equation}
\frac{d\hat p(t)}{dt} = m\omega^2 \hat x(t)
\end{equation}

Combining these, we have

\begin{align}
\frac{d^2\hat p(t)}{dt^2} + \omega^2\hat p(t) & = 0 \\
\frac{d^2\hat x(t)}{dt^2} + \omega^2\hat x(t) & = 0
\end{align}

Both observables are governed by the harmonic oscillator equation.
The momemtum observable has the general form $\hat p(t) = A\cos \omega t + B\sin\omega t$, where $\hat p(0) = \hat p$ and $d\hat p/dt(0) = -m\omega^2 \hat x$.
Therefore

\begin{equation}
\hat p(t) = (\cos\omega t) \hat p -  (m\omega \sin \omega t) \hat x
\end{equation}

A similar computation yields

\begin{equation}
\hat x(t) = (\cos\omega t) \hat x +  \left(\frac{p}{m\omega} \sin \omega t\right) \hat x
\end{equation}

As a check, one finds by direct computation that  $[\hat x(t), \hat p(t)] = i\hbar$. One also finds by direct computation (exercise), that $\hat H(t) = \hat H$.  That is, the Hamiltonian operator is independent of time.




\subsection{Creation and annihilation operators}

Recall that the creation and annihilation operators for the harmonic oscillator were defined by

\begin{equation}
\label{creationop2}
a^\dagger = \frac{m\omega x - i\hat p}{\sqrt{2m\omega\hbar}}
\end{equation}

and

\begin{equation}
\label{annihilationop2}
a = \frac{m\omega x + i\hat p}{\sqrt{2m\omega\hbar}}
\end{equation}

The  Schrödinger-picture Hamiltonian can be expressed in terms of these operators:

\begin{equation}
\hat H = \hbar \omega(a^\dagger a + \small{\frac{1}{2}})
\end{equation}

It is therefore natural to ask how $\hat a$ and $\hat a^dagger$ evolve in time. The equation of motion yields

\begin{align}
\frac{d}{dt}\hat a^\dagger(t) &= \frac{i}{\hbar}[H(t), a^\dagger(t)] \\
&= i\omega [\hat a^\dagger(t) \hat a(t), a^\dagger(t)]
\end{align}

To evaluate the commutator, use the identity $[\hat a(t), \hat a^\dagger(t)] = 1$:

\begin{align}
[\hat a^\dagger(t) \hat a(t), a^\dagger(t)]
 &= \hat a^\dagger(t) \hat a(t) \hat a^\dagger(t) -
\hat a^\dagger(t) \hat a^\dagger(t) \hat a(t) \\
&= \hat a^\dagger(t) \hat a^\dagger(t)(t) \hat a
 + a -
\hat a^\dagger(t) \hat a^\dagger(t) \hat a(t) \\
&= \hat a^\dagger(t)
\end{align}

Thus we have

\begin{equation}
\frac{d}{dt}\hat a^\dagger(t) = i\omega \hat a^\dagger(t)
\end{equation}

Either by direct argument or by taking the complex conjugate of the preceding equation, one has

\begin{equation}
\frac{d}{dt}\hat a(t) = -i\omega \hat a(t)
\end{equation}

These differential equation give explicit solutions of the equations of motion:

\begin{align}
\label{creationopst}
\hat a^\dagger(t) &= e^{i\omega t} \hat a^\dagger  \\
 \hat a(t) &= e^{-i\omega t} \hat a
\end{align}

From general principles, we know that

\begin{equation}
\hat H(t) = \hbar \omega\left(\hat a^\dagger(t)  \hat a(t)
+ \small{\frac{1}{2}} \right)
\end{equation}

It is clear from this formulation and the formulas  \eqref{creationopst} that the Hamiltonian operator $\hat H(t)$ is independent of time.

\begin{remark}
By adding and subtracting \eqref{creationop2} and \eqref{annihilationop2}, we obtain
\end{remark}


\begin{equation}
\hat x(t) = \sqrt{\frac{\hbar}{2m\omega}}\left( \hat a^\dagger(t) + \hat a(t) \right)
\end{equation}


and


\begin{equation}
\hat p(t) = i\sqrt{\frac{m\omega\hbar}{2}}\left( \hat a^\dagger(t)  - \hat a(t) \right)
\end{equation}
