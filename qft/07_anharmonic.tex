beginMetadata:
{
    "id": "255aea55-5754-4c45-bf59-cd2df1029a04",
    "documentNumber": 1,
    "author": "jxxcarlson",
    "title": "Anharmonic Oscillator",
    "path": "qft/07_anharmonic.tex",
    "tags": [
        "quantum",
        "physics",
        "oscillator",
        "harmonic"
    ],
    "keyString": "anharmonic oscillator a=jxxcarlson qft/07_anharmonic.tex t=quantum t=physics t=oscillator t=harmonic",
    "timeCreated": 1595082122908,
    "timeModified": 1598555055974,
    "public": true,
    "collaborators": [],
    "docType": "miniLaTeX",
    "versionNumber": 1,
    "versionDate": 1598555085973
}
endMetadata
\begin{mathmacro}
\newcommand{\bra}[0]{\langle}
\newcommand{\ket}[0]{\rangle}
\newcommand{\WW}[0]{\mathbb{WW}}
\end{mathmacro}



\setcounter{section}{7}

\section{Anharmonic Oscillator}

\innertableofcontents

The classical anharmonic oscillator is one with a "non-linear spring". By this we mean that the standard force law $F = -kx$ has additional terms, e.g., $F = -kx - \ell x^3$.  In this case the corresponding potential is a quartic: 

\begin{equation}
V(x) = (1/2)kx^2 + (\ell/4)x^4
\end{equation}

In this section we study the quantum anharmonic oscillator with a quartic term in the potential.  Our first task is to find the energy of the ground state when a small quartic term is added to the potential.  For this we need some basic perturbation theory.  An approximation to the ground state energy in two ways: (i) by doing a Gaussian integral (ii) by working out the expression $(a + a^\dagger)^4$ as a noncommutative polynomial using the harmonic oscillator operator calculus.

\subsection{Perturbation theory}

Let us suppose given a quantum mechanical system with Hamiltonian of the form $H = H_0 + \epsilon V$, where a complete set of non-degenerate eigenstates for $H_0$ is known.  Let us call these states $\psi^{(m)}$. By nondegenerate, we mean that all eigenspaces have dimension one. The second term is a small multiple of an operator which we generally take to be a potential.  We use this setup to                                         find solutions of the time-independent Schroedinger equation $H\psi = E\psi$ up to terms of order epsilon, where we imagine that true solutions have an expansion $\psi = \psi_0 + \epsilon \psi_1 + \epsilon^2 \psi_2 + \cdots$.  Substituting into the Schroedinger equation and ignoring terms in $\epsilon^2$ or higher powers, we have

\begin{equation}
H_0\psi_0 + \epsilon H_0 \psi_1 + \epsilon V\psi_0
 =
E_0\psi_0 + \epsilon E_0 \psi_1 + \epsilon E_1 \psi_0
\end{equation}

The $\epsilon^0$ component of this equation is


\begin{equation}
H_0\psi_0 = 
E_0\psi_0 
\end{equation}

Therefore $\psi_0$, the zeroth term in the perturbation expansion, is an eigenfunction of the unperturbed Hamiltonian.  That is, $\psi_0 = \psi^{(m)}$ up to  a constant.  

The $\epsilon^1$ component of the equation reads

\begin{equation}
\label{foperturbationequation}
 H_0 \psi_1 + V\psi_0
 =
 E_0 \psi_1 +  E_1 \psi_0
\end{equation}


Take the inner product with $\psi_0$:

\begin{equation}
 \bra \psi_0 | H_0  | \psi_1\ket +  \bra \psi_0  | V | \psi_0 \ket
 =
 E_0 \bra \psi_0  |  \psi_1 \ket +  E_1 \bra \psi_0  |  \psi_0 \ket
\end{equation}

Because $\psi_0$ is an eigenfunction of $H_0$, the first term on the left is equal to the first term on the right, so that

\begin{equation}
\label{perturbation-of-energy}
E_1 =\frac{ \langle\psi_0 | V | \psi_0\rangle}{ ||\psi_0||^2 }
\end{equation}

In other words, if $E^{(m)}$ is the energy of the unperturbed Hamiltonian, then the energy of the corresponding state of the perturbed system is $E^{(m)} + \epsilon \Delta E^{(m)}$, where

\begin{equation}
\label{perturbation-of-En}
\Delta E^{(m)} =\frac{\langle \psi^{(m)} | V | \psi^{(m)} \rangle }{ || \psi^{(m)} ||^2 }
\end{equation}

To find the wave functions for the perturbed system, assume that $\psi_0 = \psi^{(n)}$ and take the inner product of \eqref{foperturbationequation} with $\psi^{(m)}$ for $m \ne n$.  One obtains

\begin{equation}
\bra \psi^{(m)} | H_0 | \psi_1 \ket + \bra \psi^{(m)} | V | \psi_0 \ket
  =
E_0\bra \psi^{(m)} | \psi_1 \ket + E_1\bra \psi^{(m)} | \psi_0 \ket
\end{equation}

The first term on the left is $E_m\bra \psi^{(m)} | \psi_1 \ket$ and the last term on the right vanishes, so that we obtain

\begin{equation}
\label{perturbation-fourier-coefficients}
  \bra \psi^{(m)} | \psi_1 \ket
  = \frac{\langle \psi^{(m)} | V | \psi_1 \rangle}{E_0 - E_m}
\end{equation}

The numbers are the Fourier coefficients of the expansion of $\psi$:

\begin{equation}
\label{perturbation-fourier-expansion}
  \psi = \psi^{(n)} 
+ \epsilon \sum_{m \ne n} \frac{\langle \psi^{(n)} | V | \psi^{(m)} \rangle}{  E^{(n)} - E^{(m)}  }\psi^{(m)} 
\end{equation}

\subsection{Quartic perturbation by Gaussian integrals}


Let us consider a quartic perturbation of the harmonic oscillator, where 

\begin{equation}
H_0 = \frac{1}{2m}(\hat p^2 + m^2 \omega^2 x^2) +  \lambda g x^4,
\end{equation}

where $\lambda$ is a small dimensionless parameter.  Now $p^2/2m$ is a kinetic energy term, so $[gx^4] = [E]$, where $E$ is an energy and $[\  ]$ means "units of". Solving, we have $[g] = [Ex^{-4}]$.  Let us try to cook up a "natural value" for $g$.  A natural energy is the zero point energy of the oscillator, $\omega\hbar/2$.  In addition, we need a natural length scale for $x$.  To this end, consider the ground state wave function, which is proportional to $e^{-m\omega x^2/2\hbar}$.  Solve for $m\omega x^2/2\hbar = 1$. One obtains $x_0^2 = 2\hbar/m\omega$. At disance $x = x_0$, the value of the wave function has decreased from its maximum at $x = 0$ by a factor of $1/e$. Thus a natural choice is 

\begin{equation}
g = \frac{\omega\hbar}{2}\left(\frac{2\hbar}{m\omega}\right)^{-2} = \frac{m^2 \omega^3}{8\hbar}
\end{equation}


Writing $H= H_0 + \lambda g x^4$, we find ourselves in the context of perturbation theory. Then the shift in the $n$-th energy level is given by 

\begin{equation}
\label{DeltaEn]}
\Delta E_n = \lambda g \bra \psi_n | x^4 | \psi_n \ket ,
\end{equation}
and where the $\psi_n$ are normalized wave functions for the harmonic oscillator.  Referring to the definition of the normalized ground state wave function, we find that the shift in energy levels is given by a Gaussian integral:

\begin{equation}
\Delta E_0 = \lambda g \left(\frac{m\omega}{\pi\hbar}\right)^{ 1/2 } \int_{-\infty}^\infty x^4 e^{-m\omega x^2 / \hbar} dx
\end{equation}

The integral to be evaluated has the form

\begin{equation}
I_2 = \int_{-\infty}^\infty x^2 e^{-\alpha x^2} dx
\end{equation}

To compute it, recall that 

\begin{equation}
\int_{-\infty}^\infty e^{-\alpha x^2} dx 
= \left( \frac{\pi}{\alpha} \right)^{1/2}
\end{equation}

Now compare this derivative computation

\begin{equation}
\frac{d}{d\alpha} \int_{-\infty}^\infty e^{-\alpha x^2} dx = - \int_{-\infty}^\infty x^2 e^{-\alpha x^2} dx
\end{equation}

with this one:

\begin{equation}
\frac{d}{d\alpha} \left(\frac{\pi}{\alpha}\right)^{1/2} = -\frac{1}{2}\left(\frac{\pi}{ \alpha^3 } \right)^{1/2}
\end{equation}

to conclude that

\begin{equation}
 \int_{-\infty}^\infty x^2 e^{-\alpha x^2} dx =  \frac{1}{2}\left(\frac{\pi}{ \alpha^3 } \right)^{1/2}
\end{equation}

Similarly, one obtains

\begin{equation}
 \int_{-\infty}^\infty x^4 e^{-\alpha x^2} dx =  \frac{3}{4}\left(\frac{\pi}{ \alpha^5 } \right)^{1/2}
\end{equation}

Working through the constants, one obtains at long last the
first order energy shift:

\begin{equation}
\label{anharmonic_correction}
\Delta E_0 = \frac{3}{16}\left( \frac{\omega \hbar}{2}\right)  \lambda
\end{equation}


\subsection{Quartic perturbation by operator calculus}

The operator "multiplication by $x$" can also be written in terms of creation and annihilation operators.  Adding the expressions for these operators in the xref::225[last section], we find that 

\begin{equation}
x = \left[\frac{ \hbar }{  2m \omega} \right]^{1/2} (a + a^\dagger )
\end{equation}

Substituting this into \eqref{DeltaE}, we find that

\begin{align}
\label{aa4}
\Delta E_0 &= \lambda g \left(\frac{ \hbar }{  2m \omega} \right)^{2} \bra \psi_0 | ( a + a^\dagger )^4 | \psi_0 \ket \\
&= \lambda \frac{1}{16}\frac{\omega \hbar}{2}\bra \psi_0 | ( a + a^\dagger )^4 | \psi_0 \ket
\end{align}

The operator $(a + a^\dagger)^4$ is a sum of sixteen noncommutative monomials which can be listed in lexicographical order as 

\begin{align}
& a^\dagger a^\dagger a^\dagger a^\dagger \\
& a^\dagger a^\dagger a^\dagger a \\
& a^\dagger a^\dagger a  a^\dagger \\
& a^\dagger  a  a^\dagger a^\dagger \\
& etc
\end{align}

We say that a monomial is in \term{normal order} if all of the creation operators appear before the annihilation operators.  In the list above, the first, second, and fourth entries are in normal order.  Using the identity $[a,a^\dagger] = 1$, any monomial can be expressed as a sum of monomials in normal order.  One has, for instance, $aa^\dagger = a^\dagger a + 1$.  Now consider the monomials $M$ that might enter into the formula \eqref{aa4} with nonzero coefficient.  A bit of reflection tells us that such a monomial must consist of two creation operators and two annihilation operators.  This is because the expression $M\psi_0$ must be be a multiple of $\psi_0$.  That is, $M$ must raise the eigenvalue twice and also lower it twice. There are six such monomials. The monomial $M$ must also have a creation operator on the right, since otherwise $M\psi_0 = 0$.  For essentially the same reasons, it must have an annihilation operator on the left.  This constraint reduces the number of admissible monomials to two: $aa^\dagger aa^\dagger$ and 
$aaa^\dagger a^\dagger$.  Let us give the beginning of the computation for the first of these so that you see the pattern.  Then we will state the result. The computation is a kind of game in which we move $a$'s to the right using the relation $ a a^\dagger = a^\dagger a + 1$.  To win the game is to express the monomial as a sum of monomials in normal order.  The first move is $aaa^\dagger a^\dagger = aa^\dagger a a^\dagger + aa^\dagger$.  After a sequence of such moves, you will find that

\begin{equation}
aaa^\dagger a^\dagger = a^\dagger a^\dagger  aa + 4a^\dagger a + 2
\end{equation}

For the second monomial, you will find that

\begin{equation}
aa^\dagger a a^\dagger =  a^\dagger a^\dagger  aa + 2a^\dagger a + 1
\end{equation}

We are led to the conclusion that

\begin{equation}
(a + a^\dagger )^4 = M' + 3,
\end{equation}

where $M'$ is a sum of monomials different from 1.
Now think about expressions $\bra \psi_0 | M | \psi_0 \ket$, where
$M$ is a  monomial in normal order.  It is zero if $M \ne 1$ and
is 1 if  $M = 1$.  We can therefore read off the value of the perturbation term for the energy:

\begin{equation}
\Delta E_0 = \frac{3}{16}\left( \frac{\omega \hbar}{2}\right)  \lambda
\end{equation}

This is in agreement with the value computed by doing a Gaussian integral.

\begin{remark} 
Let $P(a,a^\dagger)$ be a polynomial in the creation and annihilation operators.  Let $m_M(P)$ be the multiplicity with which $M$ appears in the normal order expression for $P$. Then $\bra \psi_0 | P(a,a^\dagger) | \psi_0 \ket = m_1(P)$
\end{remark}

Note the combinatorial flavor of the computation of the energy shift by this method.  We will encounter it again in the theory of Feynman diagrams.

\subsection{References}

\href{http://www2.ph.ed.ac.uk/~ldeldebb/docs/QM/lect17.pdf}{Perturbation Theory, Edinburgh}

\href{http://www.cavendishscience.org/phys/tyoung/tyoung.htm}{Two-slit experiment}