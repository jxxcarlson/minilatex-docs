beginMetadata:
{
    "id": "531df50c-44c1-4413-a777-4b5542882aca",
    "documentNumber": 41,
    "author": "jxxcarlson",
    "title": "Fourier Transform",
    "path": "qft/09_fourier.tex",
    "tags": [
        "quantum",
        "physics",
        "fourier"
    ],
    "keyString": "fourier transform a=jxxcarlson qft/09_fourier.tex t=quantum t=physics t=fourier",
    "timeCreated": 1595380196327,
    "timeModified": 1598555245013,
    "public": true,
    "collaborators": [],
    "docType": "miniLaTeX",
    "versionNumber": 1,
    "versionDate": 1598555266008
}
endMetadata
\begin{mathmacro}
\newcommand{\bra}[0]{\langle}
\newcommand{\ket}[0]{\rangle}
\newcommand{\caF}[0]{\mathcal{F}}
\newcommand{\boR}[0]{\bf{R}}
\end{mathmacro}

\setcounter{section}{9}

\section{Fourier transform}


\innertableofcontents

The \term{Fourier transform} is defined by

\begin{equation}
(\caF g )(k) = \frac{1}{\sqrt{2\pi}} \int_{-\infty}^\infty g(x) e^{-ikx} dx
\end{equation}

The Fourier transform is a unitary operator on $L^2(\boR )$ with inverse

\begin{equation}
(\caF^{-1} g)(x) =  \frac{1}{\sqrt{2\pi}} \int_{\infty}^\infty g(k) e^{-ikx} dk
\end{equation}

One often writes $\hat g$ for $\caF(g)$. 

The fact that $\caF$ is invertible allows us to write

\begin{equation}
g(x) = (\caF^{-1} \caF g)(x) =  \frac{1}{\sqrt{2\pi}}  \int_{-\infty}^\infty \hat g(k)  e^{ikx} dk
\end{equation}

The fact that the Fourier transform is unitary is \term{Plancherel's theorem}.  Thus $g(x)$ 
isa superposition of  functions $e^{ikx}$ which appear with weight $\hat g(k)$.  In Dirac's notation, we proceed formally to write this equation as

\begin{equation}
g(x)  = \frac{1}{\sqrt{2\pi}} \int_{-\infty}^\infty \hat g(k) | k \ket dk
\end{equation}

where $| k \ket = e^{ikx}$. Now $\hat g(k) = (1/\sqrt{2\pi} )\bra k | g \ket$, so the integral can be written as

\begin{equation}
g(x)  = \frac{1}{2\pi} \int_{-\infty}^\infty  | k \ket \bra k | g \ket dk
\end{equation}

Still proceeding formally, we write his as

\begin{equation}
{\bf 1} = \frac{1}{2\pi} \int_{-\infty}^\infty dk | k \ket \bra k |  
\end{equation}

This is the resolution of the identity in the case of continuous spectrum.




\subsection{Examples}

Let $f(x) = e^{-\lambda x}$ for $x > 0$, $f(x) = 0$ for $x < 0$.  This is a sudden but exponentially decaying pulse. Then

\begin{align}
\hat f(k) &= \frac{1}{\sqrt{2\pi}}\int_0^\infty e^{-\lambda x} e^{ikx} dx \\
&= \frac{1}{\sqrt{2\pi}} \frac{-1}{ik + \lambda}\Big\vert_0^\infty \\
 &= \frac{1}{\sqrt{2\pi}(ik + \lambda)}
\end{align}

In the same manner, we find that if $f(x) = e^{\lambda x}$ for $x < 0$, $f(x) = 0$ for $x > 0$, then

\begin{equation}
\hat f(k) =  \frac{1}{\sqrt{2\pi}(ik - \lambda)}
\end{equation}


These are functions which decay exponentially at infinity.  Consider next a rectangular pulse $r_a(x)$, where $r_a(x) = 1/a$ for $x \in [-a/2,a/2]$,  and where $r_a(x) = 0$ in the complement of the interval $[-a/2,a/2]$.  The height of the pulse is chosen so that the area under the graph is 1.  With this choice

\begin{equation}
  \int_{-\infty}^\infty r_a(x-\xi)f(x) dx = \text{av}_{a, \xi}(g)
\end{equation}

where the average is the average of the function on an interval of width $a$ with center $\xi$.  The Fourier transform is given by 

\begin{align}
\frac{1}{a}\int_{-a/2}^{a/2} e^{-ikx} dx
  &= \frac{1}{-ika} e^{-ikx} \Big\vert_{x=-a/2}^{x=a/2} \\
  &= (2/ka)\sin ka/2 \\
  &= \text{sinc}(ka/2)
\end{align}

Thus the Fourier transform of the rectangular pulse is the sinc function, up to a scale factor:

\begin{equation}
\hat r_a(k) = \frac{1}{\sqrt{2\pi}}\text{sinc}(ka/2).
\end{equation}

Consider now the limit

\begin{equation}
\lim_{a \to 0}\int_{-\infty}^\infty r_a(x-\xi) f(x) dx = \lim_{a\to 0} av_{a,\xi} f = f(\xi)
\end{equation}

We ask: \emph{can we pass the limit under the integral sign, and if so, what is that limit?} The answer is yes, provided that we view convergence in the sense of convergence for linear functionals, the functional being

\begin{equation}
f \mapsto \int_{-\infty}^\infty r_a(x-\xi) f(x) dx 
\end{equation}

Given that caveat, we write

\begin{equation}
\lim_{a \to 0} r_a(x-\xi)  = \delta(x - \xi)
\end{equation}

This is the famous Dirac delta function, which we view mathematically as a distribution and physically as the idealization of a unit area spike concentrated near $\xi$.
It is characterized by its action on functions:

\begin{equation}
\label{diracdelta1}
 \int_{-\infty}^\infty \delta(x-\xi)f(x)dx = f(\xi)
\end{equation}



We ask next: \emph{what is the Fourier transform of $\delta$?}
A tentative answer is that it is the limit of the Fourier transforms of the rectangular pulses $r_a(x)$.  Therefore let us think about the limit of the functions $\text{sinc}(ak/2)$ as $a$ tends to zero.  The first node to the right of the origin occurs at $k = 2\pi/a$.  Thus the width of the principal lobe of the sinc function, which has height 1, increases without bound as $a$ tends to zero.  In other words, 

\begin{equation}
\lim_{a\to 0} \text{sinc}(ka/2)  = 1
\end{equation}

We conclude that 

\begin{equation}
\hat \delta = \frac{1}{\sqrt{2\pi}}
\end{equation}

where equality is equality of distributions.

\subsection{ODE's and Green's functions}

The Fourier transform satisfies a plethora of beautiful and useful identities.  We discuss just a few of these here, then give an application to solving ODE's with constant coefficients.  First, the Fourier transform of a derivative:

\begin{align}
\caF(f')(k) &= \frac{1}{\sqrt{2\pi}}\int_{\infty}^{\infty} f'(x) e^{-ikx}dx \\
&= -ik\frac{1}{\sqrt{2\pi}}\int_{\infty}^{\infty} f'(x) e^{-ikx}dx
\end{align}

We integrated by parts, assuming that the function $f(x)$ and its derivative decay at infinity at least as fast as $(1 + |x|)^{1/2 + \epsilon}$.  Thus differentiation of a function corresponds to multiplication of its Fourier transform by $-ik$:

\begin{equation}
\caF(f')(k) = -ik\caF(f)
\end{equation}

More generally, consider any polynomial $P(s)$.  The $P(d/dx)$ is a constant-coefficient differential operator.  We have

\begin{equation}
\caF(P(d/dx)f)(k) = P(-ik)\caF(f)
\end{equation}

The fact the Fourier transform converts differentiation into multiplication means that differential equations can be solved by a combination of ordinary algebra and Fourier analysis.  Consider, for example, the first order equation 

\begin{equation}
\label{ode1}
 u' - \lambda u = f
\end{equation}

Its Fourier transform is

\begin{equation}
-ik \hat u - \lambda \hat u = \hat f
\end{equation}

Solving for $\hat u$, we have

\begin{equation}
\hat u = \frac{-\hat f}{ik + \lambda}
\end{equation}

Applying the inverse Fourier transform, we have

\begin{align}
u(x) &= \caF^{-1}\frac{-\hat f}{ik + \lambda} \\
 &= \frac{-1}{\sqrt{2\pi}} \int_{-\infty}^{\infty} \frac{ \hat f(k) }{ik + \lambda}e^{ikx} dk \\
 &= \frac{-1}{2\pi} \int_{-\infty}^{\infty} \left[  \int_{-\infty}^{\infty} f(x')e^{-ikx} dx' \right] \frac{e^{ikx}}{ik + \lambda}dk \\
&=  \frac{-1}{2\pi} \int_{-\infty}^{\infty} \left[  \int_{-\infty}^{\infty} \frac{e^{ik(x-x')}} {ik + \lambda} dk \right] f(x')dx'
\end{align}

Thus we have

\begin{equation}
\label{convolution1}
u(x) = \int_{-\infty}^{\infty} G(x-x') f(x') dx'
\end{equation}

where 

\begin{equation}
\label{green1}
G(x-x')  = \frac{-1}{2\pi} \int_{-\infty}^{\infty} \frac{e^{ik(x-x')}} {ik + \lambda}dk
\end{equation}

From this solution to a simple problem, many lessons can be learned.  First, notice the form of  \eqref{convolution1}/  It is the \term{convolution} of two functions $G(x)$ and $f(x)$.

The general definition is

\begin{equation}
f*g(x) = \int_{-\infty}^{\infty} f(x-y)g(y)dy
\end{equation}

The function $G(x)$ in \eqref{green1} is called the \term{Green's function}.  Thus the solution to equation \eqref{ode1} is given by convolution with the Green's function:

\begin{equation}
u = G*f
\end{equation}

There is more to say about the Green's function.  First, note what happens when we differentiate a convolution:

\begin{align}
(f*g)'(x) &= \frac{d}{dx} \int_{-\infty}^{\infty} f(x-y)g(y)dy \\
&= \int_{-\infty}^{\infty} \frac{d}{dx}  f(x-y)g(y)dy \\
\end{align}

so that 

\begin{equation}
(f*g)'(x)= f'*g(x)
\end{equation}

This identity holds more generally for any differential operator with constant coefficients:

\begin{equation}
L(f*g)(x)= (Lf)*g(x)
\end{equation}

Returning to our equation $Lu = f$, where $Lu  = u' -\lambda u$, we have $u = G*f$ as general solution.  Substitute back into the ODE to obtain $L(G*f) = f$.  
Apply the above identity to write this as $LG*f = f$.
Here $f$ is abitrary (within reason) and so $LG$ reveals itself as the identity element for the operation of convolution.  The question is: \emph{is there such an object?}
There is  a hint in equation  \eqref{diracdelta1}, which looks almost like a convolution.  Now the delta function is (among other things) the limit of a sequence of even functions, and therefore is itself even: $\delta(-x) = \delta(x)$.  Thus we may write \eqref{diracdelta1}as

\begin{equation}
\label{diracdelta2}
 \int_{-\infty}^\infty \delta(\xi-x)f(x)dx =  f(\xi)
\end{equation}

In other words,

\begin{equation}
 \delta*f = f.
\end{equation}

If $(LG)* f = f$ for all $f$, then $LG = \delta$.  We conclude that \emph{the Green's function for $Lu = f$ is a solution of $LG = \delta$}.  This solution (which is not be unique if $L$ has a null space), is called the \term{fundamental solution}.  From it, all other solutions are deduced by convolution.


\subsection{More about convolution}

The operation of convolution satisfies many pleasant and useful properties.  One is that convolution of $g$ with $f$ tends to smooth out $g$ and increase its support.  
To illustrate this, let $f = r_a$ be  rectangular pulse of unit area supported on $[-a/2, a/2]$ considered above.  For any even function $f$, we have 

\begin{align}
(f*g)(x) &= \int_{-\infty}^\infty  f(x - y) g(y) dy \\
 &= \int_{-\infty}^\infty  f(y -x) g(y) dy
\end{align}

Let $(T_a f)(y) = f(x-a)$ be the translation operator.  Thus the graph of $T_af$ is the graph of $f$ shifted $a$ units to the right, and the integral above can be written as

\begin{equation}
(f*g)(x) = \int_{-\infty}^\infty T_x(f)(y)g(y)  dy
\end{equation}

Therefore

\begin{equation}
(r_af*g)(x) =\frac{1}{a} \int_{x - a/2 }^{x + a/2 } f(y)g(y)  dy = \overline{g_a}(x),
\end{equation}

where $ \overline{g_a}(x) $ is the average of $g(x)$ on $[x - a/2, x + a/2]$.  Averaging a function smooths it out, addressing the first stated property.  It also increases support.  If $g$ is supported on the interval $[b,c]$, then $r_a*g$ is supported on the larger interval $b - a, c + a]$.  In general, the large the support of $f$, where $f(x) \ge 0$ for all $x$, the large is the support of $f*g$.  Indeed, if $f$ is supported on $[b,c]$ and the width of the support of $f$ is $d$, then $f*g$ is supported on $[b - d, c + d]$ -- an interval larger by $2d$ units

\image{http://psurl.s3.amazonaws.com/images/jc/convolution2-4598.png}{Convolution}{align: center, width: 400}

\subsection{References}

\href{http://ocw.mit.edu/courses/physics/8-05-quantum-physics-ii-fall-2013/lecture-notes/MIT8_05F13_Chap_04.pdf}{Dirac's Bra and Ket notation} -- Notes from B. Zwiebach's course at MIT

\href{http://www.physics.iitm.ac.in/~labs/dynamical/pedagogy/vb/delta.pdf}{All about the Dirac delta function} -- V. Balakrishnan, IIT Madras

\href{http://math.arizona.edu/~kglasner/math456/fouriertransform.pdf}{Fourier transform techniques} -- U. Arizona notes

\href{https://www.math.utah.edu/~gustafso/s2013/3150/pdeNotes/fourierTransorm-PeterOlver2013.pdf}{Fourier transform} -- 

\href{http://www.physics.rutgers.edu/~steves/501/Lectures_Final/Lec06_Propagator.pdf}{Olver notes, Free particle propagator}

\href{http://www.reed.edu/physics/faculty/wheeler/documents/Miscellaneous%20Math/Delta%20Functions/Simplified%20Dirac%20Delta.pdf}{Delta function} -- Reed college notes

