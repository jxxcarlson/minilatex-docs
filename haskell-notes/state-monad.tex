beginMetadata:
{
    "id": "fc3d0cf0-2b87-402e-a0f6-e883d258d349",
    "documentNumber": 205,
    "author": "jxxcarlson",
    "title": "The State Monad",
    "path": "haskell-notes/state-monad.tex",
    "tags": [],
    "keyString": "the state monad a=jxxcarlson haskell-notes/state-monad.tex ",
    "timeCreated": 1602192263582,
    "timeModified": 1611320028708,
    "public": true,
    "collaborators": [],
    "docType": "miniLaTeX",
    "versionNumber": 3,
    "versionDate": 1611320577698
}
endMetadata
\setcounter{section}{4}

\section{The State Monad}

\innertableofcontents

A \term{stateful computation} is a function of type $s \to (a,s)$, where $s$ is the \term{state} and $a$ is the value of the computation.  As an example, consider a stack, defined by

\begin{verbatim}
type Stack = [Int]

push : Int -> Stack -> ((), Stack)
push k ks = ((), k:ks)

pop : Stack -> (Int, Stack)
pop (k:ks) = (k, ks)
\end{verbatim}

Thus both \code{push 5} and \code{pop} are examples of stateful computations:

\begin{verbatim}
> push 3  []
((), [3])

> pop [3, 2, 1]
(3, [2,1])
\end{verbatim}


\subsection{Chaining computations}

We cannot chain stateful computations as is, since the domain and the codomain of successive functions do not match.  The chaining problem is solved by wrapping stateful computations in the \term{State monad}, \code{State s a}.  In our discussion we shall use home-brew version of this monad, \code{ST s a} for which we build everything from scratch so as to better understand its operation.

Using monadic versions of \code{push} and \code{pull}, computations can be chained like this:


\begin{verbatim}
 > app (push 1 >> push 2 >> push 3 >> pop) []
(3,[2,1])
\end{verbatim}

An expressiion like \code{(push 1 >> push 2 >> push 3 >> pop)} has the type of the last term, in this case \code{ST Stack Int}. The function \code{app} has type \code{ST s a -> s -> (a, s)}.   The role of \code{app} is to extract the stateful computation (a function) wrapped by the monad so that it can be applied to an ordinary non-monadic argument, in this case the empty list.  Here is how the stack operations function individually:

\begin{verbatim}
> app (push 5) [1,2,3]
((),[5,1,2,3])

> app pop [1,2,3]
(1,[2,3])
\end{verbatim}

\strong{Note.} In the "real" \code{State} monad, onr sas \code{runState}, not \code{app}.

\subsection{Making ST into a monad}


We now show how the homebrew version of \code{State} can be made into a monad, thereby making it possible to chain operations like \code{push} and \code{pop} with the operator \code{>>}.  The first step is to make the definition

\begin{verbatim}
newType ST s a = ST { app :: s -> (a, s) }
\end{verbatim}

The construct \code{ST}  wraps a stateful computation in the state type \code{ST}, which is record with one field..  Monadic versions of \code{push} and \code{pop}  are now defined by applying this  constructor to a lambda function:

\begin{verbatim}
push k = ST (\s -> ((), k::s)

pop = ST (\(k:ks) -> (k, ks))
\end{verbatim}

To make \code{ST} into a monad, we first make it into a functor:

\begin{verbatim}
instance Functor (ST s) where
    fmap f s = ST $ \t -> let (a,s') = app s t
                          in ( f a, s')
\end{verbatim}

Note that the right-hand side of the definition begins with \code{ST} applied to a stateful computation \code{\bs{t }-> ... }, as the definition of \code{ST} instructs us. Further inside the right-hand side, the code \code{let (a, s') = app s t} unwraps the stateful computation \code{s}, producing the result \code{a} and the new state \code{s'}.  Applying \code{f} to the result yields the transformed result, as required.


To make \code{ST} into an applicative functor, we define the operation of \code{(<*>)}:

\begin{verbatim}
instance Applicative (ST s) where
stf <*> stx = ST (\s ->
  let (f, s') = app stf s
      (x,s'') = app stx s'
  in (f x', s'')
\end{verbatim}

Finally, to make  \code{ST} into a monad, we define the return and bind operators:

\begin{verbatim}
instance Monad (ST s) where
   return a = ST $ \s -> (a, s)
   m >>= k = ST $ \s -> let (a, s') = app m s
                           in app (k a) s'
\end{verbatim}


\subsection{More examples}

Here is an example in which we pop an element off the stack, increment it, and push it back:

\begin{verbatim}
> app (pop >>= (\n -> push (n + 1))) [1,0]
((),[2,0])
\end{verbatim}

We can also use do-notation to chain computations:

\begin{verbatim}
foo :: ST Stack Int
foo  = do
        push 2
        push 3
        push 8
        pop >>= (\n -> push (2 * n ))
        pop
\end{verbatim}

Then one has

\begin{verbatim}
> app foo []
(16,  [3, 2])
\end{verbatim}

\bigskip
\italic{Draft,  October 8, 2020, revised January 22, 2021}

