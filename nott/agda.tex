beginMetadata:
{
    "id": "da627069-1668-4999-9c6c-19f14a632e29",
    "documentNumber": 286,
    "author": "jxxcarlson",
    "title": "Agda",
    "path": "nott/agda.tex",
    "tags": [],
    "keyString": "agda a=jxxcarlson nott/agda.tex ",
    "timeCreated": 1604284514165,
    "timeModified": 1606456451240,
    "public": true,
    "collaborators": [],
    "docType": "miniLaTeX",
    "versionNumber": 0,
    "versionDate": 0
}
endMetadata
\xlink{uuid:defbbdbb-d45d-46d1-96b0-371a7048ba41}{Notes on Type Theory}

\setcounter{section}{3}

\section{Agda}

\italic{WIP: More focused on Agda than type theory per se in this section.}

\begin{mathmacro}
\newcommand{\bbN}[0]{\mathbb{N}}
\newcommand{\blah}[0]{\text{\textunderscore}}
\newcommand{\sp}[0]{\space}
\newcommand{\refl}[0]{\mathop{\text{refl}}}
\newcommand{\lhs}[0]{\mathop{\text{LHS}}}
\newcommand{\rhs}[0]{\mathop{\text{RHS}}}
\end{mathmacro}

\innertableofcontents

\subsection{Using rewrite}


Another way to do the inductive proof is to use \term{rewrite}.  To explain, consider once again \eqref{pencil:paper}, focusing on the last line:

$$
 suc ((m + n) + p) \equiv suc (m + (n + p))
$$

 If we rewrite the left-hand side using the inductive hypothesis, this becomes

\begin{equation}
suc (m + (n + p) \equiv suc (m + (n + p))
\end{equation}

In this expression, the left and right-hand sides are equal, and so a proof is given by $\refl$. 
The full sequence of simplifications need for the proof then go as follows: (a) simplify the left and right sides of the equality type using the rules for addition (b) rewrite the left-hand side of resulting expression using \code{rewrite} for the inductive hypothesis. We can encode this sequence of operations in the last line of the proof as presented below.  

\begin{align}
& \text{+-assoc}' : ∀ (m\ n\ p\ : ℕ) → (m + n) + p ≡ m + (n + p) \\
& \text{+-assoc}'\ zero\ n\ p = \refl \\
&\text{+-assoc}' (suc\ m)\ n\ p\space \text{rewrite}\space \text{+-assoc}'\  m\ n\ p = \refl 
\end{align}

Note that congruence and rewrite, while achieving the same ends, operate in different ways.   Congruence tells us that if we have evidence for $e \equiv e'$, then we also have evidence for $suc\ e \equiv suc\ e'$.  Rewrite tells us that $suc\ e \equiv suc\ e'$ reduces to $suc\ e' \equiv suc\ e'$, provided that we know (by the inductive hypothesis) that $e \equiv e'$.  Finally, $suc\ e' \equiv suc\ e'$ is inhabited by $\refl$.
