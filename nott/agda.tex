beginMetadata:
{
    "id": "da627069-1668-4999-9c6c-19f14a632e29",
    "documentNumber": 286,
    "author": "jxxcarlson",
    "title": "Agda",
    "path": "nott/agda.tex",
    "tags": [],
    "keyString": "agda a=jxxcarlson nott/agda.tex ",
    "timeCreated": 1604284514165,
    "timeModified": 1611849255412,
    "public": true,
    "collaborators": [],
    "docType": "miniLaTeX",
    "versionNumber": 1,
    "versionDate": 1607014651283
}
endMetadata
\include{type-macros-jxxcarlson}

\xlink{uuid:defbbdbb-d45d-46d1-96b0-371a7048ba41}{Notes on Type Theory}

\setcounter{section}{3}

\section{Agda}

\innertableofcontents


\italic{WIP: More focused on Agda than type theory per se in this section.}

Below we collect some information on running Agda, e.g., emacs commands, as well as some examples.  All this is explained carefully in PLFA.

\subsection{Command summary}

\begin{colored}[bash]
C-c C-l: Load and type-check the file
C-c C-n expr: normalize expr

C-c C-c x: case split on variable x
C-c C-f move into the next hole.
C-c C-space: fill in hole
C-c C-r: refine with constructor
C-c C-a: automatically fill in hole
C-c C-,: goal type and context
C-c C-.: goal type, context, and inferred type
\end{colored}


\subsection{Example: the Natural Numbers}

The type of natural numbers is defined in the code below.  We put it in a file \code{nat.agda}.

\begin{colored}[elm]
data ℕ : Set where
    zero : ℕ
    suc  : ℕ → ℕ
  
{-# BUILTIN NATURAL ℕ #-}
\end{colored}

The first line corresponds to the formation rule: it announces the fact that \ccode{ℕ} is a type in the universe \ccode{Set.} The next two lines correspond to the introduction rules.  They announce the constructors.
The last line, in effect, sets up the definitions \colored{elm}{0 = zero,} \colored{elm}{1 = suc 0,} \colored{elm}{2 = suc (suc zero),} etc.  

In Emacs, type \code{C-c C-l} to load and type-check the file. Type \code{C-c C-n}, then \ccode{suc (suc 0)} to evaluate (normalize or simplify) the latter expression.  Did you get what you expected? Now add the code below.  It gives an inductive defnition of a function which adds to numbers together:

\begin{colored}[elm]
add : ℕ → ℕ → ℕ
add zero m = m
add (suc m) n = suc (add m n)
\end{colored}


Type \code{C-c C-l} to type check, then \code{C-c C-n} and \ccode{add 1 2} followed by  \ccode{add 1 2}.  The expression evaluates to \ccode{3}  Finally, add the code

\begin{colored}[elm]
_+_ : ℕ → ℕ → ℕ
_+_ zero m = m
_+_ (suc m) n = suc (add m n)
\end{colored}

Here \code{_+_} is an \term{infix operator}, so we can say \code{C-c C-n}, then \ccode{1 + 2} to obtain \ccode{3} as before.


\subsection{Binary representation of numbers}

The type \colored{elm}{Bin} given below defines a binary representation of the natural numbers.  In it, 5 is represented as \code{((<> I) O) I}, which we think of as 101.  The definition is somewhat similar to the definition of $\bbN$.  The term \ccode{<>} plays the role of zero.  The postfix operator \ccode{_O} is the binary operator "shift left."  The postfix operator \ccode{_I} is the binary operator "shift left, than add 1." 

\begin{colored}[elm]
data Bin : Set where
  <> : Bin
  _O : Bin → Bin
  _I : Bin → Bin
\end{colored}

Using pattern-matching and recursion, we define \ccode{inc}, the operator which increases the value of a binary number by 1.  There is one base case and \italic{two} recursive clauses.

\begin{colored}[elm]
inc : Bin → Bin
inc <> = <> I          -- base case
inc (x O) = x I        -- recursion, A
inc (x I) = (inc x) O  -- recursion, B
\end{colored}

If we have two ways of representing natural numbers — unary and binary — we shoud have a way of converting from one to another.  That is what \ccode{to : ℕ → Bin} and 
\ccode{from : Bin → ℕ} do.

\begin{colored}[elm]
to : ℕ → Bin
to 0 = <> O
to (suc n) = inc (to n)

from : Bin → ℕ
from <> = 0
from (x O) = double (from x)
from (x I) = suc (double  (from x))
\end{colored}

\subsubsection{Testingt}

If all the code to this point is correct, we should be able to show that 
\code{from(to n} = n} for all \code{n}.

\begin{colored}[elm]
data List (A : Set) : Set where
  []  : List A
  _∷_ : A → List A → List A

infixr 5 _∷_

range : ℕ → List ℕ
range 0 = 0 ∷ []
range (suc n) = (suc n) ∷ range n

map : ∀ {A : Set} → (A → A) → List A → List A
map f [] = []
map f (x ∷ xs) = (f x) ∷ map f xs
\end{colored}

Then

\begin{colored}[elm]
map (λ x → (from (to x))) (range 10)
-- produces the below
10 ∷ 9 ∷ 8 ∷ 7 ∷ 6 ∷ 5 ∷ 4 ∷ 3 ∷ 2 ∷ 1 ∷ 0 ∷ []
\end{colored}

Of course proofs are better than tests.  We first aim to prove the proposition that \colored{elm}{from} is a left inverse of \colored{elm}{to}:

\begin{colored}[elm]
id-from-to : ∀ (n : ℕ) → from(to n) ≡ n
\end{colored}

We prove this by induction, addressing first the type

\begin{colored}[elm]
id-from-to 0 = from(to 0) ≡ 0
\end{colored}

The left-and side reduces as follows

\begin{colored}[elm]
from(to 0)
from(<> O)
double(from <>)
double 0
0
\end{colored}

Thus the type \colored{elm}{from(to 0) ≡ 0} is the same as the type \colored{elm}{0 ≡ 0}.  Consequently, this type is inhabited by \colored{elm}{refl}. Consider next the inductive case,

\begin{colored}[elm]
id-from-to (suc n) = from(to (suc n)) ≡ suc n
\end{colored}

We claim that the following transformations are possible, obtained by rewriting rules, which we will explain momentarily:

\begin{colored}[elm]
from(to (suc n))
from(inc(to n))  -- p-to-suc
suc(from(to n))  -- p-from-inc-to
\end{colored}

If this is so, then \colored{elm}{from(to (suc n)) ≡ suc n} is the same type as \colored{elm}{suc(from(to n) ≡ suc n}. But the congruence principle says that the type
\colored{elm}{cong suc ((from(to n) ≡ n)} is inhabited if the type \colored{elm}{from(to n) ≡ n}
But that is so by the inductive hypothesis.



\bigskip


The first step is to show that \colored{elm}{suc} and \colored{elm}{inc} are intertwined by \colored{elm}{to}:

\begin{colored}[elm]
∀ (n : ℕ) → to(suc n) ≡ inc(to n)

\end{colored}

That is, the diagram below commutes, where we have written \colored{elm}{s} for \colored{elm}{suc} and \colored{elm}{i} for \colored{elm}{inc}

\begin{CD}
ℕ @>s>>ℕ\\
@VVtoV @VVtoV\\
\text{Bin} @>i>> \text{Bin}
\end{CD}

The proof is elementary:

\begin{colored}[elm]
p-to : ∀ (n : ℕ) → to(suc n) ≡ inc(to n)
p-to n = refl
\end{colored}

In essence, the left and right-hand sides simplify to the same thing.

The proposition \code{id-from-to} is a type, and the code below, which leads off with a sequence of lemmas, shows it is inhabited:

\begin{colored}[elm]
-- ℕ LEMMAS

double : ℕ → ℕ
double 0 = 0
double (suc m) = suc (suc (double m))

p-double : ∀ (n : ℕ) → double(suc n) ≡ suc(suc(double n))
p-double n = refl
\end{colored}

\begin{colored}[elm]
-- BIN LEMMAS

p-fromO : ∀ (b : Bin) → from(b O) ≡ double(from b)
p-fromO b = refl

p-bO : ∀ (b : Bin) → from (b O) ≡ double (from b)
p-bO <> = refl
p-bO (b O) = refl
p-bO (b I) = refl

p-bI : ∀ (b : Bin) → from (b I) ≡ suc (double (from b))
p-bI <> = refl
p-bI (b O) = refl
p-bI (b I) = refl

p-from-inc : ∀ ( b : Bin ) → from(inc b) ≡ suc(from b)
p-from-inc <> = refl
p-from-inc (b O)  rewrite pbO b = refl
p-from-inc (b I) rewrite pbI b | pdouble (from b) | 
      p-from-inc b | pdouble (from b) = refl

pfi-n : ∀ (n : ℕ) → from(inc(to n)) ≡ suc(from(to n))
pfi-n n = p-from-inc(to n)

id-from-to : ∀ (n : ℕ) → from(to n) ≡ n
id-from-to 0 = refl
id-from-to (suc n) rewrite pto n | pfi-n n | 
    cong suc (id-from-to n) = refl
\end{colored}

\subsection{Using rewrite}


Another way to do the inductive proof is to use \term{rewrite}.  To explain, consider this step in the proof of the associative law:

$$
 suc ((m + n) + p) \equiv suc (m + (n + p))
$$

 If we rewrite the left-hand side using the inductive hypothesis, this becomes

\begin{equation}
suc (m + (n + p) \equiv suc (m + (n + p))
\end{equation}

In this expression, the left and right-hand sides are equal, and so a proof is given by $\refl$. 
The full sequence of simplifications need for the proof then go as follows: (a) simplify the left and right sides of the equality type using the rules for addition (b) rewrite the left-hand side of resulting expression using \code{rewrite} for the inductive hypothesis. We can encode this sequence of operations in the last line of the proof as presented below.  

\begin{align}
& \text{+-assoc}' : ∀ (m\ n\ p\ : ℕ) → (m + n) + p ≡ m + (n + p) \\
& \text{+-assoc}'\ zero\ n\ p = \refl \\
&\text{+-assoc}' (suc\ m)\ n\ p\space \text{rewrite}\space \text{+-assoc}'\  m\ n\ p = \refl 
\end{align}

Note that congruence and rewrite, while achieving the same ends, operate in different ways.   Congruence tells us that if we have evidence for $e \equiv e'$, then we also have evidence for $suc\ e \equiv suc\ e'$.  Rewrite tells us that $suc\ e \equiv suc\ e'$ reduces to $suc\ e' \equiv suc\ e'$, provided that we know (by the inductive hypothesis) that $e \equiv e'$.  Finally, $suc\ e' \equiv suc\ e'$ is inhabited by $\refl$.
