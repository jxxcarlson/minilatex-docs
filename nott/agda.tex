beginMetadata:
{
    "id": "da627069-1668-4999-9c6c-19f14a632e29",
    "documentNumber": 286,
    "author": "jxxcarlson",
    "title": "Agda",
    "path": "nott/agda.tex",
    "tags": [],
    "keyString": "agda a=jxxcarlson nott/agda.tex ",
    "timeCreated": 1604284514165,
    "timeModified": 1612296380227,
    "public": false,
    "collaborators": [],
    "docType": "miniLaTeX",
    "versionNumber": 6,
    "versionDate": 1612298913180
}
endMetadata
\include{type-macros-jxxcarlson}

\xlink{uuid:defbbdbb-d45d-46d1-96b0-371a7048ba41}{Notes on Type Theory}

\setcounter{section}{3}

\section{Agda}

\innertableofcontents


\italic{WIP: More focused on Agda than type theory per se in this section.}

Below we collect some information on running Agda, e.g., emacs commands, as well as some examples.  All this is explained carefully in PLFA.

\subsection{Command summary}

\begin{colored}[bash]
C-c C-l: Load and type-check the file
C-c C-n expr: normalize expr

C-c C-c x: case split on variable x
C-c C-f move into the next hole.
C-c C-space: fill in hole
C-c C-r: refine with constructor
C-c C-a: automatically fill in hole
C-c C-,: goal type and context
C-c C-.: goal type, context, and inferred type
\end{colored}


\subsection{Example: the Natural Numbers}

The type of natural numbers is defined in the code below.  We put it in a file \code{nat.agda}.

\begin{colored}[elm]
data ℕ : Set where
    zero : ℕ
    suc  : ℕ → ℕ
  
{-# BUILTIN NATURAL ℕ #-}
\end{colored}

The first line corresponds to the formation rule: it announces the fact that \ccode{ℕ} is a type in the universe \ccode{Set.} The next two lines correspond to the introduction rules.  They announce the constructors.
The last line, in effect, sets up the definitions \colored{elm}{0 = zero,} \colored{elm}{1 = suc 0,} \colored{elm}{2 = suc (suc zero),} etc.  

In Emacs, type \code{C-c C-l} to load and type-check the file. Type \code{C-c C-n}, then \ccode{suc (suc 0)} to evaluate (normalize or simplify) the latter expression.  Did you get what you expected? Now add the code below.  It gives an inductive defnition of a function which adds to numbers together:

\begin{colored}[elm]
add : ℕ → ℕ → ℕ
add zero m = m
add (suc m) n = suc (add m n)
\end{colored}


Type \code{C-c C-l} to type check, then \code{C-c C-n} and \ccode{add 1 2} followed by  \ccode{add 1 2}.  The expression evaluates to \ccode{3}  Finally, add the code

\begin{colored}[elm]
_+_ : ℕ → ℕ → ℕ
_+_ zero m = m
_+_ (suc m) n = suc (add m n)
\end{colored}

Here \code{_+_} is an \term{infix operator}, so we can say \code{C-c C-n}, then \ccode{1 + 2} to obtain \ccode{3} as before.


\subsection{Binary representation of numbers}

The type \colored{elm}{Bin} given below defines a binary representation of the natural numbers.  In it, 5 is represented as \code{((<> I) O) I}, which we think of as 101.  The definition is somewhat similar to the definition of $\bbN$.  The term \ccode{<>} plays the role of zero.  The postfix operator \ccode{_O} is the binary operator "shift left."  The postfix operator \ccode{_I} is the binary operator "shift left, than add 1." 

\begin{colored}[elm]
data Bin : Set where
  <> : Bin
  _O : Bin → Bin
  _I : Bin → Bin
\end{colored}

Using pattern-matching and recursion, we define \ccode{inc}, the operator which increases the value of a binary number by 1.  There is one base case and \italic{two} recursive clauses.

\begin{colored}[elm]
inc : Bin → Bin
inc <> = <> I          -- base case
inc (x O) = x I        -- recursion, A
inc (x I) = (inc x) O  -- recursion, B
\end{colored}


\subsubsection{Conversion}

If we have two ways of representing natural numbers — unary and binary — we shoud have a way of converting from one to another.  That is what \ccode{to : ℕ → Bin} and 
\ccode{from : Bin → ℕ} do.

\begin{colored}[elm]
to : ℕ → Bin
to 0 = <> O
to (suc n) = inc (to n)

from : Bin → ℕ
from <> = 0
from (x O) = double (from x)
from (x I) = suc (double  (from x))
\end{colored}

We expect that if we convert from \ccode{ ℕ } to \ccode{Bin} and back that obtain what we started with.  This expectation can be formulated as a type, that is, as a proposition:

\begin{colored}[elm]
from-to : ∀ (n : ℕ) → from(to n) ≡ n
\end{colored}

To prove this proposition is to construct an inhabitant.  Then we know that \ccode{from} is a \term{left-inverse} of \ccode{to.} We also expect that it be a right inverse:

\begin{colored}[elm]
to-from : ∀ (b : Bin) → to(from b) ≡ b
\end{colored}


\subsubsection{Proof}

Consider first the base case, 

\begin{colored}[elm]
from-to 0 = from(to 0) ≡ 0
\end{colored}

The expression \ccode{from(to 0)} normalizes to \ccode{0,} so the right-hand side is equal to \ccode{0 ≡ 0.} This type is inhabited by \ccode{refl.} For the inductive case, we have 

\begin{colored}[elm]
from-to (suc n) = from(to(suc n)) ≡ suc n
\end{colored}

We claim that the following transformations of the left-hand side are possible, where each line follows from the one above by some as-yet-undefined rule:

\begin{colored}[elm]
from(to(suc n)) 
from(inc(to n)) -- bring out suc: need a lemma
suc(from(to n)) -- bring out inc: need a lemma 
\end{colored}

Here is the first lemma:

\begin{colored}[elm]
to-suc : ∀ (n : ℕ) → to(suc n) ≡ inc(to n)
to-suc n = refl
\end{colored}

Apply the defintion of \ccode{to} to the left-hand side of the type \ccode{ to(suc n) ≡ inc(to n)} to obtain \ccode{inc(to n) ≡ inc(to n).} This type is inhabited by \ccode{refl}. 

For the second transformation, we need the lemma

\begin{colored}[elm]
from-inc-to : ∀ (n : ℕ) → from(inc(to n)) ≡ suc(from(to n))
\end{colored}

Because of the inner part \ccode{to n} which is common the left and right-hand sides of the type, it is enough to prove

\begin{colored}[elm]
from-inc-to : ∀ (b : Bin) → from(inc b) ≡ suc(from b)
\end{colored}

\subheading{Proof of from-inc}

The base case is \ccode{from-inc <> = from(inc <>) ≡ suc(from <>).}  It is inhabited by  \ccode{repl}, as is the first inductive case 
\ccode{from(b O) ≡ from(inc(b O)) ≡ suc(from(b O)).}  For the second inductive case,  the type is

\begin{colored}[elm]
from(b I) ≡ from(inc(b I)) ≡ suc(from(b I))
\end{colored}

Our strategy is to first apply obvioius tranformations to bring both left and right-hand sides to a state in which \ccode{b}, not \ccode{b O} or \ccode{b I} is the argument.  Doing this, we find that that 

\begin{colored}[elm]
from(b I) ≡ double(suc(from(inc b))) ≡ suc(suc(double(from b))) 
\end{colored}

We can use the lemma to be proved to rewrite the expression \ccode{from(inc b)} in the left-hand side of the equality type.  Doing so, we find that

\begin{colored}[elm]
from(b I) ≡ double(suc(from b)) ≡ suc(suc(double(from b))) 
\end{colored}

The type in question has the form \ccode{2(x + 1) ≡ 2 + 2x}, where \ccode{x = from b}.  Therefore this type has the form \ccode{y ≡ y.}  It is inhabited by \ccode{refl.}. In conclusion, we have 

\begin{colored}[elm]
from-inc : ∀ ( b : Bin ) → from(inc b) ≡ suc(from b)
from-inc <> = refl
from-inc (b O)  = refl
from-inc(b I) rewrite from-inc b = refl
\end{colored}

EXPLANATION ....

\subsubsection{Completing the proof}

\begin{colored}[elm]
from-to : ∀ (n : ℕ) → from(to n) ≡ n
from-to 0 = refl
from-to (suc n) rewrite to-suc n | from-inc-to n | 
    cong suc (from-to-id n) = refl

\end{colored}





We now set about showing that the type \ccode{id-from-to} is inhabited.  This we do by constructing, via induction, a function of that type.  In the base case of \ccode{n = 0}, the type is

\begin{colored}[elm]
id-from-to 0 = from(to 0) ≡ 0
\end{colored}

The left-and side reduces as follows

\begin{colored}[elm]
from(to 0)
from(<> O)
double(from <>)
double 0
0
\end{colored}

Thus the type \colored{elm}{from(to 0) ≡ 0} is the same as the type \colored{elm}{0 ≡ 0}.  Consequently it is inhabited by \colored{elm}{refl.} In the inductive case, the type is

\begin{colored}[elm]
id-from-to (suc n) = from(to (suc n)) ≡ suc n
\end{colored}

We claim that the following transformations of the left-hand side are possible, each obtained by rewriting rules, which we will explain momentarily:

\begin{colored}[elm]
from(to (suc n))
from(inc(to n))  -- le-to-suc: to suc => inc to
suc(from(to n))  -- le-from-inc-to: from inc => suc from
\end{colored}

If this is so, then \colored{elm}{from(to (suc n)) ≡ suc n} is the same type as \colored{elm}{suc(from(to n) ≡ suc n}.  Recall the congruence principle from the previous section:

\begin{indent}
If $e : x \equiv y$ then $cong\ f\ e: f\ x \equiv f\ y$
\end{indent}

The principle asserts the existence of a function \ccode{cong} that maps terms of type  \ccode{x ≡ y} to terms of type \ccode{f x ≡ f y}.   That is, if the type \ccode{x ≡ y}  is inhabited by some term \ccode{e}, then the type \ccode{f x ≡ f y} is inhabited by a term \ccode{cong f e}
By induction, we suppose that \ccode{from(to n) ≡ n} is inhabited.  By the congruence principle, we conclude that \ccode{suc(from(to n))  ≡ suc n} is in inhabited. Thus, assuming proofs of the lemmas \ccode{le-to-suc} and \ccode{le-from-inc-to}above, we have a proof of \ccode{id-from-to}.



\subsection{Associativity of addition with rewrite}


Another way to do the inductive proof is to use \term{rewrite}.  To explain, consider this step in the proof of the associative law:

$$
 suc ((m + n) + p) \equiv suc (m + (n + p))
$$

 If we rewrite the left-hand side using the inductive hypothesis, this becomes

\begin{equation}
suc (m + (n + p) \equiv suc (m + (n + p))
\end{equation}

In this expression, the left and right-hand sides are equal, and so a proof is given by $\refl$. 
The full sequence of simplifications need for the proof then go as follows: (a) simplify the left and right sides of the equality type using the rules for addition (b) rewrite the left-hand side of resulting expression using \code{rewrite} for the inductive hypothesis. We can encode this sequence of operations in the last line of the proof as presented below.  

\begin{align}
& \text{+-assoc}' : ∀ (m\ n\ p\ : ℕ) → (m + n) + p ≡ m + (n + p) \\
& \text{+-assoc}'\ zero\ n\ p = \refl \\
&\text{+-assoc}' (suc\ m)\ n\ p\space \text{rewrite}\space \text{+-assoc}'\  m\ n\ p = \refl 
\end{align}

Note that congruence and rewrite, while achieving the same ends, operate in different ways.   Congruence tells us that if we have evidence for $e \equiv e'$, then we also have evidence for $suc\ e \equiv suc\ e'$.  Rewrite tells us that $suc\ e \equiv suc\ e'$ reduces to $suc\ e' \equiv suc\ e'$, provided that we know (by the inductive hypothesis) that $e \equiv e'$.  Finally, $suc\ e' \equiv suc\ e'$ is inhabited by $\refl$.


\subsection{Appendix: Tests}

While proofs are the goal, texts are sometimes usefl.  A cheap and unsuccessufl test can tell us that looking for a proof is futile, so we need to rethink things.

The obvious test is for whether \ccode{from} is the left inverse of \ccode{to} is to compute, say, \ccode{from(to 5)}.  It has value 5, thereby confirming the propostion in one case. This is reassuring. Better would be to check a buch of cases.  We can do this by applying the function  \ccode{λ x → (from (to x))} to a list of numbers.  To do end, let us define the type of lists of terms of type \code{A}:

\begin{colored}[elm]
data List (A : Set) : Set where
  []  : List A
  _∷_ : A → List A → List A

infixr 5 _∷_
\end{colored}

Here \ccode{[]} is the empty list, and \ccode{a::list}  tacks the element \ccode{a} onto the head of list \ccode{list.}  To write \ccode{::} in emacs, we write \code{\bs{::}} so as to get a single unicode synbol.  The next task is to craft a function \ccode{range} for creating lists of consecutive numbers.  Our aim is to be able to say \ccode{range 2} to produce the list \ccode{2 :: 1 :: 0 :: [].} The code for this is, once agiain, an exercise in pattern-matching and recursion:

\begin{colored}[elm]
range : ℕ → List ℕ
range 0 = 0 ∷ []
range (suc n) = (suc n) ∷ range n
\end{colored}

Finally, as in all functional languages, we want a function \ccode{map} that applies a given function of type \ccode{A → A} to the elements of a list of type \ccode{A}.  Then we can say \ccode{map inc (range 2)} to obtain \ccode{ 3 :: 2 :: 1 :: []. }. Here is the code for \ccode{map:}


\begin{colored}[elm]
map : ∀ {A : Set} → (A → A) → List A → List A
map f [] = []
map f (x ∷ xs) = (f x) ∷ map f xs
\end{colored}

With all this in hand, our test is easy to write.  Note the use of lambda.  It gives a way of converting an expression into a function:

\begin{colored}[elm]
map (λ x → (from (to x))) (range 10)
-- produces the below
10 ∷ 9 ∷ 8 ∷ 7 ∷ 6 ∷ 5 ∷ 4 ∷ 3 ∷ 2 ∷ 1 ∷ 0 ∷ []
\end{colored}