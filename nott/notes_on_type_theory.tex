beginMetadata:
{
    "id": "defbbdbb-d45d-46d1-96b0-371a7048ba41",
    "documentNumber": 263,
    "author": "jxxcarlson",
    "title": "Notes on Type Theory",
    "path": "nott/notes_on_type_theory.tex",
    "tags": [
        "agda",
        "type theory",
        "master",
        "plfa",
        "nott"
    ],
    "keyString": "notes on type theory a=jxxcarlson nott/notes_on_type_theory.tex t=agda t=type theory t=master t=plfa t=nott",
    "timeCreated": 1603351989554,
    "timeModified": 1604368432829,
    "public": true,
    "collaborators": [],
    "docType": "miniLaTeX",
    "versionNumber": 0,
    "versionDate": 0
}
endMetadata
\title{ Notes on Type Theory}

\author{James Carlson}

\maketitle

\begin{mathmacro}
\newcommand{\bbN}[0]{\mathbb{N}}
\newcommand{\blah}[0]{\text{\textunderscore}}
\newcommand{\sp}[0]{\space}
\newcommand{\refl}[0]{\mathop{\text{refl}}}
\newcommand{\lhs}[0]{\mathop{\text{LHS}}}
\newcommand{\rhs}[0]{\mathop{\text{RHS}}}
\end{mathmacro}

\italic{I am writing these notes to myself as I study Philip Wadler's \href{https://plfa.github.io/}{PLFA notes}.  They are my attempt to clairify things to myself, nothing more.  As such, there may be errors, even grievous errors, as am a beginner at this subject.  So if you find something that is wrong, please do let me know (jxxcarlson at gmail)}


((\italic{Work in Progress} ))


\subheading{Section 1: Introductipn} A brief introduction to type theory in the context of the natural numbers $\mathbb{N} = \{ 0, 1, 2, 3, \ldots \}$ in which the definitions are motivated by the Peano axioms. Sets versus types, function types and rules of inference, defnitional versus propositional eequality.


\subheading{Section 2: Propositions as Types} The Curry-Howard isomorphism and intuitionistic logic as formulated in type theory.  Worked-out example of how to prove the associative law for the natural numbers using the Peano axioms. Discussion of dependent types and the predicate calculus, the Law of the Excluded Middle, higher order logics. Brief discussion of inference rules such as congruence.


\subheading{Section 3: Agda}
The beginning of some notes on Agda. Rewrite.

\subheading{Section 4: Lambda Calculus} Bare-bones description of the syntax and semantics of the lambda calculus; normal forms and divergence of $\Omega \Omega$.  Remarks on the simply-typed lambda calculus.

\subheading{Section 4: Type Inference} A sketch of the basic idea; unification.

\subheading{Section 4: Mathematical Notes} For now, just some remarks on the connection between the theory of groupoids and type theory.

\ilink1{uuid:6dde9777-daf0-45ab-9be3-1bef5fefb13f}{Introduction}

\ilink1{uuid:4b10dfec-c827-40a0-8692-62faf5adaefe}{Propositions as Types}

\ilink1{uuid:da627069-1668-4999-9c6c-19f14a632e29}{Agda}

\ilink1{uuid:f43aace0-3184-4ec6-935f-3c0d1e08e5b8}{Lambda Calculus}

\ilink1{uuid:03eacd0f-f134-41c4-a4f2-c3f30cf327f3}{Type Inference}

\ilink1{uuid:84cb8453-93ba-43f1-a2fe-5e8baaff06eb}{Mathematical Notes}

\ilink1{uuid:008f9c82-2063-4e01-9105-03042f0c8c3c}{References}