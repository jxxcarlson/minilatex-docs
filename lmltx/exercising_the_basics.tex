beginMetadata:
{
    "id": "6a90de6a-0a10-4788-ae2b-2b78130211cc",
    "documentNumber": 274,
    "author": "jxxcarlson",
    "title": "Math Mode Exercises",
    "path": "lmltx/exercising_the_basics.tex",
    "tags": [
        "minieditor"
    ],
    "keyString": "math mode exercises a=jxxcarlson lmltx/exercising_the_basics.tex t=minieditor",
    "timeCreated": 1603901186530,
    "timeModified": 1603918291673,
    "public": true,
    "collaborators": [],
    "docType": "miniLaTeX",
    "versionNumber": 0,
    "versionDate": 0
}
endMetadata
\xlink{uuid:f8e68aa3-429a-4030-83a8-328ecb41126d}{LearningLaTeX}

\setcounter{section}{2}

\section{Math Mode Exercises}

\innertableofcontents


In this lesson we will not introduce new things, but rather exercise ourselves with more complex problems using what we already know.


\subsection{Nesting and combining}

Superscripts can be nested: $a^{b^c}$

To do this, we wrote 

\begin{verbatim}
$ a^{b^c} $
\end{verbatim}

\subsection*{Problems}

Write source text for the following:

\begin{enumerate}

\item $2^{3^4}$

\item $2^{3^{4^5}}$

\item $\lambda(k) = \sum_{j=1}^n \mu_j^k$

\item Complicated subscripts: $ A_i = \sum_{j=1}^n a_{i,j}  $

\item More complicated: $ A_i = \sum_{j=1}^n a_{i_j}  $

\end{enumerate}

For still more fun, evaluate the expressions (1) and (2) above. Pencil and paper allowed and even recommended!

\subsection{Calculus}

Let's look at this equation.

 $$ \frac{dy}{dx} = y $$

You can write source text for it using \code{\bs{frac}}.  \blue{Do this now.}

For something a bit harder, let's do something with second derivatives:

$$
  \frac{d^2y}{dx^2} = - y
$$

You have all the elements at hand: the \code{\bs{frac}} macro, and superscripts.  \blue{Write source code for this now}.

Let's move on to multi-variable calculus with this equation (the one-dimensional wave equation):

$$
\frac{\partial^2 u}{\partial t^2} =  c^2 \frac{\partial^2 u}{\partial x^2}
$$

You need one more piece of information: the macro for the partial derivative symbol is \code{\bs{partial}}. 

\subsection{Symbol Table}

There are many, many symbols in LaTeX.  As noted previously, you can look them up in various online references, e.g,

% \begin{center}
% \href{https://oeis.org/wiki/List_of_LaTeX_mathematical_symbols}{LaTeX Mathematical Symbols}
% \end{center}

\begin{center}
\href{https://katex.org/docs/supported.html}{KaTeX Supported Symbols}
\end{center}

Take a look at this now, then work on reproducing the examples below.

\strong{Sets}

$$
A \cap (B \cup C) = (A \cap B) \cup (A \cap C)
$$

\strong{Logic}

$$
\forall x \in A : \exists y \in B\space \text{ such that } y > x
$$

Here we used the \code{\bs{text}} macro for the phrase  "such that;"   We also wrote a \code{\bs{space}} to separate it from $\exists y \in B$.



\subsection{Making up your own exercises}

As with learning a musical instrument or French or Chinese or any other human language, it is important to practice, practice practice.  Especially valuable is the exercise of making up your own exercises.   

\blue{Do this now.  You can practice using the mini-editor}.  If at the moment you happen to be short on inspiration, open up a book that has a lot of mathematics in it and see if you can reproduce some of it.

