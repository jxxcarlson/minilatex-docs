beginMetadata:
{
    "id": "ec24173c-ad9b-4596-85e2-ceb70373351d",
    "documentNumber": 283,
    "author": "jxxcarlson",
    "title": "Macros",
    "path": "lmltx/macros.tex",
    "tags": [
        "minieditor"
    ],
    "keyString": "macros a=jxxcarlson lmltx/macros.tex t=minieditor",
    "timeCreated": 1604076644977,
    "timeModified": 1604076644977,
    "public": true,
    "collaborators": [],
    "docType": "miniLaTeX",
    "versionNumber": 0,
    "versionDate": 0
}
endMetadata
\xlink{uuid:f8e68aa3-429a-4030-83a8-328ecb41126d}{Learning LaTeX}

\setcounter{section}{8}

\section{Macros}

\innertableofcontents


\subsection{Math mode}

We first encountered macro definitions in the section on theorem environments:



\begin{verbatim}
\begin{mathmacro}
\newcommand{\mod}[0]{\space\text{mod}\space}
\end{mathmacro}
\end{verbatim}

In MiniLaTeX, math-mode macro definitions are listed in the \code{mathmacro} environment.
Macro definitions have the form

 \begin{verbatim}
\newcommand{MACRO-NAME}[N]{BODY}\end{verbatim}

where 


\begin{enumerate}

\item MACRO-NAME begins with a backslash.

\item  $N$ is the number of arguments of the macro

\item BODY gives  the text to be used wherever the macro is used.

\end{enumerate}
. 
\subsection{Text mode}

NOTE. If a MiniLaTeX document is exported  to standard LaTeX, the macro definitions are exported as standard LaTeX.
