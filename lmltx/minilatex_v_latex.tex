beginMetadata:
{
    "id": "f1f22387-7bb7-446e-9806-0ea09446fcd1",
    "documentNumber": 278,
    "author": "jxxcarlson",
    "title": "MiniLaTeX versus LaTeX",
    "path": "lmltx/minilatex_v_latex.tex",
    "tags": [],
    "keyString": "minilatex versus latex a=jxxcarlson lmltx/minilatex_v_latex.tex ",
    "timeCreated": 1603939626495,
    "timeModified": 1603939626495,
    "public": true,
    "collaborators": [],
    "docType": "miniLaTeX",
    "versionNumber": 1,
    "versionDate": 1607212108781
}
endMetadata
\xlink{uuid:f8e68aa3-429a-4030-83a8-328ecb41126d}{LearningLaTeX}

\setcounter{section}{6}

\section{MiniLaTeX versus LaTeX}

MiniLaTeX is a  subset of LaTeX designed for the web.  It features 

\begin{itemize}

\item No setup, no-hassle: just click control-N for a new document and start typing.  

\item Almost instantaneous rendering

\item Real-time, in-line error messages displayed in the rendered text, no log scrolling off the screen in another window.

\item Export to standard LaTeX and to PDF.  Your data is \emph{not} prisoner to the app.

\end{itemize}

The current subset is designed to handle common cases, e.g., writing problem sets, class handouts, and class notes in math and physics courses, computer science, etc.  The notes can eaily approach the size and complexity of a book.  See for example, \href{uuid:6f5a573d-5603-4470-8dc9-b0972997a6e6}{these  notes on quantum physics}.  Together with feedback from the community, we will continue to expand the subset that MiniLaTeX can handle.

