beginMetadata:
{
    "id": "f8e68aa3-429a-4030-83a8-328ecb41126d",
    "documentNumber": 273,
    "author": "jxxcarlson",
    "title": "Learning MiniLaTeX",
    "path": "lmltx/learning_minilatex.tex",
    "tags": [
        "index",
        "minieditor"
    ],
    "keyString": "learning minilatex a=jxxcarlson lmltx/learning_minilatex.tex t=index t=minieditor",
    "timeCreated": 1603900970527,
    "timeModified": 1607206101222,
    "public": true,
    "collaborators": [],
    "docType": "miniLaTeX",
    "versionNumber": 1,
    "versionDate": 1607211871779
}
endMetadata
\title{Learning MiniLaTeX}

\author{James Carlson}


\maketitle

These notes  provide a short course on LaTeX/MiniLaTeX. 

While most of the material applies to LaTeX generally, some things are specific to MiniLaTeX, the subset of LaTeX used in this app.

The lessons provide all the information needed to write documents of the kind that you find on here, from \xlink{uuid:1f39631c-cad8-45eb-9ef2-a3bb82b3ed8a}{simple docs like this} to \xlink{uuid:6f5a573d-5603-4470-8dc9-b0972997a6e6}{complex multi-part documents like this one}. The first few lessons give enough information to write simple notes.  

\subheading{MiniEditor}

The lessons use a special mini-editor which renders LaTeX as you type.  To activate it, click on the \strong{\blue{MiniEditor}} button in the footer.  To deactivate it, click on that button again. \italic{Activate the mini-editor now. Then paste the text below into the upper window (source text).}

\begin{verbatim}
Pythogoras says that $a^2 + b^2 = c^2$
\end{verbatim}

\italic{What do you see in the lower window?}

All text in the MiniEditor is lost when you sign out.  To make permanent documents, sign in as a regular user and  click on \strong{\blue{New}} in the document list on the left.  (Or type control-N.)

\subheading{Export}


You can export a document in this app to a file on your computer using the \strong{\blue{Doc Tools}} button (see footer). 
 When you do so, whatever is specific to MiniLaTeX will be translated to standard LaTeX.

\ilink1{uuid:93ebb56f-6c6c-4631-a648-c2990325fefc}{Sample documents}

\ilink1{uuid:fd6755f5-4134-4a84-8d46-2955e8f6eeb3}{Getting Started}

\ilink1{uuid:6a90de6a-0a10-4788-ae2b-2b78130211cc}{Math Mode Exercises}

\ilink1{uuid:850f0594-d77c-4b5d-ad7b-82103a4fa7e9}{Math Environments}

\ilink1{uuid:8a47f429-e221-43c8-8a6a-7ee9e0027d00}{Text Mode LaTeX}

\ilink1{uuid:f1f22387-7bb7-446e-9806-0ea09446fcd1}{MiniLaTeX versus LaTeX}

\ilink1{uuid:47a38f1e-9446-4e51-845a-cce1ed7b4077}{Images}

\ilink1{uuid:ec24173c-ad9b-4596-85e2-ceb70373351d}{Macros}