beginMetadata:
{
    "id": "f8e68aa3-429a-4030-83a8-328ecb41126d",
    "documentNumber": 273,
    "author": "jxxcarlson",
    "title": "Learning MiniLaTeX",
    "path": "lmltx/learning_minilatex.tex",
    "tags": [
        "index",
        "minieditor"
    ],
    "keyString": "learning minilatex a=jxxcarlson lmltx/learning_minilatex.tex t=index t=minieditor",
    "timeCreated": 1603900970527,
    "timeModified": 1607206101222,
    "public": true,
    "collaborators": [],
    "docType": "miniLaTeX",
    "versionNumber": 0,
    "versionDate": 0
}
endMetadata
\title{Learning MiniLaTeX}

\author{James Carlson}


\maketitle



These notes  provide a short course on learning enough LaTeX to begin writing class notes, problem sets, solution sets, etc., for courses in mathematics, physics, and so forth.  Thus it can be of use to both students and professors.

While most of the material applies to LaTeX generally, some things are specific to MiniLaTeX. However, if you export a file to standard LaTeX using the \blue{Doc Tools} button, whatever is specific to MiniLaTeX will be translated to standard LaTeX.  

The Doc Tools button is found in the footer on the left.

\subheading{The Lessons}


\strong{\highlight{Note.}} The lessons use an embedded mini-editor.  To use the mini-editor, you must be signed in as a regular user, not a guest

\ilink1{uuid:fd6755f5-4134-4a84-8d46-2955e8f6eeb3}{Getting Started}


\ilink1{uuid:6a90de6a-0a10-4788-ae2b-2b78130211cc}{Math Mode Exercises}

\ilink1{uuid:850f0594-d77c-4b5d-ad7b-82103a4fa7e9}{Math Environments}

\ilink1{uuid:8a47f429-e221-43c8-8a6a-7ee9e0027d00}{Text Mode LaTeX}

\ilink1{uuid:f1f22387-7bb7-446e-9806-0ea09446fcd1}{MiniLaTeX versus LaTeX}

\ilink1{uuid:ec24173c-ad9b-4596-85e2-ceb70373351d}{Macros}
