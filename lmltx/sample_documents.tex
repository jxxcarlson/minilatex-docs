beginMetadata:
{
    "id": "93ebb56f-6c6c-4631-a648-c2990325fefc",
    "documentNumber": 367,
    "author": "jxxcarlson",
    "title": "Sample documents",
    "path": "lmltx/sample_documents.tex",
    "tags": [
        "minieditor"
    ],
    "keyString": "sample documents a=jxxcarlson lmltx/sample_documents.tex t=minieditor",
    "timeCreated": 1607277707768,
    "timeModified": 1607279674787,
    "public": true,
    "collaborators": [],
    "docType": "miniLaTeX",
    "versionNumber": 0,
    "versionDate": 0
}
endMetadata
\xlink{uuid:f8e68aa3-429a-4030-83a8-328ecb41126d}{Learning MiniLaTeX}

\section{Sample documents}

\innertableofcontents

To see where we are going in these lessons and what you will learn, take look at the 
documents below.  For each of them, your task is to (1)  read the text and try to understand what it says, including the equations; (2) copy the text into the mini-editor; (3) compare the source and rendered text (upper and lowe windows of the mini-editor); (4) Make some edits or additons to the text (upper window) and observe the change in the lower window.

Remember: you open the mini-editor by clicking on the \strong{\blue{MiniEditor}} buttoni in the footer.

\subsection{On Falling Bodies}

\begin{verbatim}
\section{On Falling Bodiest}

Newton's law of motion, $F = ma$, relates the 
force $F$ acting on a body of mass $m$ to its 
acceleration $a$.  If one neglects frictional forces, 
gravitational attraction of an object to the earth 
near the earths surfaces is proporotional to its 
mass: $F = -mg$.  Combining the two equations, 
we find that $ma = -mg$. Cancelling common 
factors, we find that  $a = - g$.  That is, 

\begin{indent}
\italic{The accelaration of a falling body in a 
frictionless medium is independent of its mass.}
\end{indent}

\strong{Note.} 
\href{https://www.nature.com/articles/158906e0}{Aristotle's 
views on falling bodies}
\end{verbatim}

\subsection{On the Harmonic Series}

\begin{verbatim}
\section{On Infinity}

As we learn when we study infinite series, the sum

   $$
   H(n) = \sum_{k=1}^n \frac{1}{k}
   $$

\italic{diverges}, that is, $\lim_{n \to\infty} H(n) = \infty$ 
This result was known to the fourteenth century French 
scholar and cleric 
\href{https://en.wikipedia.org/wiki/Nicole_Oresme}{Nicole Oresme}. 
See also 
\href{https://web.williams.edu/Mathematics/lg5/harmonic.pdf}{this reference}.
\end{verbatim}

\subsection{On the Flight of Bumblebees}

\begin{verbatim}
\section{On the Flight of Bumblebees}

\image{https://static1.squarespace.com/static/59551c6ed482e9b2f9cfd19e/5c7520c14e17b619cc5dfbc6/5d766e02ae60d6301a5525e4/1568109389031/bumble-bee-2361336.jpg?format=1500w}{Bumblebee in Flight}{width: 500}

\href{https://www.animal-dynamics.com/ad-blog/2019/9/9/how-do-bumblebees-fly}{How Do Bumblebees Fly?}


\end{verbatim}


