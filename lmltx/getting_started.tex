beginMetadata:
{
    "id": "fd6755f5-4134-4a84-8d46-2955e8f6eeb3",
    "documentNumber": 363,
    "author": "jxxcarlson",
    "title": "Getting Started",
    "path": "lmltx/getting_started.tex",
    "tags": [
        "minieditor"
    ],
    "keyString": "getting started a=jxxcarlson lmltx/getting_started.tex t=minieditor",
    "timeCreated": 1607205963224,
    "timeModified": 1607205963224,
    "public": false,
    "collaborators": [],
    "docType": "miniLaTeX",
    "versionNumber": 0,
    "versionDate": 0
}
endMetadata
\xlink{uuid:f8e68aa3-429a-4030-83a8-328ecb41126d}{Learning LaTeX}

\section{Getting Started}

\innertableofcontents



\subsection{Directions}

Each lesson presents some part of LaTeX, along with exercises that you will do as you read the text.  To do the exercises, you will use the \italic{MiniEditor}.  Open the MiniEditor by clicking on the blue \blue{MiniEditor} button in the footer,  To close the editor, click on the button again.  \highlight{You must be signed in as a regular user to use the MiniEditor.}

\subsection{MiniEditor}

Let's use the MiniEdtior to write our first sentence in LaTeX:

\begin{enumerate}

\item Click on the \blue{MiniEditor} button. (Footer, right half).

\item Type the following in the upper window of the editor: \italic{The number \dollar\bs{sqrt 2}\dollar is irrational.}

\item What do you see in the lower window?   Is it what you expected?

\end{enumerate}

The upper window is for the \blue{Source text}.  The lower window is for the \blue{Rendered text}.   Some observations:

\begin{itemize}

\item We put \emph{inline math text} between dollar signs.  

\item The text \code{\bs{sqrt}} is a \term{macro} which is used for making square roots.  
The number 2 following the macro is its \term{argument}.

\item We often write arguments with curly braces, e.g. \code{\bs{sqrt}\texarg{2}}.  Curly braces are mandatory if the argument contains spaces, e.g., \code{\bs{sqrt}\texarg{a + b}}

\end{itemize}

\strong{Exercise.}. Write \code{\bs{sqrt}\texarg{a + b}} in the source window of the MiniEditor.  What do you see in the RenderedText window?

\subsection{More Examples} 

Type the text below into the source text window (top):

\begin{indent}
Pythagoras said that \dollar a^2 + b^2 = c^2 \dollar
\end{indent}

What do you see in the rendered text window (bottom)

Now write this in the source window:

\begin{indent}
\dollar\dollar\smallskip \bs{int_0^1 x^n dx = } \bs{frac}\texarg{1}\texarg{n+1}\smallskip\dollar\dollar
\end{indent}

So far we have learned that we can write formulas by writing the mathematical text between dollar signs or double dollar signs.  The first way is called \term{inline math mode} and the second is called \term{display math mode}.

\subsection{Comments}

If we look back at our examples, we see that we have learned quite a lot:

\begin{enumerate}

\item We write subscripts using an underscore (_) and superscripts with a caret (^).

\item We use \term{macro expression} to do integrals and fractions.  A macro expression begins with the macro itself, which has the form \code{\bs{MACRO_NAME}}.  Thus \code{\bs{int}} is a macro.  The macro is followd by zero or more \term{arguments} enclosed in curly braces.  The \code{\bs{frac}} macro has two arguments: first the numerator, then the denominator.

\end{enumerate}

Keep thse points in mind as you continue the lessons.  Note that \bs{int} is a macro with zero arguments.  Other zero-argument macros are \bs{alpha}, \bs{beta}, \bs{gamma}, etc, which give you $\alpha, \beta, \gamma, \text{etc.}$.  The macros \bs{Gamma} and \bs{Delta} do what you expect: $\Gamma, \Delta$.

\subsection{Practice}

Write source text for the formulas below:

\begin{itemize}

\item $ a^{10} + b^{10} = c^{10} $

\item $ \delta = a_1b_2c_3 $ (See coment below)

\item $\Phi = \alpha + \beta + \gamma$.

\end{itemize}

There are many macros for mathematical symbols.  You can expect yourself to know them all, so \href{https://katex.org/docs/supported.html}{here is a handy table}.  Use it (or some other source) whenever you need to a new symbol.

\subsection{More Practice}

Write source text for this formula:

$$
\sum_{n=1}^\infty \frac{1}{n} = \infty
$$

Here is some info you will need:

\begin{itemize}

\item This problem is like the example of the integral, except that you use the macro 
\code{\bs{sum}} instead of \code{\bs{int}}.

\item When a subscript or superscript has more than one character, you must enclose it in curly braces, e.g., \texarg{n=1}.

\item The macro for $\infty$ is \code{\bs{infty}}. (You will find this in the \href{https://katex.org/docs/supported.html}{TABLE}.

\end{itemize}

When you are finished with this lesson, which we will call lesson 0, you should take a break before going on to lesson 1.  This way you will learn faster, better, and enjoy yourself more.

\medskip

\highlight{NOTE.} \italic{In the menu bar, top left, you see a button labeled \blue{Source}.  If you click on it, you can view the source text of this document.  To get back to the usual view, click on \blue{Browse}.  The \blue{Source} text button is available for any public document that someone else wrote.}

\end{enumerate}