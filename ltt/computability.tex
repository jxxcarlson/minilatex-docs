\uuid{49bcc8a6-293a-49a4-9805-a7c0d6a30d1a}

\section{Computability}

The set $PR$ of \term{primitive recursive functions} $\nat^k \to \nat$ (for arbitrary $k$) is defined as follows.  First, there is a stock $X$ of basic functions: (i) the 0-ary constant function with value zero, $o'() = 0$, (ii) the 1-ary constant function with value 0, $o(x) = 0$, (iii) the projection $p^i_k$ from $\nat^k$ on the $i$-th factor, and (iv) the successor function $succ(x) = x + 1$.  Two operations of functions are defined. The first is composition.  Suppose given a function $f$ of arity $m$ and $m$ functions $g_i$ of arity $k$.  Then one defines the composite function


\begin{equation}
C(f,g_1,\ldots,g_m)(\mathbold{x}) = f(g_1(\mathbold{x}), \ldots, g_m(\mathbold{x}))
\end{equation}

The second operation, denoted $R$ for recursion, takes as input  function $f$ of arity $k+2$ and a functions $g$ of arity $k$.  It produces as output a function $h = R(g,f)$ of arity $k$, namely, the unique function satisfying the relation

\begin{align}
  h(0, \mathbold{y}) &= g(\mathbold{y}) \\
  h(x + 1, \mathbold{y}) &= f(h(x,\mathbold{y}), x, \mathbold{y}) \\
\end{align}

\begin{definition}
The set of primitive recursive functions is the smallest set of functions containing $X$ that is closed under the operations $C$ and $R$.
\end{definition}

\subsection{Examples}


\strong{Identity.}  Let $g$, of arity 0, be the function $o'$. Let $f$, of arity 2, be the function $succ \circ p^1_2$.  Then $h = R(g,f)$ of arity 1 is the unique function satisfying $h(0) = 0$ and $h(x+1) = h(x) + 1$.  Thus $\forall x: h(x) = x$.

\strong{Addition.} Let $g$, of arity 1,  be the identity function.   Let $f$, of arity 3, be the function $succ \circ p^1_3$, so that $f(x,y,z) = x + 1$.  Then $h = R(g,f)$, of arity 2, is the unique function satisfying $h(0,y) = y$ and $h(x+1,y) = h(x,y) + 1 = (x+1) + y$.  Therefore $\forall x,y: h(x,y) = x + y$.

\strong{Exercise.} (i) Show that the 0-ary onsant function $c_m() = m$ is primitive recursive; then show the same for the 1-ary constant function; (ii) Show that $h(m,n) = m\times n$ is primitive recursive; (iii) Show that $h(m,n) = m^n$ is primitive recursive.


\subsection{Counterexamples}

Define the \term{Ackerman function} by

$$
A(m,n) = \begin{cases}
               n+ 1 \ \ \text{if} \ \ m = 0\\
              A(m-1,1) \ \ \text{if} \ \ m > 0 \ \ \text{and} \ \ n = 0 \\
              A(m-1,A(m,n-1)) \ \ \text{if} \ \ m > 0 \ \ and \ \ n > 0
            \end{cases}
$$

It is a total recursive function but not primitive recursive because it grows faster than any primitive recursive function.


\href{https://www.andrew.cmu.edu/user/kk3n/complearn/chapter2.pdf}{CMU notes on primitive recursion}
