\uuid{57618eed-21d7-4b07-a60c-ded865082794}

\xlink{75617ed5-f446-4491-bab9-a7ea85e5a817}{Notes on Logic and Type Theory}


\title{Notes on Logic and Type Theory}
\author{James Carlson}
\date{December, 2018}

\maketitle

\maintableofcontents



\section{Introduction}

I'm writing these notes in an attempt to learn the subject.  I pretend that I am preparing lectures, and try to organize what I am learning so that it makes sense to me and will also make sense to my students. The goal is to cut as straight and efficient path as possibile from classical logiic to intuitionistic logic, type theory, and considerations of automatic theorem proving.  I'd like to be able to say something sensible about homotopy type theory, but that may be much too much of a stretch.

Mathematical logic was something that I became interested in in high school.  At some point, I even had a copy of Bertrand Russell'sl \italic{Principia Mathematica}.  I was never able to get into it.  Later, after a flirtation with physics, I became a mathematician, and adopted the common (and most;y ill-informed) attitudes that mathematicians have about logic, especially constructivist logic.  I was certainly not about to give up anything so powerful and wonderful as the Law of the Excluded Middle!

Many, many years later, I became deeply involved in a project to write a parser for a subset of LaTeX using Elm, a statically typed functional language.  (These notes are wriitten using the results of that project.) The power and utility, not to mention elegant beauty of its type system has made Elm a joy to work with.  As a result,  I became interested in how type theory was used  in other languages. This led to trying to understand type theory itself, and hence finally back to logic. A very long journey.

As I say, these notes (a work in progress), are primarily as a tool for me to learn the subject.  Perhaps one day, when and if complete, they will be useful to others as well.


Note: A great deal of the material is based on the excellent reference \cite{RH}, An Introduction to Mathematical Logic. I highly recommend it.
