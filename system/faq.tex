beginMetadata:
{
    "id": "58856622-061a-4498-b182-6b06e8437761",
    "documentNumber": 290,
    "author": "jxxcarlson",
    "title": "FAQ",
    "path": "system/faq.tex",
    "tags": [
        "system"
    ],
    "keyString": "faq a=jxxcarlson system/faq.tex t=system",
    "timeCreated": 1604620441892,
    "timeModified": 1607704835428,
    "public": true,
    "collaborators": [],
    "docType": "miniLaTeX",
    "versionNumber": 0,
    "versionDate": 0
}
endMetadata
\title{FAQ}

\image{https://vschool.s3.amazonaws.com/minilatex/logo-text-only.jpg}{}{float: left, width: 250}

\bigskip


\begin{verse}
Contact: James Carlson
Email: jxxcarlson@gmail.com
Tel: 617-852-7490
\end{verse}

\strong{\href{https://minilatex.lamdera.app}{minilatex.lamdera.app}} |
\strong{\href{https://minilatex.io}{minilatex.io}} |
\strong{\href{https://minilatex.lamdera.app/h/jxxcarlson}{ jxxcarlson}}


\begin{defitem}[What is MiniLaTeX?]
It is a variant of LaTeX that compiles to HTML, the language of the web.
You can view MiniLaTeX documents anywhere, on any device. How? Use the web app
\strong{\href{https://minilatex.lamdera.app}{minilatex.lamdera.app}}. 
\end{defitem}


\begin{defitem}[What are the advantages of MiniLaTeX?]

 \image{https://psurl.s3.amazonaws.com/images/jc/beats-eca1.png}{Two-frequency beats}{width: 250, float: right}

\begin{itemize}

\item No setup. Just start typing.

\item Instant feedback: text is rendered as you type.

\item Error messages presented inline in the rendered text.

\item Easy to include images.

\item Collaborative editing; a chat to facilitate that.

\item Share documents with custom pages, your home page, or by sending a link by email.

\item Integrated document search by title, author, full text

\end{itemize}

\end{defitem}



\begin{defitem}[What else?]

 Export to standard LaTeX, convert to PDF, Github integration,  \href{https://minilatex.lamdera.app/g/273}{Interactive LaTeX lessons for beginners}.

\medskip
You can also write in Math-Markdown (Markdown + LaTeX formulas).  Convert between MiniLaTeX and Math-Markdown.

\end{defitem}


\begin{defitem}[How is MiniLaTeX different from LaTeX?]

\image{https://vschool.s3.amazonaws.com/minilatex/venn.jpg}{}{float: left, width: 150}

As indicated in the figure on the left, MiniLaTeX is a  variant of LaTeX.  Most of MiniLaTeX is standard LaTeX, but a small part is not.  When a MiniLaTeX document is exported, those parts are converted to standard LaTeX, so that the resulting document can be typeset with tools such as \code{pdflatex} and \code{xelatex}.  
\medskip
MiniLaTeX does not encompass all the features of LaTeX.  Instead, it provides a subset which meets many common needs, e.g., preparing problem sets and lecture notes, as in \href{https://minilatex.lamdera.app/g/uuid:6f5a573d-5603-4470-8dc9-b0972997a6e6}{these draft notes on quantum mechanics}.




\end{defitem}

\begin{defitem}[What has MiniLaTeX been used for?]
Mostly writing lecture notes and problem sets, but not limited to that.  Another example: \href{https://minilatex.lamdera.app/g/263}{Notes on Type Theory}.
\end{defitem}


\begin{defitem}[How do I sign up?]
Go to \href{https://minilatex.lamdera.app}{minilatex.lamdera.app} . There is also read-only access for guests, e.g., your students.
\end{defitem}

\begin{defitem}[How much does MiniLaTeX cost?]
For now, all accounts are free. Eventually there will be a paid plan at a very reasonable and affordable annual cost, with special provision for students.  
\end{defitem}

%\image{https://vschool.s3.amazonaws.com/minilatex/qft.png}{QR Link to Sample MiniLaTeX Document}

\medskip

\image{https://vschool.s3.amazonaws.com/minilatex/line.jpg}{}{width: 200}

\medskip



\begin{defitem}[What is the technology behind MiniLaTeX?]
All of the code in the two compilers, MiniLaTeX to Html and Math-Markdown to Html, is written in \href{https://elm-lang.org}{Elm}, a statically typed functional language.  The \strong{\href{https://minilatex.lamdera.app}{minilatex.lamdera.app}} is also written in Elm using \href{https://dashboard.lamdera.app/}{Lamdera}, a system that allows one to write both the front and backend of the app in the same language.
\end{defitem}