beginMetadata:
{
    "id": "53b10ece-c011-4f40-83c9-7207a474db11",
    "documentNumber": 90,
    "author": "jxxcarlson",
    "title": "MiniLaTeX: App Manual",
    "path": "minilatex/app-manual.tex",
    "tags": [
        "minilatex"
    ],
    "keyString": "minilatex: app manual a=jxxcarlson minilatex/app-manual.tex t=minilatex",
    "timeCreated": 1597355235869,
    "timeModified": 1597516157522,
    "public": true,
    "collaborators": [],
    "docType": "miniLaTeX",
    "versionNumber": 2,
    "versionDate": 1597355786863
}
endMetadata

\title{MiniLaTeX: App Manual}

\maketitle

\tableofcontents


\section{Browser compatibiity}

The app works well in Chrome, Firefox, Opera, and Min.   In Safari, the display is messed up.


\section{Home Page}

To turn one of your documents int a home page, put \italic{home} in the tags field and push the button \blue{Tags} (header, upper right).  Then make sure the document is public.  Use the \blue{Public/Private} button, to the left of the Tags button.

\section{Search}

Search on parts of words, tags, authors.  A search on \italic{man} will yield this manual.  A search on \italic{man app} will yield this document, the \strong{App Manual}, as opposed to the \strong{Language Manual}.  These searches are on fragments of words in the title.  

The search \italic{t=logic} will retrieve documents tagged with the word \italic{logic}.  Use \italic{t= logic a=aristotle} to find documents tagged as logic written by aristotle.

\section{Github Integration}

This is now working, albeit in primitive form.  See the \blue{Doc Tools} button, footer, lower left. In \blue{Github Settings}, you will have to add your Github username, the repository name, and a token that gives this app access to your public repositories.  (If you know how to restrict this to a specific repository, please let me know: jxxcarlson at gmail).  You get the token by going to \italic{Settings} in your Github account.


\section{Collaboration}

\subsection{Adding and removing collaborators}

Press the \blue{Collaborators} button in the footer to raise the collaborators popup list.  Enter collaborators by user name one per line.


\subsection{Notes to collaborators}

If you say this:

\begin{verbatim}
\attachNote{HD}{Hey, this is a good first draft.
But I would refer to Blowenpuff's article on
quantum computing in the introduction.}
\end{verbatim}

It is rendered like this:

\attachNote{HD}{Hey, this is a good first draft.
But I would refer to Blowenpuff's article on
quantum computing in the introduction.}

\section{Exporting to LaTeX}

Use the \blue{Export} button in the footer.  The exported document will be saved in the Downloads folder unless you tell your computer otherwise.  If your document has images, a button \blue{Images} will appear to the right of \blue{Export}.  Click on that button to bring up a window that lists the images in your document.  Right-click on these images to download them.  They should be stored in folder \blue{image} in the same place as your exported document.


\section{Known Issues}

We are working on these.  Please let us know of others as you find them (jxxcarlson at gmail).

\begin{enumerate}

\item Links in table of contents do not work

\end{enumerate}

