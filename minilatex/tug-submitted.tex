\title{MiniLaTeX: a subset of LaTeX for the Web}
\author{James Carlson}
\date{August 8, 2020}

\maketitle

\tableofcontents


\href{https://youtu.be/TAIYpCc3VV0}{Video version of this paper}

\begin{abstract} 
\italic{MiniLaTeX is a no-setup subset of LaTeX that can be rendered on the fly to HTML.   One can use it to build web apps with true HTML display viewable on any device from smart phone to tablet to desktop.  Typesetting occurs in real time, and error messages are displayed in-line in the rendered text. MiniLaTeX documents can be exported to standard LaTeX. 

\medskip
We describe (a) MiniLaTeX the language, (b) the main features of the document management application
\href{https://minilatex.lamdera.app}{minilatex.lamdera.app}, and (c) some of the technical work required to implement an on-the-fly LaTeX-to-Html compiler.}
\end{abstract}



\section{Introduction}

 MiniLaTeX is a no-setup subset of LaTeX that comes with an on-the-fly compiler to HTML:

\begin{itemize}

\item \strong{No setup.}  Just begin typing.  No preamble with \code{\bs{usepackage}} statements, \code{\bs{begin}\texarg{document}}, etc., is needed.


\item \strong{On-the-fly.}  A typical editor for MiniLaTeX presents two windows.  On the left is the source text, on the right is the rendered text.  Changes to the source text are immediately reflected in the rendered text window.

\item \strong{Errors.} Errors are immediately flagged in color in place in the rendered text.


\end{itemize}

The compiler is written in \href{https://elm-lang.org}{Elm}, a strictly typed language of pure functions.  Elm is a good language for writing MiniLaTeX apps because (a) it is designed for building web applications and (b) it has an excellent, high performance  \href{https://package.elm-lang.org/packages/elm/parser/latest/}{library of parser combinators} akin to Haskell's \code{parsec}.


\subheading{Apps}

Below are two links to  apps that use  MiniLaTeX.  

\begin{itemize}

\item \strong{\href{https://demo.minilatex.app}{Demo.minilatex.app}}: a simple no-signin app for experimenting with MiniLaTeX.  Feel free to edit the text you find there, or clear it and write something new.

\item \strong{\href{https://minilatex.lamdera.app/g/34}{Minilatex.lamdera.app/g/34}}:  This link takes you to some class notes
hosted on \href{https://minilatex.lamdera.app}{minilatex.lamdera.app}.

\end{itemize}

\href{https://minilatex.lamdera.app}{Minilatex.lamdera.app}, while still in alpha test, is a full content-management system for creating, editing, and distributing MiniLaTeX documents.

\begin{itemize}

\item \strong{Guest access.} Sign in to \href{https://minilatex.lamdera.app}{minilatex.lamdera.app}
as \blue{guest} with password \blue{minilatex} to explore documents that have been made public by their authors.  Both the source and rendered text are available, so you can see how MiniLaTeX documents are written. 

\item \strong{Registered user.} Create and edit documents on a desktop computer or tablet. Works in read-only mode on smart phones.

\item \strong{Smart phone.} Use in read-only mode on a smart phone. Students can read class notes and problem sets this way.

\end{itemize}


\section{Features of MiniLaTeX}

Below is a summary of MiniLaTeX features. See \href{https://minilatex.lamdera.app/g/21}{the manual} for more details.

\begin{itemize}

\item \strong{Macros and Environments.} Environments include theorem, problem, definition, etc., as well as equation, align, verbatim, and more.  The \code{\bs{maketitle}}, \code{\bs{section}}. \code{\bs{cite}} and \code{\bs{eqref}} macros and many others work as expected.  Unimplemented macros are rendered verbatim, but colored red to indicate their status.

\item \strong{Macro definitions.} One can define both math-mode and text-mode macros in MiniLaTeX.  

\item \strong{Export.} MiniLaTeX documents can be exported to standard LaTeX, complete with the necessary \code{\bs{usepackage}} statements, \code{\bs{begin}\texarg{document}}, \code{\bs{end}\texarg{document}}, and any needed macro definitions.  Exported documents are ready to process with \code{\bs{pdflatex}}.  Here is \href{https://noteimages.s3.amazonaws.com/anharmonic_oscillator.pdf}{an example} of an exported document compiled to PDF using TeXShop.


\item \strong{Images} The macro \code{\bs{image}\texarg{URL}\texarg{{Caption}\texarg{Format}} is used to place images in a MiniLaTeX document.  
Provision is made to render images in exported documents.  There is, in addition, an \code{svg} environment for rendering SVG images from SVG source code. 

\item \strong{Unicode.} MiniLaTeX accepts unicode (UTF-8) input. More needs to be done to accomodate unicode in exported documents.

\item \strong{Paragraph-centric.} MiniLaTeX is \blue{paragraph-centric}, meaning that the smallest unit of recompilation is the \italic{logical paragraph}.  A logical paragraph is either an ordinary paragraph or an outer begin-end block.  

\item \strong{Additions.} 
Besides the \code{\bs{image}} command, there are various other commands that exist in MiniLaTeX but not plain LaTeX.  These commands are provide with suitable macro definitions on export so that a MiniLaTeX document can always be compiled with standard LaTeX Tools.  Here are some examples.  Text can be colored blue using the \code{\bs{blue}} macro: \italic{I am feeling \blue{blue}}.  Text can be highlighted using the \code{\bs{highlight}} macro.  The teacher said that \highlight{all work on our class project is due by October 1}.
There is also a \code{\bs{strike}} macro: Please delete this \strike{very bad word}.  For a complete list of additions, see \href{https://minilatex.lamdera.app/g/21}{minilatex.lamdera.app/g/21}.



\end{itemize}

 An important part of the MiniLaTeX project is to properly define the subset of LaTeX to be supported.  The current rule of thumb for this is "good enough to write my lecture notes, class hand-outs, and problem sets."  Feedback on this issue is much appreciated (jxxcarlson at gmail).


\section{Some features of minilatex.lamdera.app}

The content management system \href{https://minilatex.lamdera.app}{minilatex.lamdera.app} provides a system for creating, editing, and distributing documents from a searchable repository.  Users can search by title, author, etc.  Documents can be collaboratively edited, shared by URL, and versioned on Github through a simple user interface.  Documents can also be exported to PDF.


In addition to the familiar \code{\bs{href}} macro, there are \code{\bs{xlink}}  and \code{\bs{ilink}}.  The \code{\bs{xlink}} macro 
is used to make a link from one document to another in \code{minilatex.lambdera.app}.   Thus, one can say 
\code{\bs{xlink}\texarg{21}\texarg{Manual}}, which is rendered as \xlink{21}{Manual}, to make a link to that document.

The \code{\bs{ilink}} macro is similar.  It has the same syntax, but is used in constructing a page of links which functions as a table of contents or \italic{index document}. In this way, one can assemble many documents into one to make a book.  For an example, see \href{https://minilatex.lamdera.app/g/34}{these class notes}. It is worthwhile looking at the source to see how it is done.





\section{The MiniLaTeX Compiler}

The MiniLaTeX compiler consists of two parts, a \italic{parser} and a \italic{renderer}.  The first is a function

$$
parse: \text{Source text} \to \text{AST},
$$

where \italic{AST} stands for \italic{Abstract Syntax Tree}.  This is a tree with nodes like 

\begin{verbatim}
    LXString "Pythagoras says"
\end{verbatim} 

and

\begin{verbatim}
    InlineMath "a^2 + b^2 = c^2"
\end{verbatim}

 that expresses a grammatical analysis of the source text, identifying its "parts of speech" and putting them in relationship to one another.
The second is a function

$$
render: \text{AST} \to \text{HTML}.
$$

The compiler is the composite of these two functions:

$$
compile   = render \circ parse
$$

The strategy that makes writing such a compiler feasible is \italic{divide and conquer}.  One writes the parser using parser combinators.  One then constructs function which renders the text-mode LaTeX to HTML, passing the math-mode text on to either \href{https://mathjax.org}{MathJax} or \href{https://katex.org}{KaTeX} for rendering. 

\strong{Video}: \href{https://www.youtube.com/watch?v=dmDA7iziSgs&t=15s}{Making a LaTeX-to-Html parser in Elm}

An abstract syntax tree is a tree of things that expresses a grammatical analysis of the source text. It is akin to what one gets when one diagrams a sentence in English class.




In designing a parser, one generally starts with a grammar.  However, there is no published grammar for LaTeX or for TeX, so one has to improvise.  The approach taken with MiniLaTeX was first to imagine the type of the AST, then to write a parser using parser combinators.  The parser then consists of a bunch of functions which call one another.  They are in rough correspondence with the productions of a grammar, and so one can write down a grammar after the fact. Ideally one should have written the grammar, then the parser, but real life is often messy.

\subsection{AST}

Every value in a statically typed language like Haskell, ML, or Elm, has a \italic{type}.  Below is the type of AST for the MiniLaTeX compiler.  Writing down this type definition was the first step in writing the parser.

\begin{listing}
type LatexExpr
    = LXString String
    | Comment String
    | Item Int LatexExpression
    | InlineMath String
    | DisplayMath String
    | SMacro String (List LatexExpr) (List LatexExpr) LatexExpr
    | Macro String (List LatexExpr) (List LatexExpr)
    | Environment String (List LatexExpr) LatexExpr
    | LatexList (List LatexExpr)
    | NewCommand String Int LatexExpr
    | LXError (List (DeadEnd Context Problem))
\end{listing}

Note that the definition of \code{LatexExpr} refers to itself, hence is recursive.  This is typical of the definitions of types of structured trees. 
To give an idea of how the parser works, consider the following  examples.  In each, the first line is the source text, the others constitute the resulting AST.

\begin{verbatim}
> Pythagoras
LXString "Pythagoras"

> \strong{Pythagoras}
Macro "strong" [] [LatexList [LXString "Pythagoras"]]

> \strong{Pythagoras} says that $a^2 + b^2 = c^2$
Macro "strong" [] [LatexList [LXString "Pythagoras"]]
  , LXString "says  that "
  , InlineMath "a^2 + b^2 = c^2"
\end{verbatim}


The parser takes up roughly 500 lines of Elm code and consists of a set of functions which call upon one another.  The top-level parser function is given below.  Note the close correspondence between its parts and the parts of the type definition.  It is built using the \code{oneOf} parser, a combinator which takes a list of parsers as arguments and which returns a parser as value.

Below is the top-level parser function.  Note the close correspondence
with the AST type.  The \code{lazy} function is needed for recursion.  The \code{oneOf} function is a combinator that takes a list of parrsers and produces a new parser from them.  The resulting parser operates as follows: try the first parser.  If it succeeds, return the result.  If not, try the next parser.  And so on.  If the list is exhausted without success, return an error.

\begin{listing}
latexExpression : LXParser LatexExpr
latexExpression =
    oneOf
        [ texComment
        , displayMathDollar
        , displayMathBrackets
        , inlineMath
        , newcommand
        , macro
        , smacro
        , words
        , lazy (\_ -> environment)
        ]
\end{listing}

We give one more example, the \code{macro} parser.  It uses the combinators \code{(|=)} and \code{(|.)} to \italic{sequence} parsers thereby formin a new one.  The resulting "parser pipeline" operates  in the following way.   The phrase \code{|= macro} parses the name of the macro.  Then \code{|= itemList optionalArg} recognizes the list of optional arguments, and \code{|= itemList arg} does the same for the regular macro arguments.  Finally, \code{|. whitespace} "eats" but ignores whatever white space it finds.  The values  found by the \code{(|=)} phrases are taken as arguments of  the constructor \code{Macro} for the \code{LatexExpr} type, and the result of this function call is a value of type \code{LatexExpr}.  

In this example, \code{whitespace} is a parser for white space, which can come in different flavors depending on context e.g., spaces only or spaces and newlines.

\begin{listing}
macro : LXParser () -> LXParser LatexExpr
macro whitespace =
    succeed Macro
        |= macroName
        |= itemList optionalArg
        |= itemList arg
        |. whitespace
\end{listing}

One can continue down the rabbit hole, explaining how the parsers \code{macroName}, \code{optionalArg},  \code{arg} and the combinator \code{itemList} work.  But we stop here.  What is important to understand is that there are really only three things in something like the MiniLaTeX parser: primitive parsers, combinators that choose among alternatives, and combinators that sequence other parsers. As a note, it is important to eat trailing whitespace because lexing and parsing are not separate operations in this set-up.

\subsection{Rendering}


The top-level renderiing function is much like the top-level parsing function. It analyzes the type of a LatexExpr and dispatches the appropriater renderer, which may in turn call other rendering functions, including the top level one. And so on, down the rabbit hole of function calls we go.


\subsection{Code}

Code for the MiniLaTeX compiler as well as the demo app is found at \href{https://github.com/jxxcarlson/meenylatex}{github.com/jxxcarlson/meenylatex}.  The strange name is to reserve the name \blue{github.com/jxxcarlson/minilatex} for a future stable version with a polished API.   For the minilatex.lamdera.app, the code is at \href{https://github.com/jxxcarlson/lamdera-minilatex-app}{github.com/jxxcarlson/lamdera-minilatex-app}.
Both repositories are open source. 

\section{Feedback}

I am very interested in feedback from the community regarding features, bugs, etc.  Of special interest is the subset of LaTeX used: what should it be? Comments to jxxcarlson at gmail.



\end{enumerate}

Comments to jxxcarlson at gmail.

\section{Acknowledgements}

I would like to thank Evan Czaplicki, Ilias Van Peer, Mario Rogic, and Luke Westby, all of the Elm community, Davide Cervone, \href{https://mathjax.org}{MathJax.org}, and the team at \href{https://katex.org}{Katex.org} for their generous and invaluable help.   I also wish to thank the \href{https://simonsfoundation.org}{Simons Foundation} for its support of this project.

\italic{This document was written in MiniLaTeX and is available at \href{https://minilatex.lamdera.app/g/22}{minilatex.lamdera.app/g/22}.}



