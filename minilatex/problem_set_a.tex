beginMetadata:
{
    "id": "dad130ab-0b76-4fed-9a3c-7b68b1466545",
    "documentNumber": 49,
    "author": "jxxcarlson",
    "title": "Problem Set A",
    "path": "minilatex/problem_set_a.tex",
    "tags": [],
    "keyString": "jxxcarlson minilatex/problem_set_a.tex",
    "timeCreated": 1597401444731,
    "timeModified": 1597403114877,
    "public": false,
    "collaborators": [],
    "docType": "miniLaTeX",
    "versionNumber": 2,
    "versionDate": 1597403229874
}
endMetadata
\uuid{dad130ab-0b76-4fed-9a3c-7b68b1466545}
\title{Problem Set A}
\author{Phineas Fogg, Ph.D}
\date{July 25, 2020}


\maketitle

% This is an example of how one might set up a problem set.
% The idea is that a student can copy the problem set
% and fill in the solution in the space provided.  We've given
% two examples.  In problem 1, the solution is given.  In problem 2,
% the student is to filll in the solution. Thus the template can be used
% both for problem assignments that are distributed to students
% and for solutions which the students submit for grading.

\begin{problem}
Suppose that $P = (1,2)$ and $Q = (5,-3)$.  What is the distance from $P$ to $Q$?
\end{problem}

\begin{solution}
We use the Pythagorean theorem, which says that the distance between two points is

\begin{equation}
d = \sqrt{(\Delta x)^2 + (\Delta y)^2}
\end{equation}

where $\Delta x$ is the difference between the $x$-values and $\Delta y$ is the difference between the $y$-values.  Thus

\begin{align}
d &= \sqrt{(1 - 5)^2 + (5 - (-3))^2} \\
  &= \sqrt{4^2 + 8^2} \\
  &= \sqrt{80} = 4\sqrt{5}
\end{align}

\end{solution}

\begin{problem}
Consider the path that connects the points $A = (1,0)$,
$B = (2,1)$, $C = (1,3)$, $D = (4,3)$.

\medskip
\begin{itemize}
\item Draw this path
\item Compute its length
\end{itemize}
\end{problem}

\begin{solution}
Please put your solution here.
\end{solution}

\attachNote{Phineas}{Hey Matt, would you check this out and add a couple more problems?}
