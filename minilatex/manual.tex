\title{MiniLaTeX Manual}

\maketitle

\tableofcontents


\section{Browser compatibiity}

The app works well in Chrome, Firefox, Opera, and Min.   In Safari, the display is messed up.

\section{Images}

You can place an image in a MiniLaTeX document if it it exists somewhere on the internet and you have its "locaion" or URL.  The URL for the image below is

\begin{verbatim}
https://psurl.s3.amazonaws.com/images/jc/beats-eca1.png
\end{verbatim}

You place an image flike this:

\begin{verbatim}
\image{URL}{Caption}{Format}
\end{verbatim}

The caption and format are optional.  The format has two parts, both of which are optional: the \blue{width} and the \blue{placement}. The width is a fragment like \blue{width: 350}.  Placement can be one of the following:

\begin{verbatim}
align: center
float: left
float: right
\end{verbatim}

\image{https://psurl.s3.amazonaws.com/images/jc/beats-eca1.png}{Figure 1. Two-frequency beats}{width: 350, align: center}

\section{Fonts}

MiniLaTeX accepts unicode (UTF-8) input.  Here, for example, is Arabic text,  taken from \href{https://www.ltg.ed.ac.uk/~richard/unicode-sample.html}{www.ltg.ed.ac.uk}:

، ؛ ؟ ء آ أ ؤ إ ئ ا ب ة ت ث ج ح خ د ذ ر ز س ش ص ض ط ظ ع غ ـ ف ق ك ل م ن ه و ى ي ً ٌ ٍ َ ُ ِ ّ ْ ٠ ١ ٢ ٣ ٤ ٥ ٦ ٧ ٨ ٩ ٪ ٫ ٬ ٭ ٰ ٱ ٲ ٳ ٴ ٵ ٶ ٷ ٸ ٹ ٺ ٻ ټ ٽ پ ٿ ڀ ځ ڂ ڃ ڄ څ چ ڇ ڈ ډ ڊ ڋ ڌ ڍ ڎ ڏ ڐ ڑ ڒ ړ ڔ ڕ ږ ڗ ژ ڙ ښ ڛ ڜ ڝ ڞ ڟ ڠ ڡ ڢ ڣ ڤ ڥ ڦ ڧ ڨ ک ڪ ګ ڬ ڭ ڮ گ ڰ ڱ ...


\section{Problem Sets}

Take a look at \xlink{60}{Problem Set A} for a template for problem sets that you can use in your classes.

\section{Links}

\subsection{External links}

Use the \code{\bs{href}} macro, e.g,

\begin{verbatim}
\href{https://nytimes.com}{New York Times}
\end{verbatim}

to link to the New York Times.

\subsection{Sharing documents}

To share a public document with someone, imitate this example:

\begin{verbatim}
minilatex.lamdera.app/g/21
\end{verbatim}

The  fragment \code{/g/} stands for \italic{guest}.  Using this link will take you to document 21, and you will be automatically logged in as guest.


\subsection{Cross links}

You can make link in one MiniLaTeX document that points to another.
Here is a \xlink{58}{an example}. It is done like this:

\begin{verbatim}
\xlink{58}{an example}
\end{verbatim}

The first argument, 58 in this case, is the numerical ID of the document in question. The document ID is displayed in the footer of this app.

\subsection{Index links}

One can use the \code{\bs{ilink}} macro to construct a kind of index of links that refer to other documents.  For an example of how this is done, please see \xlink{34}{these class notes} and look at the source. For best results, put the tag \italic{index} in the tag field (editor, upper right).  Then press the \blue{Tags} button.

\section{Collaboration}

\subsection{Adding and removing collaborators}

Press the \blue{Collaborators} button in the footer to raise the collaborators popup list.  Enter collaborators by user name one per line.


\subsection{Notes to collaborators}

If you say this:

\begin{verbatim}
\editnote{HD}{Hey, this is a good first draft.
But I would refer to Blowenpuff's article on
quantum computing in the introduction.}
\end{verbatim}

It is rendered like this:

\editnote{HD}{Hey, this is a good first draft.
But I would refer to Blowenpuff's article on
quantum computing in the introduction.}

\section{Defining Macros}

\subsection{Math mode}

If we put the text

\begin{verbatim}
\begin{mathmacro}
\newcommand{\bbU}[0]{\mathbb{U}}
\newcommand{\mca}[0]{\mathcal{A}}
\end{mathmacro}
\end{verbatim}


\begin{mathmacro}
\newcommand{\bbU}[0]{\mathbb{U}}
\newcommand{\mca}[0]{\mathcal{A}}
\end{mathmacro}

Then saying the below

\begin{verbatim}
\mca_0^1}= \bbU
\end{verbatim}

\subsection{Math mode}

in display math mode yields this:

$$
\mca_0^1 = \bbU(1)
$$


\subsection{Text mode}

\begin{textmacro}
\newcommand{\boss}{Phineas Fogg}
\newcommand{\hello}[1]{Hello \strong{#1}!}
\newcommand{\reverseconcat}[3]{#3#2#1}
\end{textmacro}


If you say this:

\begin{verbatim}
\begin{textmacro}
\newcommand{\boss}{Phineas Fogg}
\newcommand{\hello}[1]{Hello \strong{#1}!}
\newcommand{\reverseconcat}[3]{#3\ #2\ #1}
\end{textmacro
\end{verbatim}

The you can say that my boss is \boss. Oh, \hello{Maria}, and you can do fun things like

\begin{center}
he jumped over = \reverseconcat{he}{jumped }{over }
\end{center}

\subsection{Importing}

Put the line

\begin{verbatim}
  \include{2}
\end{verbatim}

as the first line of your document to import the contents of document 2.  The text will be inserted before the regular text.  This is a good way to include macro definitions.  To prepend the text of several documents, you can do this:

\begin{verbatim}
  \include{2, 3, 4}
\end{verbatim}

\section{Differences and Limitations}

\subsection{Macros}

MiniLaTeX has some macros that are not part of out-of-the-box LaTeX. Most of these, however, work with standard LaTeX given suitable macro definitions and packages. These are the \italic{exportable} macros.

\subsubsection{Exportable macros}

\begin{enumerate}

\item \code{\bs{blue}}, e.g., \blue{I am feeling blue today}

\item \code{\bs{red}}, e.g, the soup is \red{hot}.

\item \code{\bs{highlight}}, e.g, \highlight{this is an important message.}

\item \code{\bs{strike}}, e.g, this is \strike{mspelled} misspelled.

\item \code{\bs{image}} See the section on images above.

\item \code{\bs{ilink}},  \code{\bs{xlink}}: See the section on links

\item  \code{\bs{strong}},  \code{\bs{italic}}.  These do what you think they do.   \code{\bs{code}} renders its argument in a monospace font.

\item  \code{\bs{bs}}.  Make a backslash preceding the argument, eg, this is a \bs{command}.  Use \code{\bs{texarg}} for arguments of macros. Example: \code{\bs{strike}\texarg{TEXT}}

\item  \code{\bs{dollar}} makes a dollar sign that won't mess things up, e.g, the groceries cost  a lot \dollar\dollar!

\item \code{\bs{innertableofcontents}} Used to make an internal table of contents.  On export, it does not render.

\item \code{\bs{editnote}\texarg{AUTHOR}\texarg{COMMENT}}.  Write a comment in a document. Used for collaboration.

\item \code{\bs{include}} Used in MiniLaTeX to include things like macro definitions.  See section on macros.

\item \code{\bs{mathmacro}}, \code{\bs{textmacro}}. See section on macro definitions.


\end{enumerate}

\subsubsection{All macros}

Some explanations, others where needed are forthcoming

\begin{indent}
author,
bigskip,
blue,
bs,
cite,
code,
date,
dollar,
ellie,
email,
emph,
eqref,
highlight,
href,
ilink,
image,
imageref,
include,
index,
innertableofcontents,
italic,
label,
maintableofcontents,
maketitle,
mdash,
medskip,
meta,
ndash,
percent,
red,
ref,
revision,
section,
section*,
setcounter,
smallskip,
strike,
texarg,
strong,
subheading,
subsection,
subsection*,
subsubsection,
subsubsection*,
tableofcontents,
term,
title,
underscore,
xlink
\end{indent}

\subsection{Environments}

\subsubsection{Regular environments}

\begin{indent}
align,
center,
comment,
defitem,
enumerate,
eqnarray,
equation,
indent,
itemize,
listing,
mathmacro,
quotation,
svg,
tabular,
textmacro,
thebibliography,
verbatim,
verse,
\end{indent}

\subsubsection{Theorem-like environments}

\begin{indent}
corollary
definition
lemma
problem
proposition
theorem
\end{indent}



\subsection{Symbols}

MiniLaTeX can work with either MathJax or KaTeX.  Currently we are using KaTeX.  This
\href{https://katex.org/docs/support_table.html}{table of supported symbols} may be helpful.


\section{Exporting to LaTeX}

Use the \blue{Export} button in the footer.  The exported document will be saved in the Downloads folder unless you tell your computer otherwise.  If your document has images, a button \blue{Images} will appear to the right of \blue{Export}.  Click on that button to bring up a window that lists the images in your document.  Right-click on these images to download them.  They should be stored in folder \blue{image} in the same place as your exported document.


\section{Known Issues}

We are working on these.  Please let us know of others as you find them (jxxcarlson at gmail).

\begin{enumerate}

\item Links in table of contents do not work

\end{enumerate}

\section{Plans}


\begin{enumerate}

\item Better search and sorting.  Coming soon!

\item An image uploader which will store images you create on the web and give you URLs for them.  That way you will be able to place any image in a MiniLaTeX document if you can create it on your competer.

\end{enumerate}
