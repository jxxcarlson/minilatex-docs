beginMetadata:
{
    "id": "58a4a6be-4fca-45cf-a05a-9481be61df79",
    "documentNumber": 0,
    "author": "jxxcarlson",
    "title": "About MiniLaTeX",
    "path": "minilatex/about.tex",
    "tags": [
        "minilatex",
        "manual"
    ],
    "keyString": "about minilatex a=jxxcarlson minilatex/about.tex t=minilatex t=manual",
    "timeCreated": 1595082002906,
    "timeModified": 1598635454359,
    "public": true,
    "collaborators": [],
    "docType": "miniLaTeX",
    "versionNumber": 0,
    "versionDate": 0
}
endMetadata


\title{About MiniLaTeX}


\begin{mathmacro}
\newcommand{\bt}[1]{\bf{#1}}
\newcommand{\mca}[0]{\mathcal{A}}
\end{mathmacro}

\begin{textmacro}
\newcommand{\boss}{Phineas Fogg}
\newcommand{\hello}[1]{Hello \strong{#1}!}
\newcommand{\reverseconcat}[3]{#3#2#1}
\end{textmacro}

\maketitle

% EXAMPLE 1

\tableofcontents

\section{Introduction}



MiniLatex is a subset of LaTeX that can be
rendered live in the browser using a custom just-in-time compiler written in \href{https://elm-lang.org}{Elm}.
Mathematical text is rendered using \href{https://katex.org}{KaTeX}:

$$
\int_{-\infty}^\infty e^{-x^2} dx = \pi
$$



\strong{Guest access.} Feel free to
experiment with MiniLatex using this app.  If you signed in as 
\blue{guest} (using password \blue{minilatex}), you can browse
the document list, use the search feature, and also look at
the source text by clicking the \blue{Source} button in the footer.


\strong{User access.} If you have created an account and  are signed in, you can create and edit documents.  Documents are saved every 400 milliseconds, so you don't have to worry about doing that manually. 

\strong{Sharing documents.} Every document has a numerical ID which is displayed in the footer.  You can use it to share documents that you make public. Here is how you share document 78:

\begin{verbatim}
https://minilatex.lamdera.app/g/78
\end{verbatim} 

Just give that URL to some one using email, pencil and paper, or a class handout.  Or use it in a web page, \href{https://minilatex.lamdera.app/g/78}{like this}.

\strong{Exporting docments.} Use the \blue{export} button below to export the text you write to a
LaTeX document on your computer. Exported documents can
be processed as-is by any program that runs LaTeX,
e.g, TeXShop or \code{pdflatex}.  


For questions or more information about
the MiniLaTeX project, please  write to jxxcarlson at gmail.  In the remainder of this document, we give examples of MiniLaTeX text and discuss features at greater length.

\section{Text}

MiniLaTeX using \code{\bs{strong}} for bold text and \code{\bs{italic}} for italic text.

\section{Formulas}


The most basic integral:

\begin{equation}
\label{integral:xn}
\int_0^1 x^n dx = \frac{1}{n+1}
\end{equation}

An improper integral:

\begin{equation}
\label{integral:exp}
\int_0^\infty e^{-x} dx = 1
\end{equation}

\section{Theorems}

\begin{theorem}
There are infinitely many prime numbers.
\end{theorem}

\begin{theorem}
There are infinitley many prime numbers
$p$ such that $p \equiv 1\ mod\ 4$.
\end{theorem}

\section{Images}

Images in MiniLaTeX are accessed by URL (see the example
in section 4 below). When you export a document, images
used in it will be listed to the right
of the rendered text.  To use them in the exported
document, right (option) click on the image and
save it in a folder named \italic{image}.

\image{http://psurl.s3.amazonaws.com/images/jc/beats-eca1.png}{Figure 1. Two-frequency beats}{width: 350, align: center}

\section{SVG}

\begin{svg}
<svg version="1.1" xmlns="http://www.w3.org/2000/svg" xmlns:xlink="http://www.w3.org/1999/xlink" x="0" y="0" width="381.603" height="250.385" viewBox="0, 0, 381.603, 250.385">
  <g id="Layer_1">
    <g>
      <path d="M95.401,166.09 L71.5,124.692 L95.401,83.295 L143.203,83.295 L167.103,124.692 L143.202,166.09 z" fill="#CCDD10"/>
      <path d="M95.401,166.09 L71.5,124.692 L95.401,83.295 L143.203,83.295 L167.103,124.692 L143.202,166.09 z" fill-opacity="0" stroke="#000000" stroke-width="1"/>
    </g>
    <g>
      <path d="M166.401,126.09 L142.5,84.692 L166.401,43.295 L214.203,43.295 L238.103,84.692 L214.202,126.09 z" fill="#0D9B53"/>
      <path d="M166.401,126.09 L142.5,84.692 L166.401,43.295 L214.203,43.295 L238.103,84.692 L214.202,126.09 z" fill-opacity="0" stroke="#000000" stroke-width="1"/>
    </g>
    <g>
      <path d="M167.401,207.885 L143.5,166.487 L167.401,125.09 L215.203,125.09 L239.103,166.487 L215.202,207.885 z" fill="#0D9B53"/>
      <path d="M167.401,207.885 L143.5,166.487 L167.401,125.09 L215.203,125.09 L239.103,166.487 L215.202,207.885 z" fill-opacity="0" stroke="#000000" stroke-width="1"/>
    </g>
    <g>
      <path d="M309.401,209.885 L285.5,168.487 L309.401,127.09 L357.203,127.09 L381.103,168.487 L357.203,209.885 z" fill="#0D9B53"/>
      <path d="M309.401,209.885 L285.5,168.487 L309.401,127.09 L357.203,127.09 L381.103,168.487 L357.203,209.885 z" fill-opacity="0" stroke="#000000" stroke-width="1"/>
    </g>
    <g>
      <path d="M309.401,125.09 L285.5,83.692 L309.401,42.295 L357.203,42.295 L381.103,83.692 L357.203,125.09 z" fill="#0D9B53"/>
      <path d="M309.401,125.09 L285.5,83.692 L309.401,42.295 L357.203,42.295 L381.103,83.692 L357.203,125.09 z" fill-opacity="0" stroke="#000000" stroke-width="1"/>
    </g>
    <g>
      <path d="M23.401,126.09 L-0.5,84.692 L23.401,43.295 L71.203,43.295 L95.103,84.692 L71.203,126.09 z" fill="#0D9B53"/>
      <path d="M23.401,126.09 L-0.5,84.692 L23.401,43.295 L71.203,43.295 L95.103,84.692 L71.203,126.09 z" fill-opacity="0" stroke="#000000" stroke-width="1"/>
    </g>
    <g>
      <path d="M237.401,84.295 L213.5,42.897 L237.401,1.5 L285.203,1.5 L309.103,42.897 L285.203,84.295 z" fill="#CCDD10"/>
      <path d="M237.401,84.295 L213.5,42.897 L237.401,1.5 L285.203,1.5 L309.103,42.897 L285.203,84.295 z" fill-opacity="0" stroke="#000000" stroke-width="1"/>
    </g>
    <g>
      <path d="M239.401,249.885 L215.5,208.487 L239.401,167.09 L287.203,167.09 L311.103,208.487 L287.203,249.885 z" fill="#CCDD10"/>
      <path d="M239.401,249.885 L215.5,208.487 L239.401,167.09 L287.203,167.09 L311.103,208.487 L287.203,249.885 z" fill-opacity="0" stroke="#000000" stroke-width="1"/>
    </g>
    <g>
      <path d="M94.401,84.295 L70.5,42.897 L94.401,1.5 L142.203,1.5 L166.103,42.897 L142.202,84.295 z" fill="#CCDD10"/>
      <path d="M94.401,84.295 L70.5,42.897 L94.401,1.5 L142.203,1.5 L166.103,42.897 L142.202,84.295 z" fill-opacity="0" stroke="#000000" stroke-width="1"/>
    </g>
    <g>
      <path d="M333.302,94.897 C327.411,94.897 322.635,90.328 322.635,84.692 C322.635,79.056 327.411,74.487 333.302,74.487 C339.193,74.487 343.968,79.056 343.968,84.692 C343.968,90.328 339.193,94.897 333.302,94.897 z" fill="#D60B8E"/>
      <path d="M333.302,94.897 C327.411,94.897 322.635,90.328 322.635,84.692 C322.635,79.056 327.411,74.487 333.302,74.487 C339.193,74.487 343.968,79.056 343.968,84.692 C343.968,90.328 339.193,94.897 333.302,94.897 z" fill-opacity="0" stroke="#000000" stroke-width="1"/>
    </g>
  </g>
</svg>
\end{svg}

\section{Cross links}

You can make link in one MiniLaTeX document that points to another.
Here is a \xlink{58}{an example}. It is done like this:

\begin{verbatim}
\xlink{58}{an example}
\end{verbatim}

The first argument, 58 in this case, is the numerical ID of the document in question. The document ID is displayed in the footer of this app.

\section{Lists}

\begin{itemize}

\item This is \strong{just} a test.

\item \italic{And so is this:} $a^2 + b^2 = c^2$

\begin{itemize}

\item Items can be nested

\item And you can do this: $ \frac{1}{1 + \frac{2}{3}} $

\end{itemize}

\end{itemize}

\section{Tables}

\begin{indent}
\begin{tabular}{ l l l l }
Hydrogen & H & 1 & 1.008 \\
Helium & He & 2 & 4.003 \\
Lithium& Li & 3 & 6.94 \\
Beryllium& Be& 4& 9.012 \\
\end{tabular}
\end{indent}

\section{Math-mode macros}

Math-mode macros are added using the \code{mathmacro} environment:

\begin{verbatim}
\begin{mathmacro}
\newcommand{\bt}[1]{\bf{#1}}
\newcommand{\mca}[0]{\mathcal{A}}
\end{mathmacro}
\end{verbatim}

Then saying

\begin{verbatim}
 $$
 a^2 = \bt{Z}/\mca
 $$
\end{verbatim}

yields

$$
a^2 = \bt{Z}/\mca
$$

Likewise, saying

\begin{verbatim}
\begin{equation}
\label{eq:function.type}
\mca^{\bt{Z}} = \bt{Z} \to \mca
\end{equation}
\end{verbatim}

yields

\begin{equation}
\label{eq:function.type}
\mca^{\bt{Z}} = \bt{Z} \to \mca
\end{equation}

There are some restrictions:

\begin{verbatim}
1. No regions, e.g, use \bf{#1},
   not {\bf #1}

2. Macros with no arguments:
   Say \newcommand{\mca}[0]{\mathcal{A}},
   not \newcommand{\mca}{\mathcal{A}}

3. No recursion: the body of the macro
   cannot refer to other macros defined
   in the mathmacro environment.

4. Put the mathmacro environment at
   the beginning of the document
\end{verbatim}

Items 1—3 will be eliminated in a
future release.

\section{Text-mode Macros}

Text-mode macros are defined in a \code{textmacro} environment:

\begin{verbatim}
\begin{textmacro}
\newcommand{\boss}{Phineas Fogg}
\newcommand{\hello}[1]{Hello \strong{#1}!}
\newcommand{\reverseconcat}[3]{#3#2#1}
\end{textmacro}
\end{verbatim}

Then

\begin{verbatim}
\italic{My boss is \boss.}
\end{verbatim}

produces \italic{My boss is \boss.}
Likewise, the text

\begin{verbatim}
\hello{John}
\end{verbatim}

yields \hello{John}.

\section{MiniLatex Macros}

MiniLatex has a number of macros of its own,  For
example, text can be rendered in various colors, \red{such as red}
and \blue{blue}. Text can \highlight{be highlighted}
and can \strike{also be struck}. Here are the macros:

\begin{verbatim}
\red
\blue
\highlight
\strike
\end{verbatim}

\section{Errors and related matters}

Errors, as illustrated below, are rendered in real time and are reported in red, in place.
For example, suppose you type the  text

\begin{verbatim}
  $$
  a^2 + b^2 = c^2
\end{verbatim}

Then you will see this in the rendered text window:

\image{http://jxxcarlson.s3.amazonaws.com/miniLaTeXErrorMsg-2020-02-22.png}{Fig 2. Error message}{width: 200}

We plan to make further improvements in error reporting.
