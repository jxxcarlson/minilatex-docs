beginMetadata:
{
    "id": "c4e74b13-7c26-47b0-8a9f-d6eac31c408c",
    "documentNumber": 16,
    "author": "jxxcarlson",
    "title": "MiniLaTeX: The App",
    "path": "minilatex/app.tex",
    "tags": [],
    "keyString": "minilatex: the app a=jxxcarlson minilatex/app.tex ",
    "timeCreated": 1598484797310,
    "timeModified": 1598484840308,
    "public": true,
    "collaborators": [],
    "docType": "miniLaTeX",
    "versionNumber": 3,
    "versionDate": 1598485027304
}
endMetadata
\title{MiniLaTeX: The App}

\maketitle

\tableofcontents


\section{Browser compatibiity}

The app works well in Chrome, Firefox, Opera, and Min.   In Safari, the display is messed up.

\section{Home page}

Tag a document as  \italic{home} and make it public.  It will then appear in the list of home pages that you see when you click on the button \blue{Home pages} (header, upper right).  If your username is \code{aristotle}, you can send someone a link to your home page like this:

\begin{verbatim}
https://minilatex.lamdera.app/h/aristotle
\end{verbatim}

Your home page can have text, images, math, etc., as well as links to other documents on this site.  Use the \code{\bs{xlink}} macro fo this:

\begin{verbatim}
\xlink{30}{Magical Thinking}

\xlink{uuid:abc12...34de}{More Magical Thinking}
\end{verbatim}

The firs links to document 30, the second to the document with Uuid \code{uuid:abc12...34de}.  The document number of the current document is displayed in the footer.  To find the Uuid of  document, click on the button \blue{Version} in the footer. 

\section{Collaboration}

\subsection{Adding and removing collaborators}

Press the \blue{Collaborators} button in the footer to raise the collaborators popup list.  Enter collaborators by user name one per line.


\subsection{Notes to collaborators}

If you say this:

\begin{verbatim}
\attachNote{HD}{Hey, this is a good first draft.  
But I would refer to Blowenpuff's article on 
quantum computing in the introduction.}
\end{verbatim}

It is rendered like this:

\attachNote{HD}{Hey, this is a good first draft.  
But I would refer to Blowenpuff's article on 
quantum computing in the introduction.}

\section{Search}

Shortcut: use \code{:book} to search for "books," aka collections of documents.


\subsection{Importing}

Put the line 

\begin{verbatim}
  \include{2}
\end{verbatim}

as the first line of your document to import the contents of document 2.  The text will be inserted before the regular text.  This is a good way to include macro definitions.  To prepend the text of several documents, you can do this:

\begin{verbatim}
  \include{2, 3, 4}
\end{verbatim}

\section{Exporting to LaTeX} 

Use the \blue{Export} button in the footer.  The exported document will be saved in the Downloads folder unless you tell your computer otherwise.  If your document has images, a button \blue{Images} will appear to the right of \blue{Export}.  Click on that button to bring up a window that lists the images in your document.  Right-click on these images to download them.  They should be stored in folder \blue{image} in the same place as your exported document.


\section{Known Issues}

We are working on these.  Please let us know of others as you find them (jxxcarlson at gmail).

\begin{enumerate}

\item Links in table of contents do not work

\end{enumerate}


\end{enumerate}
