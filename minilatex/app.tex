\title{MiniLaTeX: The App}

\maketitle

\tableofcontents


\section{Browser compatibiity}

The app works well in Chrome, Firefox, Opera, and Min.   In Safari, the display is messed up.



\section{Collaboration}

\subsection{Adding and removing collaborators}

Press the \blue{Collaborators} button in the footer to raise the collaborators popup list.  Enter collaborators by user name one per line.


\subsection{Notes to collaborators}

If you say this:

\begin{verbatim}
\attachNote{HD}{Hey, this is a good first draft.  
But I would refer to Blowenpuff's article on 
quantum computing in the introduction.}
\end{verbatim}

It is rendered like this:

\attachNote{HD}{Hey, this is a good first draft.  
But I would refer to Blowenpuff's article on 
quantum computing in the introduction.}

\section{Search}

Shortcut: use \code{:book} to search for "books," aka collections of documents.


\subsection{Importing}

Put the line 

\begin{verbatim}
  \include{2}
\end{verbatim}

as the first line of your document to import the contents of document 2.  The text will be inserted before the regular text.  This is a good way to include macro definitions.  To prepend the text of several documents, you can do this:

\begin{verbatim}
  \include{2, 3, 4}
\end{verbatim}

\section{Exporting to LaTeX} 

Use the \blue{Export} button in the footer.  The exported document will be saved in the Downloads folder unless you tell your computer otherwise.  If your document has images, a button \blue{Images} will appear to the right of \blue{Export}.  Click on that button to bring up a window that lists the images in your document.  Right-click on these images to download them.  They should be stored in folder \blue{image} in the same place as your exported document.


\section{Known Issues}

We are working on these.  Please let us know of others as you find them (jxxcarlson at gmail).

\begin{enumerate}

\item Links in table of contents do not work

\end{enumerate}


\end{enumerate}
