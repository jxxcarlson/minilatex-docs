



\section{Quantum harmonic oscillator at temperature T}



Consider a quantum-mechanical harmonic oscillator with frequencye $\nu$ and energy $E = h\nu$.  The expected energy of the oscillator is computed from the Boltzmann distribution:

\begin{equation}
\bar{E} = \frac{\sum jE e^{-jE\beta}}{\sum e^{-jE\beta}}
\end{equation}

The partition function is

\begin{equation}
Z = \sum e^{-jE\beta}
\end{equation}

and its derivative is


\begin{equation}
\frac{dZ}{d\beta} =  - \sum jE e^{-jE\beta}
\end{equation}

from which it follows that 

\begin{equation}
\bar{E} = - \frac{dZ/d\beta}{Z} = - \frac{d}{d\beta} \log Z
\end{equation}

Applying the geometric formula, we find that

\begin{equation}
Z = \frac{1}{1 - e^{-E\beta}},
\end{equation}

from which it follows that

\begin{equation}
\bar{E} = \frac{E}{e^{E/kT} - 1}.
\end{equation}


or, better,


\begin{equation}
\bar{E} = \frac{h\nu}{e^{h\nu/kT} - 1}.
\end{equation}





