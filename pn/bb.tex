



\section{Blackbody radiation}

\innertableofcontents


\subsection{Derivation of the radiation law}

Consider electromagnetic radiation confined to box of size $L\times L \times L$. The functions

\begin{equation}
\phi(x,y,z,t) = \sin(k_1x)\sin(k_2y)\sin(k_3z)e^{\pm i\omega t}
\end{equation}

form a basis of solutions to the wave equation

\begin{equation}
\nabla \phi = \frac{1}{c^2} \frac{\partial^2 \phi}{\partial t^2}
\end{equation}

Substituting the basic solution into the wave equation, we find that 

\begin{equation}lo
\omega = c||k|| 
\end{equation}

Since $\phi$ vanishes on the walls of the box, $k_iL = n_i\pi$ for
integers $n_i$, we have

\begin{equation}
\omega = \frac{c\pi}{L} ||n||
\end{equation}

Let $N(\omega)$ denote the number of modes with angular frequency less than or equal to $\omega$.  Then


\begin{equation}
N(\omega) = \#\Big\{ n\ :\ ||n|| \le \frac{\omega L}{\pi c} \Big\}
\end{equation}


Since the mode vectors $n$ lie in the first octant, we have

\begin{equation}
N(\omega) = \frac{1}{8}\frac{4}{3}\pi\left(\frac{\omega L}{\pi c}\right)^3
 = \frac{\omega^3 V }{6c^3 \pi^2}
\end{equation}


or, in terms of frequency,

\begin{equation}
N(\nu) = \frac{4\pi\nu^3 V }{c^3}
\end{equation}


Adjusting for the two polarizations of photons, this becomes

\begin{equation}
N(\nu) = \frac{8\pi\nu^3 V }{c^3}
\end{equation}

Now consider the mode density

\begin{equation}
g(\nu) =\frac{ dN}{d\nu} =  \frac{8\pi\nu^3 V }{c^3}
\end{equation}

The energy density is the mode density time $kT$ divided by the volume:

\begin{equation}
\rho(\nu) = \frac{8\pi\nu^3 kT }{c^3}
\end{equation}


This is the \term{Rayleigh-Jeans Law}.  Note that if $\nu \to \infty$, then
$\rho(\nu) \to \infty$.  This the \term{ultraviolet catastrophe}, a nonsensical result that implies that there is something wrong with classical physics.  Planck's found a resolution of the ultraviolet catastrophe by assuming that the radiation in the cavity cannot assume a continuum of energy states, but rather is quantized.  In more detail, the radiation in the cavity is modelled by an infinite set of quantum harmonic oscillators with , with fundamental frequencies states $\omega_n = \hbar c\pi ||n||/L$, where $n$ is the mode vector.  Equivalently, the fundamental frequency is $\nu_n = h c\pi ||n||/L$.  The energy density is

\begin{equation}
\rho(\nu) = \frac{g(\nu)}{V}\bar E(\nu) 
  = \frac{8\pi\nu^2 }{c^3} \frac{h\nu}{e^{h\nu/kT} - 1}
\end{equation}


That is,

\begin{equation}
\rho(\nu) = \frac{8\pi }{c^3} \frac{h\nu^3}{e^{h\nu/kT} - 1}
\end{equation}

In terms of angular frequency, one has

\begin{equation}
\rho(\omega) = \frac{1}{\pi^2 c^3} \frac{\bar h\omega^3}{e^{h\nu/kT} - 1}
\end{equation}

\image{https://s3.amazonaws.com/noteimages/jxxcarlson/648px-Wiens_law_svg.png}{Blackbody spectrum}{align: center, width: 400}


\subsection{Fundamental constants}

By fitting the theoretical curve for the radiation law to experimental data, one can estimate both Planck's constant $\hbar$ and the Boltzmann constant $k$ in the the equipartion formula 

$$
E = \frac{1}{2} kT
$$

These constants are

$$ k = 1.380 \times 10^{-23}\ \text{J}\cdot\text{K}^{-1}$$

and 

$$ h = 6.626 \times 10^{-34}\ \text{J}\cdot\text{s}^{-1}$$

\subsection{Cosmic background radiation}

Fitting microwave spectrometer measurements of the cosmic backround radiation to the theoretical blackbody intensity-frequency curve, one finds a blackbody temperature of 2.725 K. 

\image{https://s3.amazonaws.com/noteimages/jxxcarlson/CMBspect.png}{}{align: center, width: 400}


\subsection{References}

Planck's law for the energy density of the radiation eld (M.
Planck, Verh. Deutsch. Phys. Ges. 2, 202 and 237 (1900)

\href{https://www.chemie.unibas.ch/~tulej/Spectroscopy_related_aspects/Lecture7_Spec_Rel_Asp.pdf}{Blackbody radiation
derivation of Planck‘s radiation law}

\href{http://www.astro.ucla.edu/~wright/CMB.html}{Cosmic Microwave Background (UCLA Astro)}

\href{http://hyperphysics.phy-astr.gsu.edu/hbase/bkg3k.html}{Cosmic background radiation at hyperphysics}



