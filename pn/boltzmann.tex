beginMetadata:
{
    "id": "ee2d21e8-6e35-4d88-9acb-c62bb0064780",
    "documentNumber": 7,
    "author": "jxxcarlson",
    "title": "Boltzmann distribution",
    "path": "pn/boltzmann.tex",
    "tags": [],
    "keyString": "jxxcarlson pn/boltzmann.tex",
    "timeCreated": 1598451316663,
    "timeModified": 1598451316663,
    "public": true,
    "collaborators": [],
    "docType": "miniLaTeX",
    "versionNumber": 2,
    "versionDate": 1598451385658
}
endMetadata
\xlink{uuid:8df12037-c8a4-4e22-be40-4939e94bd9e9}{Physics Notes}

\setcounter{section}{3}

\section{Boltzmann distribution}

Consider a system with energy levels $E_j$ which
are occupied with probabiity $p_j$. We shall
derive the Boltzmann distribution

\begin{equation}
p_j= \frac{e^{-E_j/kT}}{Z},
\end{equation}

for such a system, where

\begin{equation}
Z = \sum e^{-  E_j/kT}.
\end{equation}

is the \term{partition} function.
For the derivation, we assume that (1) energy is conserved,

\begin{equation}
\label{eq:conservedEnergy}
 \sum p_kE_k = E
\end{equation}

and that (2), the entropy

\begin{equation}
S = - k \sum p_k \log p_k
\end{equation}

is maximal.  The second condition yields the differential relation

\begin{equation}
dS = - k \sum (1 +\log p_k)  dp_k = 0
\end{equation}


Since the sum of the $p_k$ is 1, we have the "conservation of probablity" relation

\begin{equation}
\label{eq:consprob}
\sum dp_k = 0
\end{equation}

The previous equation therefore reduces to

\begin{equation}
\label{eq:diffEntropy}
 dS = -k \sum \log p_k\,  dp_k = 0
\end{equation}



The differential form of the conservation of energy equation is

\begin{equation}
\label{eq:dffConservedEnergy}
 \sum E_k\, d p_k = 0
\end{equation}

Now  equation \eqref{eq:diffEntropy} must be a linear combination of \eqref{eq:consprob}
and \eqref{eq:conservedEnergy}, so that $\log p_k = \alpha - \beta E_k$
for some constants $\alpha$ and $\beta$.  Substitute this relation in to \eqref{eq:diffEntropy} to obtain

\begin{align}
 dS & = -k \sum \log p_k\,  dp_k \\
      & = -k \sum (\alpha - \beta E_k) dp_k \\
      & = k\beta \sum E_k dp_k \\
     & = k\beta \sum dq_i \\
    & = k\beta dq_{rev},
\end{align}

where in the second to the last step we use the fact that $dq_k = E_k dp_k$  is the  amount of heat energy exchanged when the
probability of occupying state $k$ is changed by $dp_k$.  Thus we have

$$
dS = k\beta dq_{rev}
$$


But from classical thermodynamics, we also have $dS = dq_{rev}/T$.  Combining these equations, we find that

\begin{equation}
\beta = \frac{1}{kT}
\end{equation}

From $\log p_k = \alpha - \beta E_k$, we find that

\begin{equation}
p_k = e^\alpha e^{-E_k/kT}
\end{equation}

Summing this equation over all $k$ yields

\begin{equation}
1 = \sum p_k = e^\alpha \sum e^{-E_k/kT}
\end{equation}

Thus $e^\alpha = 1/Z$, where is

\begin{equation}
Z = \sum e^{-E_k/kT}
\end{equation}

is the partition function defined above.


\subheading{References}

\href{http://casegroup.rutgers.edu/lnotes/ccb341/boltzmann.pdf}{Notes on the Boltzmann distribution, Chem 341}

\href{http://www.physics.udel.edu/~glyde/PHYS813/Lectures/chapter_3.pdf}{The Method of the Most Probable Distribution}
