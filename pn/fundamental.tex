\section{Fundamental units}


\innertableofcontents

\subsection{Energy as fundamental unit}
In the fundamental system of  physical units, $c = \hbar = 1$. Because $c = 1$, time and length have the same units: $[L] = [T]$. The quantity $\hbar$ is
an action, that is, $\text{energy}\times \text{time}$, from which we conclude that $[E] = [T]^{-1}$.  Consequently, in the fundamental sytem, all units are powers of $[E]$.  For example, from $E = mc^2$, we find that $[E] = [m]$. 
\medskip

\subsection{Planck Energy}

Let us apply this to the gravitational constant.  Newton's law of universal gravitation states that 
\medskip

\begin{equation}
F = \frac{GMm}{r^2}
\end{equation} 


Now $F  = ma$, so for the left-hand side above, $[F] = [m][L][T]^{-2}$, or $[F] = [E]^2$.  Then


\begin{equation}
[F] = [G][E]^2[L]^{-2} = [G][E]^4
\end{equation}


Therefore 

\begin{equation}
[G] = [E]^{-2}
\end{equation}

It follows that there is an energy, $E_{pl}$, such that 

\begin{equation}
G = \frac{1}{E_{pl}^2}
\end{equation}



\begin{equation}
E_{pl} = \frac{1}{\sqrt G} 
\end{equation}


\subsection{Planck Energy in MKS units}

First, we know that $c = 3\times 10^8 m/s$ and that $h = 6.626\times 10^{-34}\; \text{Joule-s}$, so that 
$$
\hbar c = 3.164\times 10^{-26}
$$

Then, in MKS units, 

$$
E_{pl} = \sqrt \frac{\hbar c}{G}
$$

so that $E_{pl} = 2.177 \times 10^{-8}\; \text{kg}$. Convert this result to Joules by multiplying by $c^2$ to get $E_{pl} = 1.95\times 10^9\; \text{Joules}$.
Now  one electron volt is $1.6 \times 10^{-19}\; \text{Joules}$, so finally

\begin{equation}
  E_{pl} \sim 1.22\times 10^{19}\; \text{GeV}
\end{equation}


To get an idea of how large the Planck energy is, compare it with  the rest energy of some fundamental particles in the table below (the Planck mass is, so far as we know, imaginary).  It is rougly nineteen orders of magnitude greater than the rest energy of the proton.  The "length scale" is simple the length $L = 1/E$ associated to the energy.  To write the length scale in the MKS system, we use the conversion factor $\hbar c$, so that $L = \hbar c/E$, where $E$ is the rest energy in Joules.

\subsection{Masses}

\begin{indent}


\begin{tabular}
\strong{Object} & \strong{Mass} & \strong{Energy} & \strong{Length scale} \\
Electron & $9.109\times10^{-31}\; \text{kg}$ & 0.512 MeV & $4.86\times 10^{-13}\; \text{m}$ \\
Proton & $1.673\times10^{-27}\; \text{kg}$ & 0.941 GeV & $2.10\times 10^{-16}\; \text{m}$ \\
Higgs boson & $2.240 \times 10^{-25}\; \text{kg}$ & 126 GeV & $1.56 \times 10^{-18}\; \text{m}$ \\
Planck mass & $22 \mu g$ & $1.22\times 10^{19}\; \text{GeV}$ & $1.62\times 10^{-35}\; \text{m}$\\
\end{tabular}

\end{indent}


NOTE: The radius of a proton is 0.84 fm, comparable to to the length scale of the proton in the above table.



\subsection{References}

\begin{enumerate}

\item \href{https://newt.phys.unsw.edu.au/einsteinlight/jw/module6_Planck.htm}{About the Planck Scale}

\item \href{http://newton.ex.ac.uk/research/qsystems/collabs/constants.html}{Physical constants}

\item \href{https://www.math.upenn.edu/~deturck/m425/m425-dalembert.pdf}{Wave equation}

\end{enumerate}


