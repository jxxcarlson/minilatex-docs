beginMetadata:
{
    "id": "47a38f1e-9446-4e51-845a-cce1ed7b4077",
    "documentNumber": 365,
    "author": "jxxcarlson",
    "title": "Images",
    "path": "images.tex",
    "tags": [
        "minieditor"
    ],
    "keyString": "images a=jxxcarlson images.tex t=minieditor",
    "timeCreated": 1607270345882,
    "timeModified": 1607270364880,
    "public": true,
    "collaborators": [],
    "docType": "miniLaTeX",
    "versionNumber": 0,
    "versionDate": 0
}
endMetadata
\xlink{uuid:f8e68aa3-429a-4030-83a8-328ecb41126d}{Learning MiniLaTeX}


\setcounter{section}{7}

\section{Images}

You can place an image in a MiniLaTeX document if it it exists somewhere on the internet and you have its "location" or URL.  The URL for the image below is

\begin{verbatim}
https://psurl.s3.amazonaws.com/images/jc/beats-eca1.png
\end{verbatim}

You place an image like this:

\begin{verbatim}
\image{URL}{Caption}{Format}
\end{verbatim}

The caption and format are optional.  The format has two parts, both of which are optional: the \blue{width} and the \blue{placement}. The width is a fragment like \blue{width: 350}, where the units are pixels.  Placement can be one of the following:

\begin{verbatim}
align: center
float: left
float: right
\end{verbatim}

\image{https://psurl.s3.amazonaws.com/images/jc/beats-eca1.png}{Figure 1. Two-frequency beats}{width: 350, align: center}



\strong{Exercise.} Write source text for the \italic{Two-frequency beats} in the MiniEditor.

\strong{Exercise.} Find an image on the internet and place it  in the MiniEditor.