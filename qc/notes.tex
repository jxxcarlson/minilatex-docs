\title{Notes on Quantum Computing}

\maketitle


\section{References}

\href{https://www.technologyreview.com/2019/05/30/65724/how-a-quantum-computer-could-break-2048-bit-rsa-encryption-in-8-hours/}{How a quantum computer could break 2048-bit RSA encryption in 8 hours}, MIT Technology Review, May 19, 2019

\href{https://arxiv.org/abs/1905.09749}{
How to factor 2048 bit RSA integers in 8 hours using 20 million noisy qubits}, Craig Gidney, Martin Ekerå, Arxiv.og

\href{https://www.quintessencelabs.com/blog/breaking-rsa-encryption-update-state-art/}{Breaking RSA Encryption – an Update on the State-of-the-Art}, Andreas Baumhof, Quintessence Labs, May 19, 2019.

\href{https://www.scientificamerican.com/article/new-encryption-system-protects-data-from-quantum-computers/}{New Encryption System Protects Data from Quantum Computers}, Sophie Bushwick, Scientific American, October 8, 2019

\href{https://qudev.phys.ethz.ch/static/content/QSIT15/Shors%20Algorithm.pdf}{Shor's Algorithm}, ETH, slides


\href{https://riliu.math.ncsu.edu/437/notes3se4.html}{Shor's Algorithm}, ncsu.edu. Notes by Ricky Ini Liu \highlight{very good!}


In 1994, Peter Shor introduced an algorithm for factoring large numbers into primes using a quantum computer.  
